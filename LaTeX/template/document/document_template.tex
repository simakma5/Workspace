%% Verze pro jednostranný tisk:
% Okraje: levý 40mm, pravý 25mm, horní a dolní 25mm
% (ale pozor, LaTeX si sám přidává 1in)
%\documentclass[11pt,a4paper]{report}
%\setlength\textwidth{145mm}
%\setlength\textheight{247mm}
%\setlength\oddsidemargin{15mm}
%\setlength\evensidemargin{15mm}
%\setlength\topmargin{0mm}
%\setlength\headsep{0mm}
%\setlength\headheight{0mm}
% \openright zařídí, aby následující text začínal na pravé straně knihy
%\let\openright=\clearpage

%% Pokud tiskneme oboustranně:
%\documentclass[11pt,a4paper,twoside,openright]{report}
%\setlength\textwidth{145mm}
%\setlength\textheight{247mm}
%\setlength\oddsidemargin{14.2mm}
%\setlength\evensidemargin{0mm}
%\setlength\topmargin{0mm}
%\setlength\headsep{0mm}
%\setlength\headheight{0mm}
%\let\openright=\cleardoublepage

%% Jelikož netiskneme vůbec:
\documentclass[11pt,a4paper]{report}
\usepackage[a4paper,hmargin=1in,vmargin=1in]{geometry}

%% Vytváříme PDF/A-2u - nefunguje, nevim proč
%\usepackage[a-2u]{pdfx}

% Toto makro definuje kapitolu, která není očíslovaná, ale je uvedena v obsahu.
\def\chapwithtoc#1{
	\chapter*{#1}
	\addcontentsline{toc}{chapter}{#1}
}

\usepackage[czech]{babel}
\usepackage[utf8]{inputenc}
\usepackage[T1]{fontenc}

\usepackage{stddoc}

\usepackage{multicol}

\newlist{multienum}{enumerate}{1}
\setlist[multienum]{
	label=\alph*),
	before=\begin{multicols}{2},
		after=\end{multicols}
}

\newlist{multiitem}{itemize}{1}
\setlist[multiitem]{
	label=\textbullet,
	before=\begin{multicols}{2},
		after=\end{multicols}
}

\usepackage{lipsum}

\title{Template document}
\author{Martin Šimák}
\date{-being-constantly-updated-}

%%% ways to explicitly command the table of contents to show only some headlines
%\setcounter{tocdepth}{1} % Show sections
%\setcounter{tocdepth}{2} % + subsections
%\setcounter{tocdepth}{3} % + subsubsections
%\setcounter{tocdepth}{4} % + paragraphs
%\setcounter{tocdepth}{5} % + subparagraphs

\newtheorem{theorem}{Theorem}[section]
% The additional parameter [section] restarts the theorem counter at every new section.
\newtheorem{corollary}{Corollary}[theorem]
% An environment called corollary is created, the counter of this new environment will be reset every time a new theorem environment is used.
\newtheorem{lemma}[theorem]{Lemma}
% In this case, the even though a new environment called lemma is created, it will use the same counter as the theorem environment.
\theoremstyle{remark}
\newtheorem*{remark}{Remark}
% The syntax of the command \newtheorem* is the same as the non-starred version, except for the counter parameters. In this example a new unnumbered environment called remark is created.

\theoremstyle{definition}
\newtheorem{definition}{Definition}[section]

%\renewcommand\qedsymbol{$\blacksquare$} % černej čtvereček



\begin{document}
	
	\pagenumbering{gobble} % possible pagenumbering options: gobble, arabic, roman
	
	\maketitle
	
	\begin{abstract}
		K balíčkům: babel s parametrem czech rovnou zavádí i česká jména pro oddíly jako např. \textcolor{blue}{Abstrakt}, \textcolor{blue}{Obsah}, atd. Taky přibyde příkaz \textbackslash uv\{$\cdot$\} vytvářející české uvozovky: \uv{Ahoj.}
		
		\lettrine{
			\gothfamily\fontsize{50}{60}\scalebox{2}{M}{}
		}{artin:}
		\lipsum[1]
	\end{abstract}
	
	\newpage

	\tableofcontents % include a table of contents here
	\newpage
	
	\pagenumbering{arabic}
	
	\section{This is a section}
	\subsection{This is a subsection}
	\subsubsection{This is a subsubsection}
	\paragraph{This is a paragraph}
	\subparagraph{This is a subparagraph}
	
	\section*{This is a section*}
	\subsection*{This is a subsection*}
	\subsubsection*{This is a subsubsection*}
	\paragraph*{This is a paragraph*}
	\subparagraph*{This is a subparagraph*}
	
	\section*{Fonts and formats}
		
		\subsubsection*{Paper formats and font sizes}
			We usually define the paper size and the font size inside the square brackets []. The point size can be described in the way [10pt]. The other font sizes are 8pt, 9pt, 10pt, 11pt, 12pt, 14pt, 17pt, 20pt. The default font size for Latex is 10pt.
		
			\begin{minipage}{0.5\textwidth}
				\textbf{Paper types with their dimesions:}
				\begin{itemize}
					\item letterpaper (11 x 8.5 in)
					\item legalpaper (14 x 8.5 in)
					\item a5paper (5.8 x 8.3 in)
					\item a4paper (8.3 x 11.7 in)
					\item executivepaper (10.5 x 7.25 in)
					\item b5paper (25 x 17.6 cm)
				\end{itemize}
			\end{minipage}
			\begin{minipage}{0.5\textwidth}
				\textbf{Types of sizes:}
				\begin{itemize}
					\item \textbf{\{\textbackslash tiny\}:} \begin{tiny} hello \end{tiny}
					\item \textbf{\{\textbackslash scriptsize\}:} \begin{scriptsize} hello \end{scriptsize}
					\item \textbf{\{\textbackslash footnotesize\}:} \begin{footnotesize} hello \end{footnotesize}
					\item \textbf{\{\textbackslash small\}:} \begin{small} hello \end{small}
					\item \textbf{\{\textbackslash normalsize\}:} \begin{normalsize} hello \end{normalsize}
					\item \textbf{\{\textbackslash large\}:} \begin{large} hello \end{large}
					\item \textbf{\{\textbackslash Large\}:} \begin{Large} hello \end{Large}
					\item \textbf{\{\textbackslash LARGE\}:} \begin{LARGE} hello \end{LARGE}
					\item \textbf{\{\textbackslash huge\}:} \begin{huge} hello \end{huge}
				\end{itemize}
			\end{minipage}
			
			\noindent
			Example:
			\begin{center}  
				\begin{huge}  
					\textbf{The \LaTeX\ Technical Institute}\\  
				\end{huge}  
				\begin{LARGE}  
					\textsc{\\Certification}   
				\end{LARGE}    
				
			\end{center}  
			\noindent This to certify that Mr. R.P Sharma has undergone a course in this institute and is qualified to be Technical Expert.  
			
			\begin{flushright}  
				\textsf{The Director}\\  
				The \LaTeX\ Technical Institute  
			\end{flushright} %used for aligning to the right 
		
		\subsubsection*{Font styles}
			
			The styles are categorized into family, series, and shape. The typestyle in the output is composed of these three characteristics.
			
			\begin{minipage}{0.5\textwidth}
				\textbf{The table for the styles is given below:}
				\begin{itemize}
					\item \textrm{Roman}
					\item \texttt{Typewriter}
					\item \textsf{Sans serif}
				\end{itemize}
				\textbf{The table for the series is given below:}
				\begin{itemize}
					\item \textbf{boldface}
					\item \textmd{medium}
				\end{itemize}
			\end{minipage}
			\begin{minipage}{0.5\textwidth}
				\textbf{The table for the shape is given below:}
				\begin{itemize}
					\item \textit{italic}
					\item \textsl{slanted}
					\item \textup{upright}
					\item \textsc{small cap}
				\end{itemize}
			\end{minipage}
			
			\noindent
			Example 1:
			\textit{\textbf{ the topic discussed is Latex.}}  
			\textrm{\textsl{ it contains the process and characteristics}}
			
			\noindent
			Example 2:
			\textit{\textbf{ the topic discussed is Latex.}\newline}  
			\textrm{\textsl{ it contains the process and characteristics}}  
			
			\noindent
			\textit{Comment:} \quad \textbf{\textbackslash emph\{\}} Causes the paramater word to switch to italic when used in the middle of a straight text and vice versa.
			
			\noindent
			Example 3:
			\textit{ a figure with six sides is called \emph{hexagon} and with five sides is called \emph{pentagon}.}
			
			\noindent
			Example 4:
			\textbf{ a figure with six sides is called \emph{hexagon} and with five sides is called \emph{pentagon}.}
			
		\subsubsection*{Document formatting}
			
			\noindent
			\textbf{\textbackslash noindent} causes any applied indentation to go away for the next line\\
			\textbf{\textbackslash flushright (\textbackslash flushleft)} switches the indentation to the passed side
	
	\section*{Text color}
		
		\textcolor{purple}{Každý} \textcolor{cyan}{slovo} \textcolor{green}{jinak} pomocí \textcolor{red}{\textbackslash textcolor\{$\cdot$\}}
	
	\section*{Color box}
		
		\colorbox{green!40!white}{Základní colorbox, pohodka lahodka, ale bez rounded corners. :(}
		
		\begin{tcolorbox}[
			rounded corners,
			colback=pink!75!white,
			colframe=white,
			arc=1mm,
			text width=\textwidth
			]
			\color{black} V dokumentu s červenými referencemi, barva pink!75!white\\
				\lipsum[1]
		\end{tcolorbox}
	
		\begin{tcolorbox}[
			rounded corners,
			colback=blue!25!white,
			colframe=white,
			arc=1mm,
			text width=\textwidth
			]
			\color{black} V dokumentu s modrými referencemi, barva blue!25!white\\
			\lipsum[1]
		\end{tcolorbox}
		
		\noindent
		{ \hfil \noindent
			\fcolorbox{black}{gray}{
				{\vspace{.5cm}
					\begin{minipage}{0.15\textwidth}
						\begin{center}
							\Huge $T$ \\ $E_k$
						\end{center}
					\end{minipage}
					\hfil
					\fcolorbox{black}{white}{
						\begin{minipage}{0.5\textwidth}
							\paragraph{Kinetická energie}
							systému o hmotnosti $m$ a rychlosti $\vec v$ je definována coby práce, kterou je potřeba vykonat, abychom systém urychlili z klidu na rychlost $\vec v$ a je vždy dána vztahem
							\begin{align*}
							T := W(0 \to \vec v) = \frac 12 m \|\vec v\|^2 = \frac 12 mv^2.
							\end{align*}
							Kinetickou energii můžeme také značit $E_k$ či $A$.
					\end{minipage}}
					\vspace{.5cm}
			}}
			\hfil
		}
	
	\section*{Basic syntax}
	
		\subsection*{Some typesetting}
		
			\noindent
			\lettrine{W}{hen} you append an asterisk (*) to the end of the environment's name, you allow yourself to align the environment by the symbol you highlight by ampersand (\&). That's not true. Adding an asterisk to an environment's name means that you don't want that part to be enumerated \\
			This is some example text\footnote{\label{myfootnote}Hello footnote}
			$( \big( \Big( \bigg( \Bigg($
			\begin{align*}
				f(x) =& x^2\! +3x\! +2,
			\\
				f(x) =& x^2+3x+2,
			\\
				f(x) =& x^2\, +3x\, +2,
			\\
				f(x) =& x^2\: +3x\: +2,
			\\
				f(x) =& x^2\; +3x\; +2,
			\\
				f(x) =& x^2\ +3x\ +2,
			\\
				f(x) =& x^2\quad +3x\quad +2,
			\\
				f(x) =& x^2\qquad +3x\qquad +2.
			\end{align*}
			
		\subsection*{Combining text and math}
		
			Měřením jsme zjistili hodnotu měrného náboje elektronu
			\begin{equation*}
				\frac{e}{m_e} = (1.7 \pm 0.21) \cdot 10^{11} \; \frac{\text C}{\text{kg}} \, ,
			\end{equation*}
				kde hodnota za znakem $\pm$ udává nejistotu měření určenou metodou redukce. \\
				Tabulková hodnota měrného náboje elektronu je
			\begin{equation*}
				\frac{e}{m_e} = (1.75882012 \pm 0.00000015) \cdot 10^{11} \; \frac{\text C}{\text{kg}} \, ,
			\end{equation*}
				odchylka naměřených hodnot od hodnot tabulkových tak činí 3,78 \%. Nesrovnalosti s tabulkovými hodnotami byly nejspíše způsobeny nepřesnostmi námi provedeného měření.
			
		\subsection*{The delta-epsilon definition of limit}
		
			\begin{align*}
				\forall \epsilon > 0\; \exists \delta > 0\; \forall x \in \mathbb R : |x - x_0| < \delta \implies |f(x) - f(x_0)| < \epsilon.
			\end{align*}
		
		\subsection*{Hyperref}
		
			Used for referencing to an equation. Get some hello \ref{myfootnote}
			\begin{align*}
				\label{eq:particni-funkce}
				\tag{kuchevnik}
				Z = \frac{1}{(2\pi \h)^6}\int_{\mathbb R^6} e^{-\beta \mathcal H(\vec x, \vec p)} \: \d \vec x \, \d \vec p.
			\end{align*}
			Co kurva za jméno je \eqref{eq:particni-funkce}? \\
			Na \eqref{eq:particni-funkce} můžu clicknout odkudkoliv. \\
			Tyvole vždyť \eqref{eq:gauss-law} je Gaussův integrální zákon.
		
		\subsection*{Random theorem}
			
			\begin{theorem}[Chain rule]
				Let $g$ be a function that is differentiable at a point $c$ (i.e. the derivative $g'(c)$ exists) and $f$ is a function that is differentiable at $g(c)$, then the composite function $f \circ g$ is differentiable at $c$, and the derivative is
				\begin{align}
					(f \circ g)'(c) = f'(g(c)) \cdot g'(c).
				\end{align}
			\end{theorem}
			
			\begin{proof}[Proof]
				Proof being fucking straightforward, we just do it directly:
				\begin{align*}
					(f \circ g)'(c) &= \lim\limits_{x \to c} \frac{f(g(x)) - f(g(c))}{x-c} = \lim\limits_{x \to c} \frac{f(g(x)) - f(g(c))}{g(x) - g(c)} \cdot \frac{g(x) - g(c)}{x-c} =
				\\
					&= \lim\limits_{g(x) \to g(c)} \frac{f(g(x)) - f(g(c))}{g(x) - g(c)} \cdot \lim\limits_{x \to c} \frac{g(x) - g(c)}{x-c} = f'(g(c)) \cdot g'(c),
				\end{align*}
				where in the prelast equation, we've considered the fact that the function is \textit{differentiable at $c$}, thus is \textit{continuous at $c$}. Therefore $x \to c$ iff $g(x) \to g(c)$.
			\end{proof}
					
		\subsection*{Math typesetting}
		
			We don't use '$\times$' very much. 'Tis usually better not to write anything or '$\cdot$' in a product.
			\begin{align}
				\pi &\approx 3.14,
			\\
				f(x) &= \frac{x_0^{i+1}}{\sqrt[a]{\arctan(\omega t)}},
			\\
				F(x) &= \int_a^b e^x \: \d x = \left. e^x \right|_a^b = e^b - e^a,
			\\
				g(n) &= \left\{ \begin{matrix}
							n/2, &\text{pokud } n \text{ je liché} \\
							n/2-1, &\text{pokud } n \text{ je sudé}
						\end{matrix} \right.,
			\\
				\(\!
					\begin{array}{c}
						n \\
						k
					\end{array}
				\)\! &= \frac{n!}{(n-k)! \, k!},
			\\
				\sum_{n = 0}^{\infty} \frac{z^n}{n!} &= e^z \text{; kde } z \in \mathbb C \text{, } n \, {\not \in} \, \mathbb{C} \text{, } n \in \mathbb R,
			\\
				\prod_{i = 1}^N x_i &= x_1 \cdot x_2 \cdot x_3 \cdot \: \dots \, \cdot x_N,
			\\
				R_i{}^j{}_{kl} &= g^{jm} R_{imkl} = - g^{jm} R_{mikl} = - R^j{}_{ikl},
			\\
				\lim_{x \to \infty} \( 1 + \frac{a}{x} \)^x &= e^a,
			\\
				\label{eq:gauss-law}
				\tag{Géčko}
				\Aboxed{\int_V \mathrm{div} \, \vec F \: \d V &= \int_{S = \partial(V)} \vec F \cdot \d \vec S,}
			\\
				\mathbb R &\subset \mathbb C,
			\\
				\mathbb A &= \mathbb B \iff \mathbb A \subseteq \mathbb B \; \land \; \mathbb{A} \supseteq \mathbb B,
			\\
				\|f\| &= \inf \{ K \in \langle 0, +\infty) : |f(x)| \leq K \|x\|, \, \forall x \in \mathbb X \}.
			\\
				a &=\joinrel=\joinrel=\joinrel= b
			\end{align}
			
		\subsection{Random Hermitian adjoint}
			
			\begin{align}
			\( \begin{matrix}
					v_1 \\
					v_2 \\
					\vdots \\
					v_n
				\end{matrix} \)^\dagger \coloneqq \( v_1^*, v_2^*, \dots, v_n^* \)
			\end{align}
		
		\subsection{Simulace Newtonových rovnic}
		
			\noindent
			Na začátku máme
			\begin{align*}
				\vec F(t) &= \frac{\d \vec p}{\d t}(t) = \lim_{\delta t \to 0} \frac{m(t + \delta t) - m(t)}{\delta t} \vec v(t) + m(t) \lim_{\delta t \to 0} \frac{\vec v(t + \delta t) - \vec v(t)}{\delta t},
			\\
				\vec v(t) &= \lim_{\delta t \to 0} \frac{\vec x(t + \delta t) - \vec x(t)}{\delta t}.
			\end{align*}
			
			\noindent
			Teď provedu diferenci $\vec x_n = \vec x(t_n)$
			\begin{align*}
				\vec x(t + \delta t) &= \vec x(t) + \vec v(t) \delta t,
				&
				\vec x_{n+1} &= \vec x_n + \vec v_n \delta t,
			\\
				\vec v(t + \delta t) &= \vec v(t) + \frac{\vec F(t)}{m}\delta t,
				&
				\vec v_{n+1} &= \vec v_n + \frac{\vec F_n}{m} \delta t.
			\end{align*}
		
		\section*{Amsthm}
			
			\begin{theorem}
				Let $f$ be a function whose derivative exists in every point, then $f$ is 
				a continuous function.
			\end{theorem}
			
			\begin{theorem}[Pythagorean theorem]
				\label{pythagorean}
				This is a theorem about right triangles and can be summarised in the next 
				equation 
				\begin{align*}
					x^2 + y^2 = z^2
				\end{align*}
			\end{theorem}
			
			And a consequence of theorem \ref{pythagorean} is the statement in the next 
			corollary.
			
			\begin{corollary}
				There's no right rectangle whose sides measure 3cm, 4cm, and 6cm.
			\end{corollary}
			
			You can reference theorems such as \ref{pythagorean} when a label is assigned.
			
			\begin{lemma}
				Given two line segments whose lengths are $a$ and $b$ respectively there is a 
				real number $r$ such that $b=ra$.
			\end{lemma}
			
			Unnumbered theorem-like environments are also posible.
			
			\begin{remark}
				This statement is true, I guess.
			\end{remark}
			
			And the next is a somewhat informal definition
			
			\theoremstyle{definition}
			\begin{definition}{Fibration}
				A fibration is a mapping between two topological spaces that has the homotopy lifting property for every space $X$.
			\end{definition}
			
			\begin{lemma}
				Given two line segments whose lengths are $a$ and $b$ respectively there 
				is a real number $r$ such that $b=ra$.
			\end{lemma}
			
			\begin{proof}
				To prove it by contradiction try and assume that the statement is false,
				proceed from there and at some point you will arrive to a contradiction.
			\end{proof}
		
		\section*{Environments}
		
			\noindent
			Let's try some of the enviroments...
						
			\subsection*{itemize}
			
				\noindent
				Itemize enviroment is used to list items.
				\begin{itemize}
					\item Zdroj pro napájení Helmholtzových cívek
					
					\item Regulátor napětí
					
					\item Omezovač proudu
					
					\item Ampérmetr pro měření proudu Helmholtzovými cívkami; $\Delta \text{I} = 0,001$ A
					
					\item Helmholtzovy cívky
					
					\item Baňka naplněná argonem s elektronovou tryskou
					
					\item Zdroj nízkého napětí pro napájení elektronové trysky
					
					\item Potenciometr pro nastavení mřížkového napětí $0-50$ V; $\Delta \text{U} = 0,1$ V %špatně, má tam bejt přesnost, co zjistim v katalogu laboratorních přístrojů
					
					\item Potenciometr pro nastavení anodového napětí $0-300$ V; $\Delta 
					\text{U} = 0,1$ V
					
					\item Výstup $6,3$ V  pro žhavení katody
					
					\item Voltmetr pro určení urychlovacího napětí
				\end{itemize}
			
			\subsection*{enumerate environment}
			
				\noindent
				Enumerate environment is used the same way itemize is, but is enumerated.
				\begin{enumerate}
					\item Před zapnutím napájecího zdroje elektronové trysky musí být potenciometry nastaveny na minimální (nulovou) hodnotu.
					
					\item Po zapnutí napájecího zdroje je třeba nechat katodu elektronové trysky cca 2 minuty žhavit, než začneme zvyšovat urychlovací napětí. Tím se šetří životnost katody elektronové trysky.
					
					\item Pro různá urychlovací napětí U (experiment dobře funguje pro napětí větší než cca 100 V) najdeme takové proudy Helmholtzovými cívkami (a tedy magnetickou indukci), kdy elektrony dopadají na luminiscenční příčky, tj., kdy lze určit cyklotronový poloměr jejich trajektorií.
					
					\item Pro jednotlivé kombinace nastavených a naměřených hodnot vypočteme měrný náboj elektronu. Z vypočtených hodnot určíme aritmetický průměr a nejistotu měření metodou redukce.
					
					\item Poté, co doměříme, nastavíme potenciometry zdroje anodového a mřížkového napětí na minimum – šetříme tím životnost katody elektronové trysky.
				\end{enumerate}
				
				\paragraph{enumitem package} allows for further customization of enumerate enviroment, such as
					\begin{enumerate}[label=(\alph*)]
						\item an apple
						\item a banana
						\item a carrot
						\item a durian
					\end{enumerate}
					
					\begin{enumerate}[label=(\Alph*)]
						\item an apple
						\item a banana
						\item a carrot
						\item a durian
					\end{enumerate}
					
					\begin{enumerate}[label=(\roman*)]
						\item an apple
						\item a banana
						\item a carrot
						\item a durian
					\end{enumerate}
			
			\subsection*{multicol}
				
				\textsf{Two column enumerate}
				\begin{multienum}
					\item item 1
					\item item 2
					\item item 3
					\item item 4
					\item item 5
					\item item 6
				\end{multienum}
				
				\textsf{Two column itemize}
				\begin{multiitem}
					\item item 1
					\item item 2
					\item item 3
					\item item 4
					\item item 5
					\item item 6
				\end{multiitem}
			
			\subsection*{tabular environment}	% needs /usepackage{array}
			
				\noindent
				Tabular environment is used when creating tables.
				\begin{center}
					\begin{tabular}{| m{0.5cm} | m{1.6cm} | m{1.6cm} | m{1.6cm} || m{1.6cm} | m{2.5cm} |}
						\hline
						\# & U [V] & 2R$_c$ [cm] & I [A] & B [mT] & e/me [C/kg] \\ [0.5ex]
						\hline\hline
						1 & 191 & 4 & 3.54 & 2.45 & $1.591 \cdot 10^{11}$ \\
						\hline
						2 & 191 & 6 & 2.31 & 1.6 & $1.658 \cdot 10^{11}$ \\
						\hline
						3 & 191 & 8 & 1.7 & 1.18 & $1.715 \cdot 10^{11}$ \\
						\hline
						4 & 191 & 10 & 1.36 & 0.94 & $1.729 \cdot 10^{11}$ \\
						\hline
						5 & 153 & 4 & 3.12 & 2.16 & $1.64 \cdot 10^{11}$ \\
						\hline
						6 & 153 & 6 & 2 & 1.38 & $1.785 \cdot 10^{11}$ \\
						\hline
						7 & 153 & 8 & 1.5 & 1.04 & $1.768 \cdot 10^{11}$ \\
						\hline
						8 & 153 & 10 & 1.18 & 0.82 & $1.82 \cdot 10^{11}$ \\
						\hline
						9 & 230 & 4 & 3.87 & 2.68 & $1.601 \cdot 10^{11}$ \\
						\hline
						10 & 230 & 6 & 2.54 & 1.76 & $1.65 \cdot 10^{11}$ \\
						\hline
						11 & 230 & 8 & 1.9 & 1.32 & $1.65 \cdot 10^{11}$ \\
						\hline
						12 & 230 & 10 & 1.5 & 1.04 & $1.701 \cdot 10^{11}$ \\
						\hline
					\end{tabular}
				\end{center}
		
			\subsection*{equation environment}
			
				\noindent
				Equation environment doesn't support laying down more equations under themselves (fucking useless, align is always better).
				\begin{equation}
					1 + 2 = 3.
				\end{equation}
				
				\begin{equation*}
					1 = 3 - 2.
				\end{equation*}
				
				\begin{equation}
					x^2 = 4.
				\end{equation}
				
				\begin{equation*}
					x = \pm\, 2 \iff |x| = 2.
				\end{equation*}
				
			\subsection*{align environment}
			
				\noindent
				Align environment supports laying down more equations and even aligning them by the highlighted symbol.
				\begin{align*}
					1 + 2 &= 3,
				\\
					1 &= 3 - 2.
				\end{align*}
				
				\begin{align*}
					x^2 &= 4,
				\\
					x = \pm\, 2 &\iff |x| = 2.
				\end{align*}
			
			\subsection*{subequations environment}
			
				\noindent
				Pretty selfexplanatory
				\begin{subequations}
					\begin{align}
						\varphi
						&= \varphi' + \frac{1}{4 \pi \epsilon_0} \frac{q}{r}
						= \frac{q}{4 \pi \epsilon_0 R} \left( \frac{1}{|\alpha \vec{r_0} - 1/\beta \vec{r_{0q}}|} - \frac{\beta}{|\alpha \vec{r_0} - \beta \vec{r_q}|} + \frac{1}{\alpha} \right) \, ,
				\\
						Z &= -\text{fujky fuj} + \textit{komiko}.
					\end{align}
				\end{subequations}
				
			\subsection*{matrix environment}
			
				\noindent
				Matrix environment itself doesn't include anything except for the formatting.
				\begin{align*}
					\begin{matrix}
						12 & -4 & 91
					\\
						6 & 999 & -6
					\\
						-20 & 0 & 10
					\end{matrix}
					\;=\;
					\begin{matrix*}[r]
						12 & -4 & 91
					\\
						6 & 999 & -6
					\\
						-20 & 0 & 10
					\end{matrix*}
				\end{align*}
			\subsection*{pmatrix environment}
			
				\noindent
				Matrix with parentheses.
				\begin{align*}
					\begin{pmatrix}
						12 & -4 & 91
					\\
						6 & 999 & -6
					\\
						-20 & 0 & 10
					\end{pmatrix}
					\;=\;
					\begin{pmatrix*}[r]
						12 & -4 & 91
					\\
						6 & 999 & -6
					\\
						-20 & 0 & 10
					\end{pmatrix*}
				\end{align*}
				
			\subsection*{bmatrix environment}
			
				\noindent
				Matrix with brackets.
				\begin{align*}
					\begin{bmatrix}
						12 & -4 & 91
					\\
						6 & 999 & -6
					\\
						-20 & 0 & 10
					\end{bmatrix}
					\;=\;
					\begin{bmatrix*}[r]
						12 & -4 & 91
					\\
						6 & 999 & -6
					\\
						-20 & 0 & 10
					\end{bmatrix*}
				\end{align*}
			
			\subsection*{Bmatrix environment}
			
				\noindent
				Matrix with braces.
				\begin{align*}
					\begin{Bmatrix}
						12 & -4 & 91
					\\
						6 & 999 & -6
					\\
						-20 & 0 & 10
					\end{Bmatrix}
					\;=\;
					\begin{Bmatrix*}[r]
						12 & -4 & 91
					\\
						6 & 999 & -6
					\\
						-20 & 0 & 10
					\end{Bmatrix*}
				\end{align*}
			
			\subsection*{vmatrix environment}
			
				\noindent
				Matrix with vertical delimeters.
				\begin{align*}
					\begin{vmatrix}
						12 & -4 & 91
					\\
						6 & 999 & -6
					\\
						-20 & 0 & 10
					\end{vmatrix}
					\;=\;
					\begin{vmatrix*}[r]
						12 & -4 & 91
					\\
						6 & 999 & -6
					\\
						-20 & 0 & 10
					\end{vmatrix*}
				\end{align*}
			
			\subsection*{Vmatrix environment}
			
				\noindent
				Matrix with double vertical delimeters.
				\begin{align*}
					\begin{Vmatrix}
						12 & -4 & 91
					\\
						6 & 999 & -6
					\\
						-20 & 0 & 10
					\end{Vmatrix}
					\;=\;
					\begin{Vmatrix*}[r]
						12 & -4 & 91
					\\
						6 & 999 & -6
					\\
						-20 & 0 & 10
					\end{Vmatrix*}
				\end{align*}
			
			\subsection*{Regular text within the math environment}
			
				\noindent
				$50 \text{ apples} \times 100 \text{ apples} = \text{loads of apples}^2$ \\
				$50 \textrm{ apples} \times 100 \textbf{ apples} = \textit{loads of apples}^2$
				
			\subsection*{Minipage}
			
				\noindent
				Minipage is useful when we need to put things next to each other
				\subsubsection*{Tabular with minipage}
				
					\noindent
					Utilizing minipage to put two tables next to each other\\
					
					\noindent
					Jaxviň nadpis
					
					\noindent
					\begin{minipage}{0.15\textwidth}
						{\Huge MC}
					\end{minipage}
					\begin{minipage}{0.85\textwidth}
						\lipsum[3]
					\end{minipage}
								
					\begin{minipage}{0.2\textwidth}
						\begin{tabular}{|c|c|c|}
							\hline
							A & B & C \\
							\hline
							1 & 2 & 3  \\
							\hline 
							4 & 5 & 6 \\
							\hline
						\end{tabular}
					\end{minipage}
					\begin{minipage}{0.2\textwidth}
						\begin{tabular}{c|c|c}
							A & B & C \\
							\hline
							1 & 2 & 3  \\
							\hline 
							4 & 5 & 6 \\
						\end{tabular}
					\end{minipage}
					\begin{minipage}{0.55\textwidth}
						Here we can write shit as wel. For example we can describe the written tables by telling that they look like shit. Or anything else.
					\end{minipage}
					\lipsum[1]
			
			
			
			\chapter{Chapter}
			
				\lipsum[1]
				\lipsum[2]
				\lipsum[3]
				\lipsum[4]
				\lipsum[5]
			
			
			
			\chapter*{Chapter*}
			\addcontentsline{toc}{chapter}{Druhý}
			
				\lipsum[1]
				\lipsum[2]
				\lipsum[3]
				\lipsum[4]
				\lipsum[5]
			
			
			
			\chapwithtoc{Chapwithtoc}
			\epigraph{
				Libovolný citát, který může být klidně dlouhý, klidně krátký.
			}{
				Autor Citátu
			}
			
			
				\lipsum[1]
				\lipsum[2]
				\lipsum[3]
				\lipsum[4]
			
	
\end{document}