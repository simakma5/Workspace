\documentclass{report}

\usepackage[a4paper, total={6in, 8in}]{geometry}
\usepackage{setspace}

\usepackage[czech]{babel}
\usepackage[utf8]{inputenc}
\usepackage[T1]{fontenc}

\usepackage{amsmath, amssymb}
\usepackage{physics}
\usepackage{array}

\title{Stanovení měrného náboje elektronu}
\author{Martin Šimák}
\date{}

\onehalfspacing

\begin{document}
	\pagenumbering{gobble}
	\maketitle
	\newpage
	\pagenumbering{arabic}
	
	\section*{Úkol}
		Cíl laboratorní úlohy je stanovit měrný náboj elektronu.
		
	\section*{Pomůcky}
		\begin{itemize}
			\item Zdroj pro napájení Helmholtzových cívek
			\item Regulátor napětí
			\item Omezovač proudu
			\item Ampérmetr pro měření proudu Helmholtzovými cívkami; $\Delta \text{I} = 0,001$ A
			\item Helmholtzovy cívky
			\item Baňka naplněná argonem s elektronovou tryskou
			\item Zdroj nízkého napětí pro napájení elektronové trysky
			\item Potenciometr pro nastavení mřížkového napětí $0-50$ V; $\Delta \text{U} = 0,1$ V %špatně, má tam bejt přesnost, co zjistim v katalogu laboratorních přístrojů
			\item Potenciometr pro nastavení anodového napětí $0-300$ V; $\Delta \text{U} = 0,1$ V
			\item Výstup $6,3$ V  pro žhavení katody
			\item Voltmetr pro určení urychlovacího napětí
		\end{itemize}
	
	\section*{Postup měření}
		\begin{enumerate}
			\item Před zapnutím napájecího zdroje elektronové trysky musí být potenciometry nastaveny na minimální (nulovou) hodnotu.
			\item Po zapnutí napájecího zdroje je třeba nechat katodu elektronové trysky cca 2 minuty žhavit, než začneme zvyšovat urychlovací napětí. Tím se šetří životnost katody elektronové trysky.
			\item Pro různá urychlovací napětí U (experiment dobře funguje pro napětí větší než cca 100 V) najdeme takové proudy Helmholtzovými cívkami (a tedy magnetickou indukci), kdy elektrony dopadají na luminiscenční příčky, tj., kdy lze určit cyklotronový poloměr jejich trajektorií.
			\item Pro jednotlivé kombinace nastavených a naměřených hodnot vypočteme měrný náboj elektronu. Z vypočtených hodnot určíme aritmetický průměr a nejistotu měření metodou redukce.
			\item Poté, co doměříme, nastavíme potenciometry zdroje anodového a mřížkového napětí na minimum – šetříme tím životnost katody elektronové trysky.
		\end{enumerate}
	
	\section*{Záznam měření a výpočet}
		\begin{center}
			\begin{tabular}{| m{0.5cm} | m{1.6cm} | m{1.6cm} | m{1.6cm} || m{1.6cm} | m{2cm} |}
				\hline
				\# & U [V] & 2R$_c$ [cm] & I [A] & B [mT] & e/me [C/kg] \\ [0.5ex]
				\hline\hline
				1 & 191 & 4 & 3.54 & 2.45 & $1.591 \cdot 10^{11}$ \\
				\hline
				2 & 191 & 6 & 2.31 & 1.6 & $1.658 \cdot 10^{11}$ \\
				\hline
				3 & 191 & 8 & 1.7 & 1.18 & $1.715 \cdot 10^{11}$ \\
				\hline
				4 & 191 & 10 & 1.36 & 0.94 & $1.729 \cdot 10^{11}$ \\
				\hline
				5 & 153 & 4 & 3.12 & 2.16 & $1.64 \cdot 10^{11}$ \\
				\hline
				6 & 153 & 6 & 2 & 1.38 & $1.785 \cdot 10^{11}$ \\
				\hline
				7 & 153 & 8 & 1.5 & 1.04 & $1.768 \cdot 10^{11}$ \\
				\hline
				8 & 153 & 10 & 1.18 & 0.82 & $1.82 \cdot 10^{11}$ \\
				\hline
				9 & 230 & 4 & 3.87 & 2.68 & $1.601 \cdot 10^{11}$ \\
				\hline
				10 & 230 & 6 & 2.54 & 1.76 & $1.65 \cdot 10^{11}$ \\
				\hline
				11 & 230 & 8 & 1.9 & 1.32 & $1.65 \cdot 10^{11}$ \\
				\hline
				12 & 230 & 10 & 1.5 & 1.04 & $1.701 \cdot 10^{11}$ \\
				\hline
			\end{tabular}
		\end{center}
		Magnetickou indukci B jsme zde zjistili na základě naměřeného proudu I procházejícího Helmholtzovými cívkami dle vztahu
		\begin{equation*}
			B \approx B_0 = \frac{8}{5\sqrt{5}} \, \frac{\mu_0 N I}{a} \, ,
		\end{equation*}
		kde $\mu_0 = 4 \pi \cdot 10^{-7} \text{ N A}^{-2}$ je permeabilita vakua, N je počet závitů každé z cívek (v tomto případě N = 154), I je velikost proudu protékaného cívkami a a je jejich poloměr (v tomto případě a = 200 mm). \\
		Hodnotu měrnéno náboje elektronu (jak jsme zjistili v teoretickém úvodu) lze vyjádřit jako
		\begin{equation*}
			\frac{e}{m_e} = \frac{2 U}{B^2 R_c^2} \, .
		\end{equation*}
		Nejpravděpodobnější hodnotu měření získáme pomocí aritmetického průměru naměřených hodnot
		\begin{equation*}
			\overline{\left( \frac{e}{m_e} \right)} = \frac{1}{n} \sum_{i=1}^{n} \left( \frac{e}{m_e} \right)_i \approx 1.6923 \cdot 10^{11} \, \frac{\text{C}}{\text{kg}} \, .
		\end{equation*}
		Dále nejistotu měření určíme metodou redukce jako
		\begin{equation*}
			u \left( \overline{\left( \frac{e}{m_e} \right)} \right)
			= \sqrt{\frac{\sum_{i=1}^{n} \left( \left( \frac{e}{m_e} \right)_i - \overline{\left( \frac{e}{m_e} \right)} \right)^2}{n(n-1)}}
			\approx 0.2108  \; \frac{\text{C}}{\text{kg}} ,
		\end{equation*}
		kde (stejně jako u aritmetického průměru hodnot) označujeme jako $(e/m_e)_i$  hodnotu měrného náboje elektronu v i-tém měření.
		
	\section*{Výsledek měření}
		Měřením jsme zjistili hodnotu měrného náboje elektronu
		\begin{equation*}
			\frac{e}{m_e} = (1.7 \pm 0.21) \cdot 10^{11} \; \frac{\text{C}}{\text{kg}} \, ,
		\end{equation*}
		kde hodnota za znakem $\pm$ udává nejistotu měření určenou metodou redukce. \\
		Tabulková hodnota měrného náboje elektronu je
		\begin{equation*}
			\frac{e}{m_e} = (1.75882012 \pm 0.00000015) \cdot 10^{11} \; \frac{\text{C}}{\text{kg}} \, ,
		\end{equation*}
		odchylka naměřených hodnot od hodnot tabulkových tak činí 3,78 \%. Nesrovnalosti s tabulkovými hodnotami byly nejspíše způsobeny nepřesnostmi námi provedeného měření.
		
		\subsection*{Seznam použité literatury}
			\begin{itemize}
				\item B. Sedlák, I. Štoll: Elektřina a magnetismus, Academia, Praha, 2002
				\item Milan Červenka: Zpracování fyzikálních měření, FEL ČVUT 2013
				\item laboratorní server Herodes: http://herodes.feld.cvut.cz
			\end{itemize}
\end{document}