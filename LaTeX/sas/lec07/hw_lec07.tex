\documentclass[11pt, a4paper]{article}

\setlength\textwidth{145mm}
\setlength\textheight{247mm}
\setlength\oddsidemargin{15mm}
\setlength\evensidemargin{15mm}
\setlength\topmargin{0mm}
\setlength\headsep{0mm}
\setlength\headheight{0mm}
\let\openright=\clearpage

\usepackage[british]{babel}
\usepackage{lmodern}
\usepackage[T1]{fontenc}
\usepackage{textcomp}

\usepackage[utf8]{inputenc}

\usepackage{stddoc}
\usepackage{mathtools}

\newtheorem{theorem}{Theorem}[section]
\newtheorem*{theorem*}{Theorem}

\begin{document}
	
	\pagenumbering{arabic}
	
	\section*{A8B37SAS: Homework from lecture 7 - 31.3.2020}
	\noindent\rule{12cm}{0.2pt}
	
	First of all, I would like to \textit{apologize} for any notational anomalies I'm using that might cause any kind of confusion. Some of the conventions I'm using in the following text are that $\theta(t)$ is the Heaviside step function or $\int_{\mathbb R} = \int_{-\infty}^{\infty}$.
	
	\section{Laplace and Z transforms of a convolution of functions}
		
		\begin{theorem}
			Let $f,g \in L_0$ (meaming that the generally complex functions are at least partially continuous and are of the exponential of lesser order of growth) have Laplace transforms $F(p),G(p)$. Then a convolution of those functions $h(t) \equiv f(t) * g(t)$ has Laplace transform
			\begin{align}
				H(p) \equiv \mathcal L[h(t)](p) = F(p) G(p).
			\end{align}
		\end{theorem}
			\begin{proof}
				Since (for Laplace transform to exist and for Fubini's theorem to work) we consider only measurable functions with value zero in the region of $\mathrm{Re}[f(t)] < 0$, we can write for the Laplace transform of the convolution of two such functions
				\begin{align*}
					H(p) &= \int_{\mathbb R} \( \int_{\mathbb R} \theta(\tau) f(\tau) \theta(t-\tau) g(t-\tau) \: \d \tau \) e^{-pt} \: \d t =
				\\
					&= \int_{\mathbb R} \( \int_{\mathbb R} \theta(\tau) f(\tau) e^{-p \tau} \theta(t-\tau) g(t-\tau) e^{-p(t-\tau)} \: \d t \) \: \d \tau.
				\end{align*}
				Furthermore, we can substitute $u \coloneqq t-\tau$, which yields
				\begin{align*}
					H(p) &= \int_{\mathbb R} \( \int_{\mathbb R} \theta(\tau) f(\tau) e^{-p \tau} \theta(u) g(u) e^{-pu} \: \d u \) \: \d \tau =
				\\
					&= \int_{\mathbb R} \theta(\tau) f(\tau) e^{-p \tau} \: \d u \int_{\mathbb R} \theta(u) g(u) e^{-pu} \: \d \tau = F(p) G(p).
				\end{align*}
			\end{proof}
		
		\begin{theorem}
			Let's assume that sequences $(a_n)_{n\in\mathbb N_0}, (b_n)_{n\in\mathbb N_0} \in Z_0$ (meaning that the generally complex sequences are of the exponential or lesser order of growth) and $(c_n)_{n\in\mathbb N_0} = (a_n)_{n\in\mathbb N_0} * (b_n)_{n\in\mathbb N_0}$. Then is true that
			\begin{align}
				\mathcal Z[c_n](z) = \mathcal Z[a_n](z) \mathcal Z[b_n](z).
			\end{align}
		\end{theorem}
			\begin{proof}
				We will prove this one directly. Starting from the right, we get
				\begin{align*}
					\mathcal Z[a_n](z) \mathcal Z[b_n](z) = \sum_{k\in\mathbb N_0} \frac{a_k}{z^k} \sum_{\l\in\mathbb N_0} \frac{b_\l}{z^\l} = \sum_{n=0}^{\infty} \( \sum_{k=0}^{n} a_k b_{n-k} \) \frac{1}{z_n} = \mathcal Z[c_n](z).
				\end{align*}
			\end{proof}
		
		
	
	
	
	
\end{document}