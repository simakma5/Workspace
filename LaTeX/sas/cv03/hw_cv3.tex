\documentclass[11pt, a4paper]{article}

\setlength\textwidth{145mm}
\setlength\textheight{247mm}
\setlength\oddsidemargin{15mm}
\setlength\evensidemargin{15mm}
\setlength\topmargin{0mm}
\setlength\headsep{0mm}
\setlength\headheight{0mm}
\let\openright=\clearpage

\usepackage[czech]{babel}
\usepackage{lmodern}
\usepackage[T1]{fontenc}
\usepackage{textcomp}

\usepackage[utf8]{inputenc}

\usepackage{stddoc}
\usepackage{mathtools}

\begin{document}
	
	\pagenumbering{arabic}
	
	\section*{Domácí úkol A8B37SAS - 5.3.2020}
	\noindent\rule{12cm}{0.2pt}

	\section{Rozklad signálu}
		
		Jsou dány signály (vektory) $\{ \zeta_i(t) \}_{i=1}^3$
		\begin{align*}
			\zeta_1(t) &= 1 - t,
		\\
			\zeta_2(t) &= - \frac 74 + \frac 34 t,
		\\
			\zeta_3(t) &= \frac{22}{3} - 8t + 2t^2
		\end{align*}
		a signál
		\begin{align*}
			s(t) = -2t^2 + 5
		\end{align*}
		na intervalu $I = \langle 1,3 \rangle$.
		
		\subsection{Ortogonalita}
			
			\paragraph{Zadání}
			
				Ukažte, že signály $\{ \zeta_i(t) \}_{i=1}^3$ jsou ortogonální na intervalu $I = \langle 1,3 \rangle$.
			
			\paragraph{Řešení}
			
				\begin{enumerate}
					\item Ortogonalita $\zeta_1(t), \zeta_2(t)$ na $I$:
						\begin{align*}
							\langle \zeta_1(t), \zeta_2(t) \rangle &= \int_1^3 \zeta_1(t) \zeta_2^*(t) \: \d t = \int_1^3 \( 1 - t \) \( - \frac 74 + \frac 34 t \) \: \d t =
						\\
							&= \int_1^3 - \frac 74 + \frac 34 t + \frac 74 t - \frac 34 t^2 \: \d t = \int_1^3 - \frac 74 + \frac {10}{4} t - \frac 34 t^2 \: \d t =
						\\
							&= \[ - \frac 74 t + \frac{10}{8} t^2 - \frac{3}{12} t^3 \]_1^3 = \( - \frac{21}{4} + \frac{90}{8} - \frac{81}{12} \) - \( - \frac 74 + \frac{10}{8} - \frac{3}{12} \) =
						\\
							&= 0.
						\end{align*}
						$\implies$ Signály $\zeta_1(t), \zeta_2(t)$ jsou ortogonální na $I$.
						
					\item Ortogonalita $\zeta_2(t), \zeta_3(t)$ na $I$:
						\begin{align*}
							\langle \zeta_2(t), \zeta_3(t) \rangle &= \int_1^3 \zeta_2(t) \zeta_3^*(t) \: \d t = \int_1^3 \( - \frac 74 + \frac 34 t \) \( \frac{22}{3} - 8t + 2t^2 \) \: \d t =
						\\
							&= \int_1^3 - \frac{77}{6} + 14 t - \frac 72 t^2 + \frac{11}{2} t - 6 t^2 + \frac 32 t^3 =
						\\
							&= \int_1^3 - \frac{77}{6} + \frac{39}{2} t - \frac{19}{2} t^2 + \frac 32 t^3 \: \d t = \[ - \frac{77}{6} t + \frac{39}{4} t^2 - \frac{19}{6} t^3 + \frac 38 t^4 \]_1^3
						\\
							&= \( - \frac{77}{2} + \frac{351}{4} - \frac{513}{6} + \frac{243}{8} \) - \( - \frac{77}{6} + \frac{39}{4} - \frac{19}{6} + \frac 38 \) = 0.
						\end{align*}
						$\implies$ Signály $\zeta_2(t), \zeta_3(t)$ jsou ortogonální na $I$.
					
					\item Ortogonalita $\zeta_1(t), \zeta_3(t)$ na $I$:
						\begin{align*}
							\langle \zeta_1(t), \zeta_3(t) \rangle &= \int_1^3 \zeta_1(t) \zeta_3^*(t) \: \d t = \int_1^3 \( 1 - t \) \( \frac{22}{3} - 8t + 2t^2 \) \: \d t =
						\\
							&= \int_1^3 \frac{22}{3} - 8t + 2t^2 - \frac{22}{3} t + 8 t^2 - 2 t^3 \: \d t =
						\\
							&= \int_1^3 \frac{22}{3} - \frac{46}{3} t + 10 t^2 - 2 t^3 \: \d t = \[ \frac{22}{3} t - \frac{23}{3} t^2 + \frac{10}{3} t^3 - \frac 12 t^3 \]_1^3 =
						\\
							&= \( 22 - 69 + 90 - \frac{81}{2} \) - \( \frac{22}{3} - \frac{23}{3} + \frac{10}{3} - \frac 12 \) = 0.
						\end{align*}
						$\implies$ Signály $\zeta_1(t), \zeta_3(t)$ jsou ortogonální na $I$.
						
					\item Ortogonalita je relace symetrická (stejně jako skalární součin je symetrická operace), tudíž nemusíme testovat ortogonalitu signálů v permutovaném pořadí.
				\end{enumerate}
			
		\subsection{Rozklad signálu}
			
			\paragraph{Zadání}
			
				Rozložte signál $s(t)$ pomocí signálů $\{ \zeta_i(t) \}_{i=1}^3$ na intervalu $I = \langle 1,3 \rangle$, tj. najděte koeficienty rozkladu.
				
			\paragraph{Řešení}
				
				Pokud hledáme koeficienty (v tomto případě ortogonálního) rozkladu, hledáme tak, obecně komplexní, čísla $\alpha_1, \alpha_2, \alpha_3$ tak, aby splňovaly rovnost
				\begin{align*}
					s(t) = -2t^2 + 5 &= \sum_{i=1}^{3} \alpha_i \zeta_i(t) = \alpha_1 \zeta_1(t) + \alpha_2 \zeta_2(t) + \alpha_3 \zeta_3(t) =
				\\
					&= \alpha_1 \( 1 - t \) + \alpha_2 \( - \frac 74 + \frac 34 t \) + \alpha_3 \( \frac{22}{3} - 8t + 2t^2 \)
				\\
					&= \( \alpha_1 - \frac 74 \alpha_2 + \frac{22}{3} \alpha_3 \) + \( - \alpha_1 + \frac 34 \alpha_2 -8 \alpha_3 \) t + \( 2 \alpha_3 \) t^2.
				\end{align*}
				Jelikož polynomy $t, t^2, t^3$ jsou nad prostorem funkcí lineárně nezávislé, dostáváme následující 3 nezávislé rovnice:
				\begin{align}
					\alpha_1 - \frac 74 \alpha_2 + \frac{22}{3} \alpha_3 &= 5,
				\\
					-\alpha_1 + \frac 34 \alpha_2 -8 \alpha_3 &= 0,
				\\
					\label{eq:3}
					2 \alpha_3 &= -2.
				\end{align}
				Z rovnice \eqref{eq:3} vidíme rovnou hodnotu třetího koeficientu $\alpha_3 = -1$. Zbývají nám tedy již pouze 2 rovnice, které můžeme dosazením hodnoty $\alpha_3$ a přenásobením dvanácti přepsat do tvaru
				\begin{align}
					\tag{1a} \label{eq:1a}
					12 \alpha_1 - 21 \alpha_2 &= 148,
				\\
					\tag{2a} \label{eq:2a}
					-4 \alpha_1 + 3 \alpha_2 &= -32.
				\end{align}
				Dále provedením kroku $\eqref{eq:1a} + 3 \times \eqref{eq:2a}$ rovnice sloučíme do jedné
				\begin{align*}
					-12 \alpha_2 = 52.
				\end{align*}
				Z této poslední rovnosti vyplývá fakt, že $\alpha_2 = -13/3$. Konečně koeficient $\alpha_1$ získáme zpětným dosazením např. do rovnice \eqref{eq:2a}
				\begin{align*}
					-4 \alpha_1 = -19,
				\end{align*}
				odkud vyplývá, že $\alpha_1 = 19/4$. Zjistili jsme tedy takto koeficienty rozkladu\footnote{Značení ve finálním výsledku vyjadřuje souřadnice vektoru vůči uspořádané bázi, např. $\mathrm{coord}_{(\vec b_1, \dots, \vec b_n)}(\vec v)$ značí souřadnice vektoru $\vec v$ vůči uspořádané bázi $(\vec b_1, \dots, \vec b_n)$.}:
				\begin{align*}
					\Aboxed{\mathrm{coord}_{(\zeta_1,\zeta_2,\zeta_3)}(s(t)) = \( \frac{19}{4}, - \frac{13}{3}, -1 \).}
				\end{align*}
	
	
\end{document}