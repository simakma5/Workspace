\documentclass{article}

\usepackage{amsmath, amssymb, esint}
\usepackage{setspace}
\usepackage{physics}
\usepackage{bm}
\usepackage[czech]{babel}
\usepackage[utf8]{inputenc}
\usepackage[T1]{fontenc}

\newcommand{\vecvar}[1]{\bm{#1}}
\newcommand{\vecconst}[1]{\mathbf{#1}}

\title{Příklady pro týden 1}
\author{Martin Šimák}
\date{}

\begin{document}
	\pagenumbering{gobble}
	\maketitle{}
	
	\paragraph{Příklad 1;} $\vecvar{r} \neq \vecvar{r'}$\\
	
	\subparagraph{(1)} \[
	\div\left(\frac{\vecvar{R_0}}{R^n}\right)
	= \frac{R^n (\div\vecvar{R_0}) - \vecvar{R_0} \cdot (\grad R^n)}{R^{2n}}
	= \frac{R^n (\frac{2}{R}) - \vecvar{R_0} \cdot (nR^{n-1} \vecvar{R_0})}{R^{2n}} = \frac{2R^{n-1} - nR^{n-1}}{R^{2n}}
	= \frac{2 - n}{R^{n + 1}}
	\]

	\subparagraph{(2)} \[
	\div{\left( \grad{\frac{1}{R}} \right)}
	= \nabla^2(\frac{1}{R})
	= -4 \pi \delta^3 (\vecvar{R})
	= 0, \\
	\]
	kde poslední rovnost platí díky faktu, že $\vecvar{r} \neq \vecvar{r'}$.
	\newpage
	
	\paragraph{Příklad 2} Geometricky (1) představuje přirozený dekrement působení silového radiálního pole generovaného bodovým zdrojem. Divergence je v každém bodě nulová, kdybychom však spočítali průtok pole sférou o poloměru $R$, dostaneme:\\ \[
	\oiint \limits_{S} \vecvar{F} \cdot \mathrm{d}\vecvar{A}
	= \int \limits_{S} \left( \frac{1}{R} \, \vecvar{r} \right) \cdot \left( R^2 \sin(\theta) \, \mathrm{d}\theta \mathrm{d}\phi \, \vecvar{r} \right)
	= \left( \int_{0}^{\pi} \sin(\theta) \, \mathrm{d}\theta \right) \left( \int_{0}^{2\pi} \mathrm{d}\phi \right)
	= 4\pi, \\
	\]
	což neodpovídá našemu očekávání (alternativní řešení pomocí Gaussovy integrální věty vede na nulové řešení). Paradox je tvořen tím, že při výpočtu divergence nepřipouštíme $\vecvar{R} = 0$, což by způsobilo dělení nulou (pole má v bodě $\vecvar{R} = 0$ singularitu). Divergence v tomto místě je tedy odpovědna za průtok pole v celém prostoru.
	Tento problém lze elegantně vyřešit zapomoci Diracovy funkce delta, která by nám v tomto případě udávala nulovou divergenci všude, krom počátku.
	
	
\end{document}
