\documentclass[11pt,a4paper]{report}
\setlength\textwidth{145mm}
\setlength\textheight{247mm}
\setlength\oddsidemargin{15mm}
\setlength\evensidemargin{15mm}
\setlength\topmargin{0mm}
\setlength\headsep{0mm}
\setlength\headheight{0mm}
\let\openright=\clearpage

\usepackage[czech]{babel}
\usepackage{lmodern}
\usepackage[T1]{fontenc}
\usepackage{textcomp}

\usepackage[utf8]{inputenc}

\usepackage{stddoc}
\usepackage{mathtools}
\usepackage{circuitikz}
\tikzstyle{densely dashed}=[dash pattern=on 4pt off 3pt]

\begin{document}
	
	\pagenumbering{arabic}
	
	\section*{Příklady pro týden 7}
	\noindent\rule{12cm}{0.2pt}
	
	\paragraph{Zadání}
		Určete elektrický odpor na vstupních svorkách nekonečné žebříčkové struktury dle Obr. 1.
		
		\begin{figure}[h!]
			\begin{center}
				\begin{circuitikz}[american voltages]
					\draw (0,2)
					to[short, *-] (0,2)
					to[R=$R_1$] (2,2)
					to[R=$R_2$] (2,0)
					to[short, -*] (0,0);
					\draw (2,2)
					to[short, *-] (2,2)
					to[R=$R_1$] (4,2)
					to[R=$R_2$] (4,0)
					to[short, -*] (2,0);
					\draw (4,2)
					to[short, *-] (4,2)
					to[R=$R_1$] (6,2)
					to[short, -*] (6,2)
					to[R=$R_2$] (6,0)
					to[short, *-*] (4,0);
					\draw[densely dashed] (6,2)--(8,2);
					\draw[densely dashed] (6,0)--(8,0);
				\end{circuitikz}
				\\[3mm] Obr. 1
			\end{center}
		\end{figure}
	
	\noindent\rule{8cm}{0.4pt}
	\paragraph{Řešení}
		Daný problém můžeme pojmout jako nekonečnou posloupnost, jejíž členy představují opakující se části zadaného obvodu (jeden člen je znázorněn na Obr. 2). Obr. 2 zároveň znázorňuje první člen takto definované posloupnosti, u kterého můžeme jednoduše určit odpor jako
		\begin{align*}
			x_1 = R_1 + R_2 .
		\end{align*}
		
		\begin{figure}[h!]
			\begin{center}
				\begin{circuitikz}
					\draw[densely dashed] (0,2)--(2,2);
					\draw[densely dashed] (0,0)--(2,0);
					\draw (2,2)
					to[short, *-] (2,2)
					to[R=$R_1$] (4,2)
					to[short, -*] (4,2)
					to[R=$R_2$] (4,0)
					to[short, *-*] (2,0);
					\draw[densely dashed] (4,2)--(6,2);
					\draw[densely dashed] (4,0)--(6,0);
				\end{circuitikz}
				\\[3mm] Obr. 2
			\end{center}
		\end{figure}
		\newpage
		
		Následné dva členy posloupnosti lze jednoduchou obvodovou analýzou určit jako (obvodové diagramy Obr. 3a, Obr. 3b)
		\begin{align*}
			x_2 &= \frac{(R_1 + R_2) R_2}{R_1 + 2 R_2} + R_1 , \\
			x_3 &= \frac{\( R_1 + \frac{R_1 R_2 + R_2^2}{R_1 + 2R_2} \) R_2}{\( R_1 + \frac{R_1 R_2 + R_2^2}{R_1 + 2R_2} \) + R_2} .
		\end{align*}
		
		\begin{figure}[h!]
			\begin{center}
				\begin{circuitikz}
					\draw (0,2)
					to[short, *-] (0,2)
					to[R=$R_1$] (2,2)
					to[short, -*] (2,2)
					to[R=$R_2$] (2,0)
					to[short, -*] (0,0);
					\draw (2,2)
					to[short, *-] (2,2)
					to[R=$R_1$] (4,2)
					to[short, -*] (4,2)
					to[R=$R_2$] (4,0)
					to[short, *-*] (2,0);
				\end{circuitikz}
				\\[3mm] Obr. 3a
			\end{center}
		\end{figure}
	
		\begin{figure}[h!]
			\begin{center}
				\begin{circuitikz}
					\draw (0,2)
					to[short, *-] (0,2)
					to[R=$R_1$] (2,2)
					to[R=$R_2$] (2,0)
					to[short, -*] (0,0);
					\draw (2,2)
					to[short, *-] (2,2)
					to[R=$R_1$] (4,2)
					to[R=$R_2$] (4,0)
					to[short, -*] (2,0);
					\draw (4,2)
					to[short, *-] (4,2)
					to[R=$R_1$] (6,2)
					to[short, -*] (6,2)
					to[R=$R_2$] (6,0)
					to[short, *-*] (4,0);
				\end{circuitikz}
				\\[3mm] Obr. 3b
			\end{center}
		\end{figure}
		
		V členech posloupnosti lze tedy lehce poznat rekurentní vzorec
		\begin{align*}
			x_{n+1} = \frac{x_n R_2}{x_n + R_2} + R_1 ,
		\end{align*}
		přičemž naše řešení je ntý člen posloupnosti, když n se bude blížít nekonečnu. Pro tento nekonečný člen můžeme uvést rovnost
		\begin{align*}
			\lim_{n \to \infty} a_{n+1} = \lim_{n \to \infty} a_n .
		\end{align*}
		Označíme-li tedy $L \equiv \lim_{n \to \infty} a_n$, můžeme psát rekurentní vzorec jako
		\begin{align*}
			L = \frac{L R_2}{L + R_2} + R_1 ,
		\end{align*}
		což nám umožňuje řešit rovnici pro L, kdy jediné kladné řešení (záporné samozřejmě z fyzikálních důvodů nelze uvažovat) této rovnice je
		\begin{align*}
			\Aboxed{L \equiv R = \frac{R_1}{2} \sqrt{1 + 4 \frac{R_2}{R_1}}} \, .
		\end{align*}
		
		
		
		
\end{document}