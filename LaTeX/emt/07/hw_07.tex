\documentclass[12pt,a4paper]{report}
\setlength\textwidth{145mm}
\setlength\textheight{247mm}
\setlength\oddsidemargin{15mm}
\setlength\evensidemargin{15mm}
\setlength\topmargin{0mm}
\setlength\headsep{0mm}
\setlength\headheight{0mm}
\let\openright=\clearpage

\usepackage[czech]{babel}
\usepackage{lmodern}
\usepackage[T1]{fontenc}
\usepackage{textcomp}

\usepackage[utf8]{inputenc}

\usepackage{stddoc}
\usepackage{mathtools}


\renewcommand{\vec}{\boldsymbol}
\def\endl{\\[3mm]}
\newcommand*\colvec[3][]{
	\begin{pmatrix}
		\ifx \relax#1 \relax
		\else #1\\
		\fi
		#2 \\ #3
	\end{pmatrix}
}

\begin{document}
	\pagenumbering{gobble}
	
	\begin{center}
		\section*{Příklady pro týden 7 - Martin Šimák}
		\noindent\rule{13cm}{1.6pt} \\[5mm]
	\end{center}
	
	\subsection*{Zadání}
		Dvě kruhové nekonečně tenké smyčky o poloměru a jsou protékány stacionárním proudem I (proud teče v obou ve stejném směru). Smyčky jsou souosé a vzdálené d od sebe. Určete magnetické pole na ose systému. Určete Taylorův rozvoj pole v okolí geometrického středu systému. Nastavte poměr poloměru a vzdálenosti smyček tak, aby průběh magnetického pole v okolí středu systému byl co neplošší (snažte se   popořadě   vynulovat   co   největší   počet   vyšších   členů   Taylorova   rozvoje).   Určete   velikost magnetického pole ve středu systému pro tento ideální případ. \\
	\noindent\rule{8cm}{0.4pt}
	
	\subsection*{Řešení}
		Nejprve začneme s výpočtem jednodušší (kanonické)  úlohy, která spočívá ve výpočtu intenzity magnetického pole $B$ na ose jedné smyčky položené v ose $x,y$.
		
		\subsubsection*{Část první - kanonická paralela}
			Vektor $B$ určitě nebude záviset na $\varphi$, jelikož je v této souřadnici invariantní, bude tedy velice výhodné přejít při výpočtu do cylindrické souřadnicové soustavy. Měříme $B$ na ose smyčky (cylindrická vzdálenost od osy $\rho$ je nulová), můžeme tedy položit
			\begin{align*}
				\vec r = (0,0,z), \, \vec r' = (a \cos(\varphi), a \sin(\varphi), 0) \implies (\vec r - \vec r') = (-a \cos(\varphi), -a \sin(\varphi), z).
			\end{align*}
			Při výpočtu vyjdeme ze základního vztahu magnetostatiky, tedy z Biot-Savartova zákona
			\begin{align*}
				\vec B(\rho = 0, z) = \vec B_0(z) = \frac{\mu_0}{4 \pi} \int\limits_{V} \frac{\vec J(\vec r') \times (\vec r - \vec r') }{|| \vec r - \vec r' ||^3} \d V',
			\end{align*}
			který v našem případě můžeme upravit na
			\begin{align*}
				\vec B_0(z) &= \frac{\mu_0}{4 \pi} \int\limits_0^{2 \pi} \frac{I}{(a^2 + z^2)^{3/2}} \, \vec e_\varphi \times \colvec[-a \cos(\varphi)]{-a \sin(\varphi)}{z} a \, \d \varphi ,
				&	
				\vec e_\varphi &= \colvec[-\sin(\varphi)]{\cos(\varphi)}{0}.
			\end{align*}
			Známe-li tedy složky vektorů $(\vec r - \vec r'), \vec e_\varphi$, můžeme vypočítat jejich vektorový součin jako
			\begin{align*}
				\vec e_\varphi \times (\vec r - \vec r')
				&= \colvec[- \sin(\varphi)]{\cos(\varphi)}{0} \times \colvec[-a \cos(\varphi)]{-a \sin(\varphi)}{z} = \colvec[z \cos(\varphi)]{z \sin(\varphi)}{a}.
			\end{align*}
			Dosadíme-li do vztahu, dostaneme dále již lehce řešitelný výraz
			\begin{align*}
				\vec B_0(z) = \frac{\mu_0 I}{4 \pi} \int\limits_0^{2 \pi} \frac{a}{\( a^2 + z^2 \)^{3/2}} \colvec[z \cos(\varphi)]{z \sin(\varphi)}{a} \d \varphi,
			\end{align*}
			u kterého, využijeme-li vlastností trigonometrických funckí, které (stejně jako všechny periodické funkce) mají nulový integrál přes celou periodu, tak se naše řešení zredukuje jen na složku ve směru osy $z$. Výsledné řešení kanonického případu je tedy
			\begin{align*}
				\Aboxed{\vec B_0(z) = \frac{\mu_0 I}{2} \frac{a^2}{\( a^2 + z^2 \)^{3/2}} \, \vec e_z.}
			\end{align*}
			
		\subsubsection*{Část druhá - přechod k řešení zadaného příkladu}
			Při řešení celého příkladu můžeme vyjít z vyřešené úlohy kanonické, kdy výsledný vztah z předchozí části se tentokrát ve vztahu pro magnetickou intenzitu objeví dvakrát (pro každou ze smyček jednou), přičemž jsou vzájemně posunuté na ose $z$ o vzdálenost $d$, můžeme tedy pro vektor magnetické intenzity na ose smyček psát
			\footnote{Budeme dále počítat pouze s velikostí $B$, směr se totiž derivacemi nijak nezmení}
			\begin{align*}
				B(\rho = 0, z) = B_0(z) = \frac{\mu_0 I a^2}{2} \( \frac{1}{(a^2 + (z + d/2)^2)^{3/2}} + \frac{1}{(a^2 + (z - d/2)^2)^{3/2}} \) \vec e_z .
			\end{align*}
			Jelikož jsme si takto zvolili souřadnou soustavu, je geometrický střed naší dvojice smyček přesně v počátku $\vec o$. Pro Taylorův rozvoj v tomto geometrickém středu tedy můžeme psát
			\begin{align*}
				B_0(z) = B_0(0) + B'_0(0) z + \frac{1}{2} B''_0(0) z^2 + \cdots ,
			\end{align*}
			kde $B'_0(0), B''_0(0)$ značí první a druhou derivaci (další členy Taylorovy řady v zájmu přehlednosti dokumentu neuvádíme) podle $z$, které jsou určeny jako
			\begin{align*}
				B'_0(0) &= \left. \frac{\mu_0 I a^2}{2} \( \frac{-3 (z + d/2)}{(a^2 + (z + d/2)^2)^{5/2}} + \frac{-3 (z - d/2)}{(a^2 + (z - d/2)^2)^{5/2}} \) \right|_{z=0} = 0, \endl
				B''_0(0) &= \frac{\mu_0 I a^2}{2} \( \frac{-3}{(a^2 + (z + d/2)^2)^{5/2}} + \frac{15(z + d/2)^2}{(a^2 + (z + d/2)^2)^{7/2}} \right. + \\
				&\left. \left. \quad + \frac{-3}{(a^2 + (z - d/2)^2)^{5/2}} + \frac{15(z - d/2)^2}{(a^2 + (z - d/2)^2)^{7/2}} \) \right|_{z=0} = \endl
				&= \frac{\mu_0 I a^2}{2} \( \frac{-6 (a^2 + (d/2)^2) + 30(d/2)^2)}{(a^2 + (d/2)^2)^{7/2}} \) = \endl
				&= 3 \mu_0 I a^2 \( \frac{4(d/2)^2 - a^2}{(a^2 + (d/2)^2)^{7/2}} \) .
			\end{align*}
			Jelikož se snažíme o co nejplošší průběh intenzity magnetického pole uvnitř soustavy, snažíme se tak vlastně o eliminaci (vynulování) co největšího počtu členů Taylorova rozvoje v okolí geometrického středu dvojice smyček (pouze tak dochází totiž k nejmenším (ideálně nulovým) fluktuacím zmíněného pole). Z této úvahy tak plyne další postup, který nám umožní určit vzdálenost smyček tak, aby vyšla i druhá derivace nulová. Tento požadavek lze uspokojit pouze pokud bude nulový čitatel zlomku, tedy
			\begin{align*}
				B''_0(0) = 3 \mu_0 I a^2 \( \frac{4(d/2)^2 - a^2}{(a^2 + (d/2)^2)^{7/2}} \) = 0 \iff 4 (d/2)^2 - a^2 = 0 .
			\end{align*}
			Jelikož obě veličiny $a$ i $d$ jsou z fyzikálního hlediska vzdálenosti, víme tak, že $a, d \geq 0$, proto je poslední podmínka splněna právě tehdy, když
			\begin{align}
				\tag{H}
				\label{eq:helmholtz}
				\Aboxed{a = d.}
			\end{align}
			
	\subsection*{Závěr}
			Pokud tedy dále budeme předpokládat rovnost \ref{eq:helmholtz}, intenzita magnetického pole v geometrickém středu dvojice smyček\footnote{V tomto případě můžeme hovořit o Helmholtzových cívkách (s jedním závitem), proto výslednou intenzitu magnetického pole značíme $\vec B_H$} bude
			\begin{align*}
				\vec B_H(0) = \frac{\mu_0 I a^2}{2} \( \frac{2}{(a^2 + (a/2)^2)^{3/2}} \) \vec e_z = \frac{\mu_0 I}{a} \( \frac{1}{(1 + 1/4)^{3/2}} \) \vec e_z ,
			\end{align*}
			\begin{align*}
				\Aboxed{\vec B_H(0) = \frac{8 \mu_0 I}{5 \sqrt{5} a} \vec e_z . }
			\end{align*}
\end{document}