\documentclass[11pt,a4paper]{article}
\usepackage[a4paper,hmargin=1in,vmargin=1in]{geometry}

\usepackage[czech]{babel}
\usepackage[utf8]{inputenc}
\usepackage[T1]{fontenc}

\usepackage{stddoc}
\usepackage{multicol}




\begin{document}
	
	\pagenumbering{arabic}
	
	%% Hlavička
	\begin{center}
		\section*{Domácí úkol pro týden 2}\vspace*{-5mm}
		\begin{minipage}{0.4\textwidth}
			\begin{flushleft}
				\textsc{3. října 2020}
			\end{flushleft}
		\end{minipage}
		~
		\begin{minipage}{0.4\textwidth}
			\begin{flushright}
				\textsc{Martin Šimák}
			\end{flushright}
		\end{minipage}
		\noindent\rule{14.5cm}{0.6pt}
	\end{center}
	
	%% Povinná část
	\paragraph*{Zadání.} Určete střed a poloměr konvergence každé z následujících mocninných řad.
		\begin{multicols}{2}
			\begin{enumerate}[label=(\alph*)]
				
				\item
				\begin{align*}
					\sum_{n=0}^\infty (2+3i)^n(z-2+i)^{3n+1},
				\end{align*}
				
				\item
				\begin{align*}
					\sum_{n=0}^\infty \frac{2^n+n^5}{5^n+n^2} (z-1)^n,
				\end{align*}
				
				\item
				\begin{align*}
					\sum_{n=0}^\infty n(2z+i)^n,
				\end{align*}
				
				\item
				\begin{align*}
					\sum_{n=0}^\infty \frac{(-1)^n}{(2n+1)!} z^{2n+1}.
				\end{align*}
				
			\end{enumerate}
		\end{multicols}
	
	\paragraph*{Řešení.} Pro řešení daných příkladů budeme vycházet ze znalosti, že každou mocninnou řadu lze převést do standardního tvaru
		\begin{align}
				\sum_{n=0}^{\infty} a_n (z-z_0)^n,
		\end{align}
		kde $z_0$ je právě hledaný střed mocninné řady, a znalosti dvou následujících vzorců jasně determinujících poloměr konvergence mocninné řady:
		\begin{align}
				\frac 1R &= \limsup_{n\to\infty} \sqrt[n]{|a_n|},
			&
				\frac 1R &= \lim_{n\to\infty} \left| \frac{a_{n+1}}{a_n} \right|.
		\end{align}
		
		\begin{enumerate}[label=(\alph*)]
			
			\item $z_0 = 2-i$.
				\begin{align*}
						\frac 1R &= \limsup_{k\to\infty} \sqrt[k]{|a_k|} = \Big| k \equiv 3n+1 \Big| = \lim_{n\to\infty} \sqrt[3n+1]{|2+3i|^n} = 	\lim_{n\to\infty} 13^{\frac{n}{6n+2}} = \sqrt[6]{13},
					\\
						\Aboxed{R &= 13^{-1/6}.}
				\end{align*}
			
			\item $z_0 = 1$.
				\begin{align*}
						\frac 1R &= \lim_{n\to\infty} \left| \frac{a_{n+1}}{a_n} \right| = \lim_{n\to\infty} \left|\frac{2^{n+1}+(n+1)^5}{5^{n+1}+(n+1)^2} \cdot \frac{5^n+n^2}{2^n+n^5}  \right| =
					\\
						&= \lim_{n\to\infty} \left| \frac{2 \cdot 10^n+2^{n+1}n^2+5^n (n+1)^5 + n^2 (n+1)^5}{5 \cdot 10^n + 5^{n+1} n^5 + 2^n (n+1)^2 + n^5 (n+1)^2} \right| = \frac 25,
					\\
						\Aboxed{R &= \frac 52.}
				\end{align*}
			
			\item $z_0 = -i/2 \impliedby \sum_{n=0}^\infty n(2z+i)^n = \sum_{n=0}^\infty n2^n(z+i/2)^n$.
				\begin{align*}
						\frac 1R &= \limsup_{n\to\infty} \sqrt[n]{|a_n|} = \lim_{n\to\infty} \sqrt[n]{n 2^n} = 2 \lim_{n\to\infty} \sqrt[n]{n} = 2,
					\\
						\Aboxed{R &= \frac 12.}
				\end{align*}
			
			\item $z_0 = 0$.
				\begin{align*}
						\frac 1R &= \limsup_{n\to\infty} \sqrt[n]{|a_n|} = \lim_{n\to\infty} \frac{1}{\sqrt[n]{n!}} = 0,
					\\
						\Aboxed{R &= \infty.}
				\end{align*}
			
		\end{enumerate}
	
	%% Bonus
	\paragraph*{Bonus.} Určete poloměr konvergence následujících mocninných řad.
		\begin{multicols}{2}
			\begin{enumerate}[label=(\alph*)]
				
				\item
					\begin{align*}
							\sum_{n=0}^\infty (4^n + (-2)^n) z^n,
					\end{align*}
				
				\item
					\begin{align*}
							\sum_{n=0}^\infty \sin\( e^{2^{-n}} - 1 \) z^n.
					\end{align*}
				
			\end{enumerate}
		\end{multicols}
		
	\paragraph*{Řešení.} I v tomto případě budeme postupovat stejným způsobem.
		
		\begin{enumerate}[label=(\alph*)]
			
			\item
				\begin{align*}
						\frac 1R &= \lim_{n\to\infty} \left| \frac{a_{n+1}}{a_n} \right| = \lim_{n\to\infty} \left| \frac{4^{n+1} + (-2)^{n+1}}{4^n + (-2)^n} \right| = \lim_{n\to\infty} \left| \frac{4 - 2 \cdot (-1/2)^n)}{1 + (-1/2)^n} \right| = 4,
					\\
						\Aboxed{R &= \frac 14.}
				\end{align*}
			
			\item
				\begin{align*}
					\frac 1R &= \lim_{n\to\infty} \left| \frac{a_{n+1}}{a_n} \right| = \left|\lim_{n\to\infty} \frac{\sin\(e^{2^{-n-1}} - 1\)}{\sin\( e^{2^{-n}} - 1 \)} \right| = \left\langle \frac 00 \right\rangle \overset{\mathcal{L'H}}{=\joinrel=}
				\\
					&\overset{\mathcal{L'H}}{=\joinrel=} \left| \lim_{n\to\infty} \frac{\cos\(e^{2^{-n-1}}-1\)e^{2^{-n-1}} 2^{-n-1} \ln(2) (-1)}{\cos\(e^{2^{-n}} - 1 \) e^{2^{-n}} 2^{-n} \ln(2) (-1)} \right| =
				\\
					&= \left| \lim_{n\to\infty} \frac{\cos\(e^{2^{-n-1}}-1\)e^{2^{-n-1}} }{2 \cos\(e^{2^{-n}} - 1 \) e^{2^{-n}}} \right| =
				\\
					&\quad \left\langle \frac{e^{2^{-n-1}}}{e^{2^{-n}}} = e^{2^{-n-1}} \cdot e^{-2^{-n}} = e^{2^{-n-1} - 2^{-n}} = e^{2^{-n}\( 2^{-1}-1\)} = e^{-2^{-n-1}} \right\rangle
				\\
					&= \left| \lim_{n\to\infty} \frac{\cos\(e^{2^{-n-1}}-1\) }{2 e^{2^{-n-1}} \cos\(e^{2^{-n}} - 1 \)} \right| = \left| \frac{\lim_{n\to\infty} \cos\(e^{2^{-n-1}}-1\)}{2 \lim_{n\to\infty} e^{2^{-n-1}} \cos\(e^{2^{-n}} - 1\) } \right| = \frac 12,
				\\
					\Aboxed{R &= 2.}
				\end{align*}
			
		\end{enumerate}
	
	
\end{document}