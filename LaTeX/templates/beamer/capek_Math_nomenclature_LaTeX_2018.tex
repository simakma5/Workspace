\documentclass[xcolor=dvipsnames, aspectratio=169, handout, intlimits]{beamer} 
% 1610, 169, 149, 54, 43, 32 ratios are available
% [eulervm] / [eulervm] / [eulervm] / [eulervm]
% [draft]              - no figures, no footlines, no media...
% [handout]            - no overlays
% [10pt,11pt,12pt,...] - font size

% optimal video resolution [mm]: (1/4)*[CamtasiaX(pt) CamtasiaY(pt)]*0.3528

%================================================================================
% TITLE PAGE:
%================================================================================
\newcommand{\myEmail}{\texttt{\href{mailto:miloslav.capek@fel.cvut.cz}{miloslav.capek@fel.cvut.cz}}}
\newcommand{\myTitleMain}{ % SHORT TITLE FOR BOTTOM LINE
Mathematical Nomenclature and \LaTeX
}
\newcommand{\myTitleBottom}{ % MAIN TITLE OF THE PRESENTATION
Mathematical Nomenclature and \LaTeX
}
\newcommand{\mySubtitle}{ % SUBTITLE
}
\newcommand{\myAuthorsShort}{ % SHORT AUTHORS
\v{C}apek, M.
}
\newcommand{\myAuthors}{ % AUTHORS
{
  \textcolor{elmagLight}{Miloslav~\v{C}apek}
}}
\newcommand{\myInstituteShort}{ % SHORT AFFILIATION
CTU in Prague
}
\newcommand{\myInstitute}{ % FULL AFFILIATION
{
Department of Electromagnetic Field \\
Czech Technical University in Prague, Czech Republic \\
\fontsize{8}{5}{\myEmail} 
}
}
\newcommand{\myPlace}{ % PLACE OF EVENT
Seminar \\
Prague, Czech Republic \\
November 6, 2018}
\newcommand{\myDate}{ % THE DATA
\today}
\newcommand{\myVersion}{ % VERSION OF THE PRESENTATION
v1.1
}

\usepackage{ragged2e}
\usepackage{tikz}
\usepackage{tikz-3dplot}
\usepackage{calc}
\usepackage{booktabs}
\usepackage{array}
\usepackage{gensymb}
\usepackage{verbatim}
\usepackage{eurosym}

\newcommand{\myData}{data/}

\usepackage{capekBeamerClass} % \usepackage{\myLaTeX capekBeamerClass}
\usepackage{capekCommands} % \usepackage{\myLaTeX capekCommands}
\usepackage{plotGraphics}

\providecommand{\Cmat}{\ensuremath{\M{C}}}
\providecommand{\setB}{\ensuremath{\OP{B}}}

\graphicspath{{\myData}} % WHERE ARE THE FIGURES...
%\addbibresource{references/references_LIST_UpToDate} % WHERE ARE THE REFERENCES... \myLaTeX
\addbibresource{d:/Work/Projekty/GIT/capek-CMtext/references}

\newcommand{\finalVersion}{0}

\begin{document}
\begin{frame}
  \titlepage
\end{frame}

%%%%%%%%%%%%%%%%%%%%%%%%%%%%%%%%%%%%%%%%%%%%%%%%%%%%%%%%%%%%%%%%%%%%%%%%%%%%%%%%%
% OUTLINE:
%%%%%%%%%%%%%%%%%%%%%%%%%%%%%%%%%%%%%%%%%%%%%%%%%%%%%%%%%%%%%%%%%%%%%%%%%%%%%%%%%
\begin{frame}{Outline}

\begin{columns}
	\begin{column}{0.6\textwidth}
\hfill
\vspace{1cm}
\hspace{2cm}		

\parbox[c]{.88\textwidth}{
	\small\tableofcontents}

\vspace{2cm}  
\noindent\makebox[\linewidth]{\rule{0.9\textwidth}{0.2pt}} \\
\scriptsize
Disclaimer:
\begin{itemize}
	\item I am not an expert in the topic, just a fan.
	\item Often just a best practice or personal experience is presented.
\end{itemize}

	\end{column}
	\begin{column}{0.35\textwidth}
		
	\pgfmathsetmacro{\svec}{0.6}	
	\pgfmathsetmacro{\rvec}{.8}
	\pgfmathsetmacro{\thetavec}{35}
	\pgfmathsetmacro{\phivec}{70}
	\def\mFrameRate{10}
	\begin{animateinline}[autoplay, loop]{\mFrameRate}
		\multiframe{72}{rt=1+5}{%
			\tdplotsetmaincoords{50}{120+\rt}
			\begin{tikzpicture}[scale=3,tdplot_main_coords]
			% plot spheres
			\shade[tdplot_screen_coords,ball color = white] (0,0) circle (\svec); 
			\shade[tdplot_screen_coords,ball color=blue,opacity=0] (0,0) circle (1.2*\rvec); % bounding box
			% center of the coordinate system and arrows
			\coordinate (O) at (0,0,0) node[anchor=east, yshift=5pt]{$\mathbf{0}$};
			\draw[thick,->] (0,0,0) -- (3/4,0,0) node[anchor=north east]{$x$};
			\draw[thick,->] (0,0,0) -- (0,3/4,0) node[anchor=north west]{$y$};
			\draw[thick,->] (0,0,0) -- (0,0,7/8) node[anchor=south]{$z$};
			% point of observation
			\tdplotsetcoord{P}{\rvec}{\thetavec}{\phivec}
			\draw[-stealth,color=red,line width=1pt] (O) -- (P) node[above right] {$\boldsymbol{r}$};
			\draw[dashed, color=red] (O) -- (Pxy);
			\draw[dashed, color=red] (P) -- (Pxy);
			% phi and theta planes (lines and tags)
			\tdplotdrawarc{(O)}{0.2}{0}{\phivec}{anchor=north}{$\varphi$}
			\tdplotsetthetaplanecoords{\phivec}
			\tdplotdrawarc[tdplot_rotated_coords]{(0,0,0)}{0.5}{0}%
			{\thetavec}{anchor=south west}{$\vartheta$}
			% plot planes
			\fill[fill=blue,opacity=0.15] (O) -- (0.2,0,0) arc (0:\phivec:0.2);
			\tdplotsetrotatedcoords{-20}{-90}{0}
			\fill[fill=blue,opacity=0.15,tdplot_rotated_coords] (O) -- (0.5,0,0) arc (0:\thetavec:0.5);    
			\end{tikzpicture}
			
		}%
	\end{animateinline}
	\end{column}		
\end{columns}
	
\end{frame}

%%%%%%%%%%%%%%%%%%%%%%%%%%%%%%%%%%%%%%%%%%%%%%%%%%%%%%%%%%%%%%%%%%%%%%%%%%%%%%%%%
\begin{frame}{About the Talk}

\vspace{-0.5cm}
\begin{itemize}
	\item Extremely wide topic. Here: \textcolor{elmagLight}{overview only!}
	\begin{itemize}
		\item From pure aesthetics, through typography, typesettings, graphics, towards colors, proportions, data processing and DTP (\small{desktop publishing}).
		\item High-level (style, stylistic, templates) to low-level (figures, tables, lists, headings),
		\item Appropriate number of seminars would span an entire semester.
		\item Instead of being complete, let's build some interest in the topic.
	\end{itemize}
	\item<2-> \textcolor{elmagLight}{what?} $\times$ \textcolor{elmagLight}{how?}
	\item<3-> Mainly for technical writing.
	\item<4-> \LaTeX{} emphasized.	
\end{itemize}

\vspace{0.25cm}
\uncover<5->{
\begin{columns}
	\begin{column}{0.1\textwidth}	
		
	\end{column}
	\begin{column}{0.8\textwidth}
		\begin{block}{}
			\centering
			Be prepared for a slow going learning curve.
		\end{block}			
	\end{column}
	\begin{column}{0.1\textwidth}
		
	\end{column}		
\end{columns}
}

\end{frame}

%%%%%%%%%%%%%%%%%%%%%%%%%%%%%%%%%%%%%%%%%%%%%%%%%%%%%%%%%%%%%%%%%%%%%%%%%%%%%%%%%
\begin{frame}{Structure of the Talk}

\vspace{-0.5cm}
\begin{columns}
	\begin{column}{0.25\textwidth}	
		
	\end{column}
	\begin{column}{0.5\textwidth}
		\begin{block}{}
			\centering
			Why?
		\end{block}			
	\end{column}
	\begin{column}{0.25\textwidth}
		
	\end{column}		
\end{columns}

\vspace{0.25cm}
\begin{itemize}
	\item Because \Quot{good enough} is not your way\dots
	\item Because you respect standards and good practice.
	\item Because quality of your work and its presentation goes hand-in-hand.
\end{itemize}

\vspace{0.25cm}
\uncover<2->{
\textcolor{elmagLight}{Only}
\begin{enumerate}
	\item nomenclature and
	\item \LaTeX.
\end{enumerate}
All other topics from writing solid paper, formatting, etc. are skipped for time reasons.
}

\end{frame}

\section{Mathematical Nomenclature}
%%%%%%%%%%%%%%%%%%%%%%%%%%%%%%%%%%%%%%%%%%%%%%%%%%%%%%%%%%%%%%%%%%%%%%%%%%%%%%%%%
\begin{frame}{Mathematical Nomenclature}

Serves
\begin{itemize}
	\item clarity,
	\item standardization.
\end{itemize}

\vspace{0.25cm}
\uncover<2->{
Known standards:
\begin{itemize}
	\item \textcolor{elmagLight}{ISO} (\small{International Organization for Standardization}),
	\item ANSI (\small{American National Standards Institute}),
	\item IEEE  (\small{Institute of Electrical and Electronics Engineers}),
	\item IUPAP (\small{International Union of Pure and Applied Physics}),
	\item \v{C}SN.
\end{itemize}
}

\end{frame}

%%%%%%%%%%%%%%%%%%%%%%%%%%%%%%%%%%%%%%%%%%%%%%%%%%%%%%%%%%%%%%%%%%%%%%%%%%%%%%%%%
\begin{frame}{ISO 80000}

\vspace{-0.75cm}
\begin{table}
	\caption{International standards for physical quantities and units, part 1.}
	%\vspace{-0.1cm}
	\begin{tabular}{lcp{66mm}p{37mm}}
		Part & Year & Name & Replaces \\ 
		\toprule
		ISO 80000-1 & 2009 & \small{\textcolor{elmagLight}{General}} & \small{ISO 31-0, IEC 60027-1, and IEC 60027-3} \\ 
		ISO 80000-2 & 2009 & \small{\textcolor{elmagLight}{Mathematical signs and symbols to be used \newline in the natural sciences and technology}} & \small{ISO 31-11, IEC 60027-1} \\ 
		ISO 80000-3 & 2006 & \small{\textcolor{elmagLight}{Space and time}} & \small{ISO 31-1 and ISO 31-2} \\ 
		ISO 80000-4 & 2006 & \small{Mechanics} & \small{ISO 31-3} \\ 
		ISO 80000-5 & 2007 & \small{Thermodynamics} & \small{ISO 31-4} \\ 
		ISO 80000-6 & 2008 & \small{\textcolor{elmagLight}{Electromagnetism}} & \small{ISO 31-5 and \mbox{IEC 60027-1}} \\ 
		ISO 80000-7 & 2008 & \small{Light} & \small{ISO 31-6} \\ 
		ISO 80000-8 & 2007 & \small{Acoustics} & \small{ISO 31-7} \\ 
		\bottomrule
	\end{tabular} 
\end{table}

\end{frame}

%%%%%%%%%%%%%%%%%%%%%%%%%%%%%%%%%%%%%%%%%%%%%%%%%%%%%%%%%%%%%%%%%%%%%%%%%%%%%%%%%
\begin{frame}{ISO 80000}

\vspace{-0.75cm}
\begin{table}
	\caption{International standards for physical quantities and units, part 2.}
	\vspace{-0.1cm}
	\begin{tabular}{lcp{66mm}p{37mm}}
		Part & Year & Name & Replaces \\ 
		\toprule
		ISO 80000-9 & 2008 & \small{Physical chemistry and molecular physics} & \small{ISO 31-8} \\ 
		ISO 80000-10 & 2009 & \small{Atomic and nuclear physics} & \small{ISO 31-9 and ISO 31-10} \\ 
		ISO 80000-11 & 2008 & \small{Characteristic numbers} & \small{ISO 31-12} \\ 
		ISO 80000-12 & 2009 & \small{Solid state physics} & \small{ISO 31-13} \\ 
		ISO 80000-13 & 2008 & \small{Information science and technology} & \small{IEC 60027-2:2005 and IEC 60027-3} \\ 
		ISO 80000-14 & 2008 & \small{Telebiometrics related to human physiology} & \small{IEC 60027-7} \\ 
		\bottomrule
	\end{tabular}
\end{table}

\begin{itemize}
	\item SI units (not only) used.
	\item One unit is \EUR{138}.
\end{itemize}

\end{frame}

\section{Nomenclature -- Rules}
%%%%%%%%%%%%%%%%%%%%%%%%%%%%%%%%%%%%%%%%%%%%%%%%%%%%%%%%%%%%%%%%%%%%%%%%%%%%%%%%%
\begin{frame}{Variables and Units}

\vspace{-0.5cm}
\begin{equation*}
f_0 = \left\{f_\T{quantity} \right\} \left[f_\T{unit} \right] = 12\,345(67)\,\T{Hz}
\end{equation*}

\vspace{-0.25cm}
\begin{itemize}
	\item<2-> Quantity always in \textbf{italic}.
	\begin{itemize}
		\item<3-> Note that $12\,345 \pm 67\,\T{Hz}$ is incorrect from mathematical point of view.
	\end{itemize}
	\item<4-> Unit always in \textbf{roman}.
	\begin{itemize}
		\item<5-> A short space (\texttt{\textbackslash ,} in \LaTeX) placed between the quantity and the unit symbol (except the units of degree, minute, and second).
		\item<6-> Units are always in lowercase (meter, second), except those derived from a proper name of a person (Tesla, Volt) and symbols containing signs in exponent position (\degree C).
		\item<7-> Different units are separated by a space (N\,m not Nm) or a c-dot ($1\,\T{N}\cdot\T{m}$).
		\item<8-> Prefixes are written in roman with no space between symbol and prefix ($1\,\T{THz}$ vs. $1\,\T{T\,Hz}$ vs. 1 T Hz vs. 1 THz).
		\item<9-> $l = 1.31 \times 10^3\,\T{m}$, $l = 1.31 \cdot 10^3\,\T{m}$, $S = 20\,\T{m} \times 30\,\T{m}$.
	\end{itemize}
\end{itemize}

\end{frame}

%%%%%%%%%%%%%%%%%%%%%%%%%%%%%%%%%%%%%%%%%%%%%%%%%%%%%%%%%%%%%%%%%%%%%%%%%%%%%%%%%
\begin{frame}{Decimal Sign and Exponents}

\begin{itemize}
	\item Decimal sign is either a comma or a point ($1,234$ or $1.234$).
	\item<2-> Numbers can be grouped from the decimal sign or from left (12\,345.678\,9 or 1\,234), use small space then.
	\item<3-> Negative exponents should be avoided when the numbers are used, except when the base 10 is used ($10^{-5}$ not $4^{-8}$, type $1/4^8$ instead).
	\item<4-> Multiplication with $\cdot$ or $\times$. Do not use any symbol for products like $ab$, $\M{A}\M{x}$, etc. Use when multiplication operation has to be highlighted, \ie{}, multi-line equation or $2.125 \cdot 10^8$.
	\item<5-> Number of significant digits ($410\,008$ vs $410\,000$ vs $4.1\cdot 10^5$).
\end{itemize}

\vspace{0.25cm}
\uncover<6->{
\begin{columns}
	\begin{column}{0.33\textwidth}	
		\centering		
		\href{https://en.wikipedia.org/wiki/Unit_prefix}{\beamergotobutton{\texttt{Unit prefixes}}}		
	\end{column}
	\begin{column}{0.33\textwidth}	
		\centering		
		\href{https://en.wikipedia.org/wiki/List_of_mathematical_symbols}{\beamergotobutton{\texttt{Mathematical symbols}}}	
	\end{column}
	\begin{column}{0.33\textwidth}
		\centering		
		\href{https://physics.nist.gov/cuu/pdf/sp811.pdf}{\beamergotobutton{\texttt{Guide for the use of SI units}}}	
	\end{column}		
\end{columns}
}

\end{frame}

%%%%%%%%%%%%%%%%%%%%%%%%%%%%%%%%%%%%%%%%%%%%%%%%%%%%%%%%%%%%%%%%%%%%%%%%%%%%%%%%%
\begin{frame}{Constants}

\vspace{-0.5cm}
\begin{description}[mathematical]
	\item[mathematical] Dimensionless with fixed numerical value of no direct physical meaning or necessity of a physical measurement.
	\begin{itemize}
		\item<2-> Examples: Archimedes' constant ($\pi$), Euler's number ($\T{e}$), imaginary unit ($\J$).
	\end{itemize}
	\item[physical] Often carry dimensions, they are universal and constant in time.
	\begin{itemize}
		\item<2-> Examples: speed of light in vacuum ($c_0$), electron charge ($e$), permittivity of vacuum ($\EPS$), impedance of vacuum ($\ZVAC$).
	\end{itemize}
\end{description}

\vspace{0.25cm}
\uncover<3->{
\begin{description}[mathematical]
	\item[mathematical] always in \textcolor{elmagLight}{roman} type, \ie{}, $\T{e}^{\J \pi} + 1 = 0$
	\item[physical] always in \textcolor{elmagLight}{italic} type, \ie{}, $2c_0$, \cf{} $\T{e}^2$ vs. $e^2$
\end{description}
}

\end{frame}

%%%%%%%%%%%%%%%%%%%%%%%%%%%%%%%%%%%%%%%%%%%%%%%%%%%%%%%%%%%%%%%%%%%%%%%%%%%%%%%%%
\begin{frame}{Functions}

\vspace{-0.5cm}
Functions always in \textcolor{elmagLight}{roman}, they are not variables!

\vspace{0.25cm}
$\sin \left(xy\right)$, $y \sin x$

$\T{j}_1 \left(x\right)$, $-\J \T{j}_1 \left(x\right)$

$\lim_{x\rightarrow \infty} f \left(x\right)$

\vspace{0.5cm}
\begin{columns}
	\begin{column}{0.1\textwidth}	
		
	\end{column}
	\begin{column}{0.8\textwidth}
		\begin{block}{}
			\centering
			Use parentheses whenever clarity is in question.
		\end{block}			
	\end{column}
	\begin{column}{0.1\textwidth}
		
	\end{column}		
\end{columns}

\end{frame}

%%%%%%%%%%%%%%%%%%%%%%%%%%%%%%%%%%%%%%%%%%%%%%%%%%%%%%%%%%%%%%%%%%%%%%%%%%%%%%%%%
\begin{frame}{Sub- and Superscripts}

\vspace{-0.5cm}
\begin{itemize}
	\item \textcolor{elmagLight}{Italic}: index represents an unknown variable or a running number/index/counter:
	\begin{itemize}
		\item $\sum_n \alpha_n f_n \left(x\right)$, $c_i$, $z_{mn}$, $\M{u}_{\tau \rho m l}^{\left(p\right)} \left(kr\right)$.
	\end{itemize}
	\item \textcolor{elmagLight}{Roman}: index represents a number or an abbreviation:
	\begin{itemize}
		\item $\varepsilon_\T{r}$, $c_0$, $\Prad$, $Q_\T{lb}$.
	\end{itemize}
	\item Should not be overused ($n^{m^{k^{l}}}_0$).
\end{itemize}

\vspace{0.25cm}
\uncover<2->{
\begin{enumerate}
	\item Whenever possible, simplify and shorten, \ie{}, $\V{n}_0 \rightarrow \UV{n}$, $P_\T{radiated} \rightarrow \Prad$.
	\item Prioritize clarity, consistence.
\end{enumerate}
}

\end{frame}

%%%%%%%%%%%%%%%%%%%%%%%%%%%%%%%%%%%%%%%%%%%%%%%%%%%%%%%%%%%%%%%%%%%%%%%%%%%%%%%%%
\begin{frame}{In-line and Full Equations}

\vspace{-0.5cm}
Different approach needed, \cf{}

\begin{columns}
	\begin{column}{0.5\textwidth}	
\begin{equation*}
\frac{a}{b}
\end{equation*}
	\end{column}
	\begin{column}{0.5\textwidth}	
$a/b$
	\end{column}
\end{columns}

\begin{columns}
\begin{column}{0.5\textwidth}	
	\begin{equation*}
	\lim\limits_{x\rightarrow \infty} f \left(x\right)
	\end{equation*}
\end{column}
\begin{column}{0.5\textwidth}	
	$\lim_{x\rightarrow \infty} f \left(x\right)$
\end{column}
\end{columns}

\begin{columns}
\begin{column}{0.5\textwidth}	
	\begin{equation*}
	\T{e}^{-\J\omega t}
	\end{equation*}
\end{column}
\begin{column}{0.5\textwidth}	
	$\exp\left\{-\J\omega t\right\}$
\end{column}
\end{columns}

\begin{columns}
	\begin{column}{0.5\textwidth}	
		\begin{equation*}
		\int\limits_0^{2 \pi} \frac{x}{x + a} \D{x}
		\end{equation*}
	\end{column}
	\begin{column}{0.5\textwidth}	
		$\int_0^{2\pi} x / \left(x + a\right) \D{x}$
	\end{column}
\end{columns}

\vspace{0.25cm}
\uncover<2->{
\begin{itemize}
	\item In-line equations prioritize space-saving strategy.
	\item Equations are always a part of the text.
\end{itemize}
}

\end{frame}

%%%%%%%%%%%%%%%%%%%%%%%%%%%%%%%%%%%%%%%%%%%%%%%%%%%%%%%%%%%%%%%%%%%%%%%%%%%%%%%%%
\begin{frame}{Integration}

\vspace{-0.5cm}
A small space between integrand and differential, differential roman typed:
\begin{equation*}
\frac{1}{T} \int\limits_{t}^{t+T} \int\limits_\srcRegion f \left(\V{r}, t \right) \D{\T{V}} \D{t}, \quad \V{r} \in \srcRegion.
\end{equation*}

\begin{itemize}
	\item Be careful about in-line and full equations, \ie{}, usage of $\int$ and $\displaystyle \int$.
	\item Limits of integral are written over and under the symbol, unless spatial requirements prevents it (in-line eq.).
	\item The variable of integration shall be written in italics if it relates to a coordinate system or if the integration domain has explicitly defined limits, roman otherwise.
\end{itemize}

\end{frame}

%%%%%%%%%%%%%%%%%%%%%%%%%%%%%%%%%%%%%%%%%%%%%%%%%%%%%%%%%%%%%%%%%%%%%%%%%%%%%%%%%
\begin{frame}{Differentiation}

\begin{equation*}
\frac{\T{d} f\left(x\right)}{\T{d}x}
\end{equation*}

\begin{equation*}
\nabla \cdot \V{J} \left(\V{r}\right) = - \frac{\partial \rho \left(\V{r}\right)}{\partial t}
\end{equation*}

\uncover<2->{
For fans: partial derivative should be rotated to be typed roman.

\vspace{0.5cm}
\href{https://www.tug.org/TUGboat/tb18-1/tb54becc.pdf}{\beamergotobutton{\texttt{Typesetting mathematics for science, Beccari C., 1997}}}
}

\end{frame}

%%%%%%%%%%%%%%%%%%%%%%%%%%%%%%%%%%%%%%%%%%%%%%%%%%%%%%%%%%%%%%%%%%%%%%%%%%%%%%%%%
\begin{frame}{Usage of Equations, Part 1}

\vspace{-0.5cm}
Be careful about the details
\begin{equation*}
f = \frac{1}{1 + \frac{\pi}{2}n} \quad \T{vs.} \quad f = \frac{1}{1 + \displaystyle\frac{\pi}{2}n}.
\end{equation*}

\vspace{0.25cm}
Keep in mind that equation is always a part of the text, \ie{},
\begin{equation*}
g = x \left(\frac{n}{2} + \left(k^2 - 2 \left(x - 3\right)\right)\right) \quad \T{vs.} \quad g = x (\frac{n}{2} + (k^2 - 2 (x - 3))),
\end{equation*}
and no matter if properly typed (left) or not (right).

\vspace{0.25cm}
$\V{r}_1 \cdot \V{r}_2$, $\V{r}_1 \times \V{r}_2$, $\pm 5$, $f'$, $f''$

\begin{itemize}
	\item MathType can be used for initial code generation.
\end{itemize}

\end{frame}

%%%%%%%%%%%%%%%%%%%%%%%%%%%%%%%%%%%%%%%%%%%%%%%%%%%%%%%%%%%%%%%%%%%%%%%%%%%%%%%%%
\begin{frame}{Usage of Equations, Part 2}

\vspace{-0.5cm}
Complex numbers:
\begin{equation*}
z = \overbrace{
	\underbrace{x}_\T{real} + \J
	\underbrace{y}_\T{imaginary}
}^\T{complex\,\,number} = \RE \left\{z\right\} + \J \IM \left\{z\right\},
\end{equation*}
not $\Re\left\{z\right\} + \J \Im\left\{z\right\}$ (this is obsolete).

\vspace{0.25cm}
\begin{itemize}
	\item<2-> Transpose $\M{A}^\trans$, complex conjugate $z^\ast$, Hermitian conjugate $\left(\M{A}^\ast\right)^\trans \equiv \M{A}^\herm$.
	\item<3-> More equations are always separated (\eg{}, by a comma).
	\item<4-> Physical units always on the same line as the equation.
	\item<5-> Prepositions and conjunctions should not be alone at the end of the line.
\end{itemize}

\uncover<6->{
\vspace{0.25cm}
\href{http://tug.ctan.org/info/symbols/comprehensive/symbols-a4.pdf}{\beamergotobutton{\texttt{The comprehensive \LaTeX symbol list}}}
}

\end{frame}

%%%%%%%%%%%%%%%%%%%%%%%%%%%%%%%%%%%%%%%%%%%%%%%%%%%%%%%%%%%%%%%%%%%%%%%%%%%%%%%%%
\begin{frame}{Vectors and Matrices}

\vspace{-1cm}
\begin{table}
\begin{center}
\caption{Scalars, vectors, dyads, matrices, and unit vectors.}
\begin{tabular}{lc}
\toprule 
$a$ & a scalar number \\ 
$a_m$ & an element of a vector~$\M{a}$ \\
$a_{mn}$ & an element of a matrix~$\M{A}$ \\
$\M{a}$	& a vector \\
$\V{a}$	& a vector function \\
$\M{a}_n$& a column of a matrix \\
$\UV{a}$ & unit vector \\ 
$\M{A}$	& a matrix \\ 
$\V{A}$	& a (time-harmonic) vector function, phasor \\ 
$\OP{A}$ & a functional or a time-dependent function \\ 
$\V{\OP{A}}$ & a vector time-dependent function \\ 
$\mathbb{A}$ & a field, a domain \\
\bottomrule
\end{tabular} 
\end{center}
\end{table}

\end{frame}

%%%%%%%%%%%%%%%%%%%%%%%%%%%%%%%%%%%%%%%%%%%%%%%%%%%%%%%%%%%%%%%%%%%%%%%%%%%%%%%%%
\begin{frame}{Brackets}

\vspace{-1.25cm}
\begin{table}
\begin{center}
\caption{Brackets and their usage (personal preference).}
\begin{tabular}{lcc}
\toprule 
$\left(\,\right)$ & $x \left(x + 2 \right)$ & structuring of an equation \\ 
& $f \left(x\right)$ & arguments of a function \\
& $x \in \left(0,1\right)$ & an open interval \\ \midrule
$\left[\,\right]$ & $\left[x_1 \,\,\, x_2 \,\,\, \cdots \,\,\, x_n\right]^\trans$ & a vector, a matrix \\
 & $x \in \left[0, 5\right]$ & a closed interval \\ \midrule
$\left\{\,\right\}$ & $n \in \left\{1, \dots, N\right\}$ & set operations \\
& $\OP{L} \left\{ \V{J}_1 \left(\V{r} \right), \V{J}_2 \left(\V{r} \right)\right\}$ & arguments of operators and transformations \\ \midrule
$\left\langle\,\right\rangle$ & $\left\langle \V{x}, \OP{L} \left\{ \V{x} \right\} \right\rangle$ & inner product \\
&  $\left\langle \phi | \psi \right\rangle$ & bra--ket \\ \midrule
$|\,|$ & $| \V{x} |$ & absolute value, modulus \\ \midrule
$\lceil\,\rceil$, $\lfloor\,\rfloor$ & $\lceil x \rceil$, $\lfloor x \rfloor$ & ceiling, floor \\
\bottomrule
\end{tabular} 
\end{center}
\end{table}

\end{frame}

%%%%%%%%%%%%%%%%%%%%%%%%%%%%%%%%%%%%%%%%%%%%%%%%%%%%%%%%%%%%%%%%%%%%%%%%%%%%%%%%%
\begin{frame}{Matrix Typesetting}

\vspace{-0.5cm}
Linear system $\M{y} = \M{A} \M{x}$, quadratic form $y = \M{x}^\herm \M{A} \M{x}$.

\begin{equation*}
\Cmat_\setB =
\left[ {\begin{array}{ccccc}
	1 & 0 & 0 & \cdots & 0\\
	\end{array} } \right]^\trans
\end{equation*}

\begin{equation*}
\Cmat_\setB R_\infty \Cmat_\setB^\trans =
\left[ {\begin{array}{ccccc}
	R_\infty & 0 & 0 & \cdots & 0\\
	0 & 0 & 0 & \cdots & 0\\		
	0 & 0 & R_\infty & \cdots & 0\\
	\vdots & \vdots & \vdots & \ddots & \vdots \\
	0 & 0 & 0 & \cdots & 0\\		
\end{array} } \right]
\end{equation*}

\end{frame}

%%%%%%%%%%%%%%%%%%%%%%%%%%%%%%%%%%%%%%%%%%%%%%%%%%%%%%%%%%%%%%%%%%%%%%%%%%%%%%%%%
\begin{frame}{System of Equations, Complicated Equations}

\vspace{-0.75cm}
\begin{align}
f(x) &= x^4 + 7x^3 + 2x^2 \nonumber \\
&\qquad {} + 10x + 12
\end{align}

\vspace{-0.5cm}
\begin{align}
f(x)  &= a x^2+b x +c \\
f'(x) &= 2 a x +b
\end{align}

\vspace{-0.2cm}
\begin{equation*}
C_{\setB,nn} = \left\{
\begin{array}{lll}
0 & \Leftrightarrow & n \not\in \setB \\
1 & \Leftrightarrow & \mathrm{otherwise} \\
\end{array}
\right.
\end{equation*}

\begin{columns}
\begin{column}{0.1\textwidth}	
	
\end{column}
\begin{column}{0.8\textwidth}
	\begin{block}{}
		\centering
		When you are not sure, google it out! (\texttt{tex.stackexchange.com})
	\end{block}			
\end{column}
\begin{column}{0.1\textwidth}
	
\end{column}		
\end{columns}

\end{frame}

%%%%%%%%%%%%%%%%%%%%%%%%%%%%%%%%%%%%%%%%%%%%%%%%%%%%%%%%%%%%%%%%%%%%%%%%%%%%%%%%%
\begin{frame}{Some Hints}{Leslie's Corner}

\vspace{-0.5cm}
\begin{enumerate}
\item \Quot{the free space} (not \Quot{free space})
\item \Quot{wave-number} (not \Quot{wavenumber} or \Quot{wave number})
\item \Quot{the speed of light} (not \Quot{speed of light})
\item \Quot{Poynting's theorem} (not \Quot{Poynting theorem})
\item \Quot{Maxwell's equations} (not \Quot{Maxwell equations})
\item \Quot{energy in a vacuum} (not \Quot{energy in vacuum})
\item \Quot{state-of-the-art} (not \Quot{state of the art})
\item and many, many others\dots
\end{enumerate}

\end{frame}

\section{\LaTeX}
%%%%%%%%%%%%%%%%%%%%%%%%%%%%%%%%%%%%%%%%%%%%%%%%%%%%%%%%%%%%%%%%%%%%%%%%%%%%%%%%%
\begin{frame}{About \LaTeX}

\vspace{-0.5cm}
Document preparation system, opened, for free
\begin{itemize}
	\item To allow anybody to produce high-quality books using minimal effort,
	\item to provide a system that would give exactly the same results on all computers.
\end{itemize}
	
\vspace{0.5cm}
\LaTeX = Lamport \TeX
\begin{description}
	\item[\TeX] Donald Knuth, 1st release: 1978
	\begin{itemize}
		\item \TeX = $\tau \epsilon \chi$ $\rightarrow$ \Quot{t$\varepsilon$x} or \Quot{t$\varepsilon$k}
	\end{itemize}
	\item[\LaTeX] Leslie Lamport, 1st release: 1984
	\begin{itemize}
		\item \Quot{la:t$\varepsilon$x} or \Quot{leit$\varepsilon$x}
	\end{itemize}
\end{description}

%\vspace{0.25}
%Personal experience: all teaching/scientific documents in Lund written in \LaTeX.

\end{frame}

%%%%%%%%%%%%%%%%%%%%%%%%%%%%%%%%%%%%%%%%%%%%%%%%%%%%%%%%%%%%%%%%%%%%%%%%%%%%%%%%%
\begin{frame}{MS Office \textit{Contra} \LaTeX}

\vspace{-0.5cm}
Matter of taste (and professional honor).

\begin{columns}[t]
\begin{column}{0.475\textwidth}
	\begin{block}{Features favoring MS Office}
		\begin{itemize}
			\item Requires almost no skills and knowledge.
			\item Linear learning curve.
			\item May be \Quot{good enough} approach if one is not concerned about quality.
		\end{itemize}
	\end{block}
\end{column}
\begin{column}{0.475\textwidth}
	\begin{block}{Features favoring \LaTeX}
		\begin{itemize}
			\item Open-source (for free).
			\item Typesetting (fonts, kerning, math).
			\item Well documented.
			\item All (text, math, figures) in the same environment.
			\item 100\,\% controllability.
			\item Can be heavily automated.
			\item Movable and inter-media content.
			\item Superb outputs.
		\end{itemize}
	\end{block}
\end{column}
\end{columns}

\end{frame}

%%%%%%%%%%%%%%%%%%%%%%%%%%%%%%%%%%%%%%%%%%%%%%%%%%%%%%%%%%%%%%%%%%%%%%%%%%%%%%%%%
\begin{frame}{Conception}

\vspace{-0.5cm}
Distribution (\eg{}, MikTeX) + Packages (\eg{}, Amsmath) + Style/template files (sty, cls)

\vspace{0.5cm}
To learn: \\
\LaTeX, Overleaf, data processing, Beamer, PGFplot and Ti$k$Z.

\vspace{0.25cm}
To start with:
\begin{columns}
	\begin{column}{0.33\textwidth}	
		\centering
		\href{https://en.wikibooks.org/wiki/LaTeX/Basics}{\beamergotobutton{\texttt{\LaTeX basics}}}		
	\end{column}
	\begin{column}{0.33\textwidth}	
		\centering		
		\href{https://www.overleaf.com/learn/latex/Learn_LaTeX_in_30_minutes}{\beamergotobutton{\texttt{\LaTeX in 30 minutes}}}	
	\end{column}
	\begin{column}{0.33\textwidth}
		\centering		
		\href{https://www.latex4technics.com/}{\beamergotobutton{\texttt{On-line equations}}}	
	\end{column}		
\end{columns}

\end{frame}

%%%%%%%%%%%%%%%%%%%%%%%%%%%%%%%%%%%%%%%%%%%%%%%%%%%%%%%%%%%%%%%%%%%%%%%%%%%%%%%%%
\begin{frame}{Packages to Get}

\vspace{-0.5cm}
\begin{columns}[t]
	\begin{column}{0.475\textwidth}
		\begin{block}{Must have}
	\begin{enumerate}
		\item \LaTeX{} distribution \href{http://miktex.org/download}{\beamergotobutton{\texttt{MikTeX}}}
		\item \LaTeX{} editor \href{http://www.texstudio.org/}{\beamergotobutton{\texttt{TeXstudio}}}
		\item \LaTeX{} packaged (can be installed on the fly)
		\item Spell-checker \href{http://www.rohitfarmer.in/computing/linux-and-open-source/how-to-make-dictionary-work-in-texstudio/}{\beamergotobutton{\texttt{How to install}}}
		\item Reference database editor \href{http://www.jabref.org/}{\beamergotobutton{\texttt{JabRef}}}											
	\end{enumerate}
		\end{block}
	\end{column}
	\begin{column}{0.475\textwidth}
		\begin{block}{Optional}
	\begin{enumerate}	
		\item GhostScript \href{http://www.ghostscript.com/download/}{\beamergotobutton{\texttt{GhostScript}}}
		\item GhostViewer \href{http://pages.cs.wisc.edu/~ghost/gsview/}{\beamergotobutton{\texttt{GhostViewer}}}
		\item GNUplot \href{http://www.gnuplot.info/download.html}{\beamergotobutton{\texttt{GNUplot}}}
		\item Matlab2Ti$k$Z \href{https://www.mathworks.com/matlabcentral/fileexchange/22022-matlab2tikz-matlab2tikz}{\beamergotobutton{\texttt{Matlab2Ti$k$Z}}}
		\item GeoZebra \href{https://www.geogebra.org/}{\beamergotobutton{\texttt{GeoZebra}}}
		\item MeshLab \href{http://www.meshlab.net/}{\beamergotobutton{\texttt{MeshLab}}}
		\item ParaView \href{https://www.paraview.org/}{\beamergotobutton{\texttt{ParaView}}}
		\item Asymptote \href{http://asymptote.sourceforge.net/}{\beamergotobutton{\texttt{Asymptote}}}
	\end{enumerate}
		\end{block}
	\end{column}
\end{columns}

\vspace{0.25cm}
Codes from MATLAB fileexchange (\texttt{mcode}, \texttt{cbrewer}, \texttt{fig2u3d}, \texttt{vrml}, \texttt{export\_fig}).

\end{frame}

%%%%%%%%%%%%%%%%%%%%%%%%%%%%%%%%%%%%%%%%%%%%%%%%%%%%%%%%%%%%%%%%%%%%%%%%%%%%%%%%%
\begin{frame}{A Few Highlights}

\vspace{-0.5cm}

\begin{itemize}
	\item citations
	\item math
	\item acronyms
	\item internal references (equations, figures, tables)
	\item index
\end{itemize}

\end{frame}

%%%%%%%%%%%%%%%%%%%%%%%%%%%%%%%%%%%%%%%%%%%%%%%%%%%%%%%%%%%%%%%%%%%%%%%%%%%%%%%%%
\begin{frame}{Lists}

\vspace{-0.5cm}
A list can be either
\begin{itemize}
\item a long sentence
\item or a set of independent bullets.
\end{itemize}

\vspace{0.5cm}
\begin{columns}[t]
\begin{column}{0.33\textwidth}	
	itemization 
	\begin{itemize}
		\item no numbering
		\item most common
		\item[{\includegraphics[height=0.4cm]{ElmaglogoComplete.png}}] user-defined bullet symbols
	\end{itemize}
\end{column}
\begin{column}{0.33\textwidth}	
	enumeration
	\begin{enumerate}
		\item numbered
		\item different numbering possible (A,B,\dots)
		\item when order or amount is of interest
	\end{enumerate}
\end{column}
\begin{column}{0.33\textwidth}	
	description
	\begin{description}
		\item[difference] bullet symbol is a word or a sentence
		\item[usage] for descriptive lists
	\end{description}
\end{column}
\end{columns}

\vspace{0.25cm}
Ellipsis: \dots (not ...); a space before and/or after is a matter of used style.

Notice that for math we have $\cdots$, $\vdots$, $\ddots$.

\end{frame}

%%%%%%%%%%%%%%%%%%%%%%%%%%%%%%%%%%%%%%%%%%%%%%%%%%%%%%%%%%%%%%%%%%%%%%%%%%%%%%%%%
\begin{frame}[t]{Capitalization}

\vspace{-0.5cm}
\begin{columns}[t]
	\begin{column}{0.5\textwidth}	
		We \textcolor{elmagLight}{do} capitalize
		\begin{itemize}
			\item nouns (man, bus, book),
			\item adjectives (angry, lovely, small),
			\item verbs (run, eat, sleep),
			\item adverbs (slowly, quickly, quietly),
			\item pronouns (he, she, it),
			\item subordinating conjunctions (as, because, that).
		\end{itemize}
	\end{column}
	\begin{column}{0.5\textwidth}
		We \textcolor{elmagLight}{do not} capitalize	
		\begin{itemize}
			\item articles: a, an, the,
			\item coordinating conjunctions: and, but, or, for, nor, etc.,
			\item prepositions (fewer than five letters): on, at, to, from, by, etc.
		\end{itemize}
	\end{column}		
\end{columns}

If you capitalize, then no full stop.

\vspace{0.25cm}
\href{https://capitalizemytitle.com/}{\beamergotobutton{\texttt{Title capitalization}}}
\end{frame}

%%%%%%%%%%%%%%%%%%%%%%%%%%%%%%%%%%%%%%%%%%%%%%%%%%%%%%%%%%%%%%%%%%%%%%%%%%%%%%%%%
\begin{frame}{Dash $\times$ Hyphen}

\vspace{-0.5cm}
We differentiate between
\begin{description}[em dash \Quot{---}]
\item[em dash \Quot{---}] punctuation (yes---or no?),
\item[en dash \Quot{--}] range (6--10 days, pp. 40--42),
\item[hyphen \Quot{-}] connects two words (front-end),
\item[minus \Quot{$-$}] math ($a-b$).
\end{description}

\vspace{0.25cm}
Quotation is \Quot{this}, not "this" or 'this'.

``Nested `quotation'{}'' or ``nested `quotation'\thinspace'', but not ``nested `quotation'''.

\end{frame}

\section{Style}
%%%%%%%%%%%%%%%%%%%%%%%%%%%%%%%%%%%%%%%%%%%%%%%%%%%%%%%%%%%%%%%%%%%%%%%%%%%%%%%%%
\begin{frame}{Stylistic and Style}

gutter

panchart

hyphenation

kerning

fonts

\ie{}, \eg{}, \cf{}, etc.

viz = see

vs. $\times$ vs (vs. possible as well)

\begin{itemize}
	\item Self-study of books, forums, personal interest needed.
\end{itemize}

\end{frame}

\section{Next Week}
%%%%%%%%%%%%%%%%%%%%%%%%%%%%%%%%%%%%%%%%%%%%%%%%%%%%%%%%%%%%%%%%%%%%%%%%%%%%%%%%%
\begin{frame}{Next week(s)}

\begin{itemize}
	\item \LaTeX and Overleaf (on-line collaborative \LaTeX writing).
	\item Elements of data processing and Ti$k$Z (graphics, colors and color maps, figures,\dots).
	\item How to get data from MATLAB to Tik$k$Z, how to externalize data.
\end{itemize}

\vspace{0.5cm}
In the following weeks:
\begin{itemize}
	\item Graphical and stylistic manual of the department (math commands?).
	\item Beamer (a \LaTeX class for creating presentations, Beamer template?).
\end{itemize}

\end{frame}

% CLOSING SECTIONS:
%%%%%%%%%%%%%%%%%%%%%%%%%%%%%%%%%%%%%%%%%%%%%%%%%%%%%%%%%%%%%%%%%%%%%%%%%%%%%%%%%
%\begin{frame}
%\printbibliography % to enumerate all references
%\end{frame}
%%%%%%%%%%%%%%%%%%%%%%%%%%%%%%%%%%%%%%%%%%%%%%%%%%%%%%%%%%%%%%%%%%%%%%%%%%%%%%%%%
\section*{}
\begin{frame}
	%\vfill
	\vspace{12mm}
	\begin{center}
		\huge{\textcolor{elmagDark}{Questions?}} \\
		\vspace{15mm} 
		\small{For a complete PDF presentation see \href{www.capek.elmag.org}{\beamergotobutton{\texttt{capek.elmag.org}}}} \\
		\vspace{15mm}
		\normalsize{Miloslav \v{C}apek \\
			% \fontsize{8}{5}{\myEmail} \\
			\myEmail \\
			\vspace{3mm}
			\myDate, \myVersion}
	\end{center}
	\vfill
\end{frame}
\end{document}