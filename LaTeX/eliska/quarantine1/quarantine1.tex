\documentclass[11pt,a4paper]{report}
\setlength\textwidth{145mm}
\setlength\textheight{247mm}
\setlength\oddsidemargin{15mm}
\setlength\evensidemargin{15mm}
\setlength\topmargin{0mm}
\setlength\headsep{0mm}
\setlength\headheight{0mm}
% \openright zařídí, aby následující text začínal na pravé straně knihy
\let\openright=\clearpage

\usepackage[czech]{babel}
\usepackage[utf8]{inputenc}
\usepackage[T1]{fontenc}

\usepackage{array}
\usepackage{mathtools}
\usepackage{stddoc}



\begin{document}
	
	\pagenumbering{gobble}
	
	\section*{Zásilka matematiky do karantény č. 1}
	\noindent\rule{12cm}{0.2pt}
	
	\begin{itemize}
		
		\item 34/22
			\begin{enumerate}
				
				\item $\log_8(x) > 0$ \\
					Jelikož vim, že platí $\log_8(1) = 0$ (protože $a^0 = 1$ pro každý číslo $a$), tak můžu rovnici přepsat jako
					\begin{align*}
						\log_8(x) > \log_8(1).
					\end{align*}
					Dál vím, že logaritmus je funkce prostá, proto můžu bez problémů zavolat inversi (odlogaritmovat) a rovnou vychází výsledek
					\begin{align*}
						x &> 1,
					\\
						\Aboxed{x &\in (1, \infty).}
					\end{align*}
				
				\item $\log_{\frac 54}(x) \leq 0$ \\
					Stejný proces:
					\begin{align*}
						\log_{\frac 54}(x) &\leq \log_{\frac 54}(1),
					\\
						x &\leq 1,
					\\
						\Aboxed{x &\in (0, 1 \rangle.}
					\end{align*}
					Stále platí $x > 0$ z definičního oboru, proto interval $(0, 1 \rangle.$ a ne $(-\infty, 1 \rangle.$
				
				\item $\log_{0.8}(x) > 0$ \\
					Tady pozor, základ je menší než jedna ($0.8 < 1$), takže graf funkce není ten rostoucí, ale klesající, což způsobí to, že při odlogaritmování musíme přehodit znaménko nerovnosti.
					\begin{align*}
						\log_{0.8}(x) &> \log_{0.8}(1),
					\\
						x &< 1,
					\\
						\Aboxed{x &\in (0, 1).}
					\end{align*}
					
				\item $\log_{\sqrt{2}}(x) \leq 0$ \\
					\begin{align*}
						\log_{\sqrt{2}}(x) &\leq \log_{\sqrt{2}}(1),
					\\
						x &\leq 1,
					\\
						\Aboxed{x &\in (0, 1 \rangle.}
					\end{align*}
				
			\end{enumerate}
		
		\item 35/23 je v pracovním sešitě řešený, to asi nemusim dělat.
		
		\item 35/24
			Celá myšlenka tohoto cvičení (a 35/23 taky) je, aby sis zvykla na to, jak vypadá graf logaritmický funkce a to i v případě, že základ bude menší než jedna (vždycky ale větší než 0! ...logaritmus se záporným základem je divná funkce). Z těch grafů pak na základě toho, jestli roste (základ větší než 1) nebo klesá (základ mezi nulou a jedničkou), můžeme vyvodit porovnání hodnot. Pro rostoucí přece platí, že čím větší je x (číslo v závorce za log), tím větší je hodnota. U záporných funkcí je tomu však naopak.
			\begin{enumerate}
				\item $\boxed{\log_2(5) > \log_2(1)}$, protože $5 > 1$ a $\log_2(x)$ je rostoucí,
				\item $\boxed{\log_{\frac 12}(5) < 1} = \log_{\frac 12}\(\frac 12\)$, protože $5 > \frac 12$ a $\log_{\frac 12}(x)$ je klesající,
				\item $0 = \boxed{\log_2(1) = \log_{\frac 12}(1)} = 0$,
				\item $\boxed{\log_2 \( \frac 32 \) > \log_2 \( \frac 23 \)}$,
				\item $\boxed{\log_{\frac 13}(9) < 0}$, protože logaritmus se základem menším než 0 je vždy záporný, \\ \\  když $x > 1$ (zde $x = 9$),
				\item $\boxed{\log_2 \( \sqrt 2 \) > \log_2(1)}$, protože $\sqrt 2 > 1$,
				\item $\boxed{\log_{\frac 12} \( \sqrt 2 \) < 0} = \log_{\frac 12}(1)$, protože $\sqrt 2 > 1$,
				\item $\boxed{\log_{\frac 12} \( 2^{-4} \) > \log_{\frac 12} \( 2^{-3} \)}$,
				\item $\log_{\mathrm e}(\mathrm e) = \boxed{\ln(\mathrm e) = 1}$,
				\item $\log_{\mathrm e}(\mathrm e) = 1 = \boxed{\ln(\mathrm e) < \log_2(\mathrm e)}$,
				\item $0 = \log_{\mathrm e}(1) = \boxed{\ln(1) = \log(1)} = \log_{10}(1) = 0$.
			\end{enumerate}
		
		\item 38/30 řešený.
		
		\item 38/31 graf na Desmosu.
		
		\item 38/32 $f: x \mapsto \log_3(x-9)$ (to je trochu víc fancy napsaný $f: y = \log_3(x-9)$)
			\begin{enumerate}
				\item ANO, je roustoucí,
				\item NE, není omezená,
				\item ANO, je prostá,
				\item NE, není sudá,
				\item NE, není lichá,
				\item NE, je rovna 0,
				\item NE, jejím definičním oborem je interval $D(f) = (9, \infty)$.
			\end{enumerate}
		
		\item 39/33 graf na Desmosu.
		
		\item 39/34 graf na Desmosu + c) $D(f) = (-3, \infty)$, $H(f) = \mathbb R$.
		
		\item 39/35 c).
		
		\item 36/26
			\begin{enumerate}
				
				\item $f(x) = \log_{0.2}(2x-4)$
					\begin{align*}
						2x - 4 &> 0,
					\\
						2x &> 4,
					\\
						x &> 2,
					\\
						\Aboxed{x &\in (2, \infty).}
					\end{align*}
				
				\item $f(x) = \log_3 \( \sqrt{3 - x} \)$
					\begin{align*}
						\sqrt{3-x} &> 0,
					\\
						3-x &> 0,
					\\
						x & < 3,
					\\
						\Aboxed{x &\in (-\infty, 3).}
					\end{align*}
				
				\item $f(x) = \ln \( \frac{4}{x+5} \)$
					\begin{align*}
						\frac{4}{x+5} &> 0,
						& x \not= -5,
					\\
						x+5 &> 0,
					\\
						x &> -5,
					\end{align*}
					\begin{align*}
						\Aboxed{x \in (-5, \infty).}
					\end{align*}
				
				\item $f(x) = \log \( \frac{x+3}{1-0.2x} \)$
					\begin{align*}
						\frac{x+3}{1-0.2x} &> 0
					\end{align*}
					Tato podmínka je splněna pouze pokud čitatel i jmenovatel jsou kladní (první odrážka) nebo oba záporní (druhá odrážka)
					
					\begin{itemize}
						
						\item čitatel i jmenovatel kladní
							\begin{align*}
								x+3 &> 0,
							&	1-0.2x &> 0,
							\\
								x &> -3,
							&	\frac 15 x &< 1,
							\\
								& 
							&	x &< 5,
							\\
								x &\in (-3, \infty),
							&	x &\in (-\infty, 5),
							\end{align*}
							\begin{align*}
								x &\in (-3, \infty) \cap (-\infty, 5),
							\\
								x &\in (-3, 5).
							\end{align*}
						
						\item čitatel i jmenovatel záporní
							\begin{align*}
								x+3 &< 0,
								&	1-0.2x &< 0,
								\\
								x &< -3,
								&	\frac 15 x &> 1,
								\\
								& 
								&	x &> 5,
								\\
								x &\in (-\infty, -3),
								&	x &\in (5, \infty),
							\end{align*}
							\begin{align*}
								x &\in (-\infty, -3) \cap (5, \infty),
								\\
								x &\in \emptyset.
							\end{align*}
						
					\end{itemize}
					
					Z jednoho řešení nám vyšel interval, z druhého prázdná množina. Finální řešení je ale sjednocení dílčích mezivýsledků (protože nám stačilo jedno z řešení, buď to nebo to), takže můžeme psát
					\begin{align*}
						x &\in (-3, 5) \cap \emptyset,
					\\
						\Aboxed{x &\in (-3, 5).}
					\end{align*}
				
			\end{enumerate}
		
		\item 42/2
			\begin{enumerate}
				\item $\log_2(64) = 6$,
				\item $\log_2(1) = 0$,
				\item $\log_4(4) = 1$,
				\item $\log_3 \( \frac 13 \) = -1$,
				\item $\log_{\frac 12}(2) = -1$,
				\item $\log_{\frac 12} \( \frac 18 \) = 3$,
				\item $\log_3 \( \frac{1}{27} \) = -3$,
				\item $\log_2 \( 2^{14} \) = 14$,
				\item $\ln(e) = 1$,
				\item $e^{\ln(e)} = e^1 = e$,
				\item $10^{\log(1)} = 10^0 = 1$,
				\item $\log(\log_2(2)) = \log(1) = 0$.
			\end{enumerate}
						
	\end{itemize}
	
	
	
	
	
\end{document}