\documentclass[a4paper,11pi]{article}
\usepackage[a4paper,hmargin=1in,vmargin=1in]{geometry}

\usepackage[czech]{babel}
\usepackage[utf8]{inputenc}
\usepackage[T1]{fontenc}

\usepackage{stddoc}

\usepackage{gensymb}

\begin{document}
	
	\section*{Jednotková kružnice}
		
		\subsection*{1.}
		
			Obrázky
			
		\subsection*{2.}
			
			Obrázky
			
		\subsection*{3.}
			
			Dá se jednoduše řešit podle toho, ve kterym kvadrantu ten úhel je:
			\begin{itemize}
				\item I. kvadrant, neboli $(0\degree,90\degree)$: cos kladný, sin kladný
				\item II. kvadrant, neboli $(90\degree,180\degree)$: cos záporný, sin kladný
				\item III. kvadrant, neboli $(180\degree,270\degree)$: cos záporný, sin záporný
				\item IV. kvadrant, neboli $(270\degree,360\degree)$: cos kladný, sin záporný
			\end{itemize}
			\begin{enumerate}[label=(\alph*)]
				\item $\cos(\alpha) < 0$, $\sin(\alpha) > 0$
				\item $\cos(\alpha) > 0$, $\sin(\alpha) < 0$
				\item $\cos(\alpha) > 0$, $\sin(\alpha) < 0$
				\item $\cos(\alpha) < 0$, $\sin(\alpha) < 0$
				\item $\cos(\alpha) > 0$, $\sin(\alpha) > 0$
			\end{enumerate}
		
		\subsection*{4.}
			
			\begin{enumerate}[label=(\alph*)]
				\item $x_1=40\degree$, $x_2=140\degree$
				\item $x_1=230\degree$, $x_2=310\degree$
				\item $x_1=280\degree$, $x_2=80\degree$
				\item $x_1=105\degree$, $x_2=255\degree$
			\end{enumerate}
		
		\subsection*{5.}
			
			\begin{enumerate}[label=(\alph*)]
				\item
				\begin{align*}
					&\cos\(\frac 56 \pi\) = -\cos\(\frac \pi 6\) = -\frac{\sqrt 3}{2}
				\\
					&\sin\(\frac 56 \pi\) = \sin\(\frac{\pi}{6}\) = \frac 12
				\end{align*}
				
				\item
				\begin{align*}
				&\cos\(\frac 76 \pi\) = -\cos\(\frac \pi 6\) = -\frac{\sqrt 3}{2}
				\\
				&\sin\(\frac 76 \pi\) = -\sin\(\frac{\pi}{6}\) = -\frac 12
				\end{align*}
				
				\item
				\begin{align*}
				&\cos\(\frac{11}6 \pi\) = \cos\(\frac \pi 6\) = \frac{\sqrt 3}{2}
				\\
				&\sin\(\frac{11}6 \pi\) = -\sin\(\frac{\pi}{6}\) = -\frac 12
				\end{align*}
				
				\item
				\begin{align*}
				&\cos\(\pi\) = -\cos(0) = -1
				\\
				&\sin\(\pi\) = \sin(0) = 0
				\end{align*}
				
				\item
				\begin{align*}
				&\cos\(\frac 34 \pi\) = -\cos\(\frac \pi 4\) = -\frac{\sqrt 2}{2}
				\\
				&\sin\(\frac 34 \pi\) = \sin\(\frac \pi 4\) = \frac{\sqrt 2}{2}
				\end{align*}
				
				\item
				\begin{align*}
				&\cos\(\frac 54 \pi\) = -\cos\(\frac \pi 4\) = -\frac{\sqrt 2}{2}
				\\
				&\sin\(\frac 54 \pi\) = -\sin\(\frac\pi 4\) = -\frac{\sqrt 2}{2}
				\end{align*}
				
				\item
				\begin{align*}
				&\cos\(\frac 74 \pi\) = \cos\(\frac \pi 4\) = \frac{\sqrt 2}{2}
				\\
				&\sin\(\frac 74 \pi\) = -\sin\(\frac\pi 4\) = -\frac{\sqrt 2}{2}
				\end{align*}
				
				\item
				\begin{align*}
				&\cos\(\frac 32 \pi\) = \cos\(\frac \pi 2\) = 0
				\\
				&\sin\(\frac 32 \pi\) = -\sin\(\frac \pi 2\) = -1
				\end{align*}
				
				\item
				\begin{align*}
				&\cos\(\frac 12 \pi\) = \cos\(\frac \pi 2\) = 0
				\\
				&\sin\(\frac 12 \pi\) = \sin\(\frac \pi 2\) = 1
				\end{align*}
				
				\item
				\begin{align*}
				&\cos\(\frac 23 \pi\) = -\cos\(\frac \pi 3\) = -\frac 12
				\\
				&\sin\(\frac 23 \pi\) = \sin\(\frac \pi 3\) = \frac{\sqrt 3}2
				\end{align*}
				
				\item
				\begin{align*}
				&\cos\(\frac 43 \pi\) = -\cos\(\frac \pi 3\) = -\frac 12
				\\
				&\sin\(\frac 43 \pi\) = -\sin\(\frac\pi 3\) = -\frac{\sqrt 3}2
				\end{align*}
				
				\item
				\begin{align*}
				&\cos\(\frac 53 \pi\) = \cos\(\frac \pi 3\) = \frac 12
				\\
				&\sin\(\frac 53 \pi\) = -\sin\(\frac{\pi}{3}\) = -\frac{\sqrt 3}2
				\end{align*}
				
				\item
				\begin{align*}
				&\cos\(945\degree\) = \cos\(225\) = \cos\({\frac 54 \pi}\) = -\frac{\sqrt 2}{2}
				\\
				&\sin\(945\degree\) = \sin\(225\) = \sin\(\frac 54 \pi\) = -\frac{\sqrt 2}{2}
				\end{align*}
			\end{enumerate}
	
	
\end{document}