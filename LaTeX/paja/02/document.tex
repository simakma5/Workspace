\documentclass[11pt,a4paper]{article}

\usepackage[czech]{babel}
\usepackage[utf8]{inputenc}
\usepackage[T1]{fontenc}

\usepackage{array}
\usepackage{mathtools}
\usepackage{stddoc}
	
\newcommand{\comb}[2]{
	\left( \begin{matrix}
		#1 \\
		#2
	\end{matrix} \right)
}	
	
	
	
\begin{document}	

	\pagenumbering{gobble}
	
	\begin{enumerate}
				
			\item
				\begin{align*}
					\comb{x+3}{x} - \comb{x+1}{x-1} &= 3 \comb 66,
				\\
					\frac{(x+3)!}{3! \cdot x!} - \frac{(x+1)!}{2! \cdot (x-1)!} &= 3,
				\\
					\frac{(x+3)(x+2)(x+1)}{6} - \frac{(x+1)x}{2} &= 3,
				\\
					(x+3)(x+2)(x+1) - 3x(x+1) &= 18,
				\\
					x^3 + 3x^2 + 8x -12 &= 0,
				\\
					\left(x-1\right)\left(x^2+4x+12\right)&=0.
				\end{align*}
				Nevím, proč vám tohle zadává. Nevidím jiný řešení, než že jeden kořen ($x=1$) uhádneš a tím pádem ten polynom můžeš napsat v podobě $P(x) = (x-1) P'(x)$, kde $P(x)$ je původní polynom $P(x) = ^2+4x+12$ a $P'(x)$ je polynom, který vznikne jako výsledek dělení polynomů
				\begin{align*}
					P'(x) = P(x) : (x-1) = (x^2+4x+12):(x-1) = x^2 + 4x + 12,
				\end{align*}
				což je kvadratická rovnice se záporným diskriminantem, tudíž jediný řešení v množině reálných číšel je $x=1$, který jsi uhádla na začátku.
					
			\item
				\begin{align*}
					\comb{x+3}{1}^3 + 6 \comb{x+1}{2} - 6 \comb{x}{3} &= 9x^2 + 25,
				\\
					(x+3)^3 + 6 \frac{(x+1)!}{(x-1)! \cdot 2!} - 6 \frac{x!}{(x-3)! \cdot 3!} &= 9x^2 + 25,
				\\
					(x+3)^3 + 3(x+1)x - x(x-1)(x-2) &= 9x^2 + 25,
				\\
					6x^2 + 28x + 2 &= 0.
				\end{align*}
				Tady se asi zase přepsala, protože z tohohle ani výpočetní technika nezíská ten výsledek, kterej získala paní učitelka xD.
					
			\item
				\begin{align*}
					\comb{x+1}{2} + \comb{x+4}{2} + \comb{x+7}{x+5} &< 9x^2 + 25,
				\\
					\frac{(x+1)!}{(x-1)! \cdot 2!} + \frac{(x+4)!}{(x+2)! \cdot 2!} + \frac{(x+7)!}{2! \cdot (x+5)!} &< 9x^2 + 25,
				\\
					(x+1)x + (x+4)(x+3) + (x+7)(x+6) &< 18x^2 + 50,
				\end{align*}
				\begin{align*}
					x &\in \(-\infty; -\frac{\sqrt{681}}{30}+\frac{7}{10}\) \cup \(\frac{\sqrt{681}}{30}+\frac{7}{10}; +\infty\).
				\end{align*}
				To taky neni uplně ono, ale jako z mojí hlavy to neni, jen počáteční úpravy. Rovnice jsem nechal řešit Symbolab.
				
		\end{enumerate}
	
\end{document}