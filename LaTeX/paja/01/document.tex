\documentclass[11pt,a4paper]{report}
\setlength\textwidth{145mm}
\setlength\textheight{247mm}
\setlength\oddsidemargin{0mm}
\setlength\evensidemargin{0mm}
\setlength\topmargin{0mm}
\setlength\headsep{0mm}
\setlength\headheight{0mm}
\usepackage{yfonts}

\usepackage[czech]{babel}
\usepackage[utf8]{inputenc}
\usepackage[T1]{fontenc}

\usepackage{array}
\usepackage{mathtools}
\usepackage{stddoc}


\begin{document}
	
	\section*{Příklad}
		
		\paragraph*{Zadání} Pro okamžité hodnoty napětí a proudu v obvodu střídavého proudu platí rovnice
			\begin{align}
				u(t) &= 325 \sin(100 \pi t) \; \mathrm V,
			\\
				i(t) &= 0.7 \sin \( 100 \pi t + \frac \pi 3 \) \mathrm A.
			\end{align}
			Určete
			\begin{enumerate}
				\item efektivní hodnoty napětí a proudu,
				\item frekvenci střídavého proudu,
				\item činný výkon střídavého proudu.
			\end{enumerate}
	
		\paragraph*{Řešení}
			\begin{enumerate}
				\item 
				\begin{align}
					u(t) &= 325 \sin(100 \pi t) \; \mathrm V
				\implies
					\mathrm{U_m} = 325 \; \mathrm V
				\implies
					\Aboxed{\mathrm{U_{ef}} = \frac{\mathrm{U_m}}{\sqrt{2}} \approx 230 \; \mathrm V.}
				\\
					i(t) &= 0.7 \sin \( 100 \pi t + \frac \pi 3 \) \mathrm A
				\implies
					\mathrm{I_m} = 0.7 \; \mathrm A
				\implies
					\Aboxed{\mathrm{I_{ef}} = \frac{\mathrm{I_m}}{\sqrt{2}} \approx 0.5 \; \mathrm A.}
				\end{align}
				
				\item Jelikož víme, že rovnice střídavého napětí je ve tvaru $u(t) = \mathrm{U_m} \sin(2 \pi f + \varphi_u)$ (případně rovnice střídavého proudu $i(t) = \mathrm{I_m} \sin(2 \pi f + \varphi_i)$), můžeme od oka určit frekvenci $f$ ze zadání jako $f = 50$ Hz.
				
				\item Víme, že činný výkon spočítáme jako
				\begin{align}
					P = \mathrm{U_{ef}} \cdot \mathrm{I_{ef}} \cos(\varphi),
				\end{align}
				kde $\varphi = \varphi_i - \varphi_u$ a hodnoty $\mathrm{U_{ef}}$ a $\mathrm{I_{ef}}$ jsme již vypočítali, takže můžeme psát
				\begin{align}
					P &= \mathrm{U_{ef}} \cdot \mathrm{I_{ef}} \cos(\varphi_i - \varphi_u),
				\\
					P &= 230 \cdot 0.5 \cdot \cos\( \frac \pi 3 \) \mathrm W,
				\\
					P &= 230 \cdot 0.5 \cdot 0.5 \; \mathrm W,
				\\
					\Aboxed{P &= 57.5 \; \mathrm W}.
				\end{align}
			\end{enumerate}
	
	
	
	
\end{document}