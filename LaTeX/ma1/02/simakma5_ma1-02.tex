\documentclass[a4paper,11pt]{article}
\usepackage[a4paper,hmargin=1in,vmargin=1in]{geometry}

\usepackage[czech]{babel}
\usepackage[utf8]{inputenc}
\usepackage[T1]{fontenc}
\usepackage{fancyhdr}

\pagestyle{fancy}
\fancyhf{}
\lhead{Matematická analýza 1}
\rhead{A8B01MC1}
\rfoot{Martin Šimák}

\usepackage{stddoc}


\begin{document}
	
	\pagenumbering{gobble}
	\section{Monotonie a extrémy}
	
	\paragraph*{Zadání.} Vyšetřete monotonii a lokální extrémy funkce
	\begin{align*}
		f(x) = \arctan(x|x-2|).
	\end{align*}
	
	\paragraph*{Řešení.} Nejprve rozdělme naši práci dle linearity funkce $|x-2|$ na svém definičním oboru, tj.
	\begin{align*}
		f(x) = \begin{cases}
			\arctan(2x-x^2), & x \in (-\infty,2];
		\\
			\arctan(x^2-2x), & x \in (2,\infty).
		\end{cases}
	\end{align*}
	Dále stanovme derivaci
	\begin{align*}
		f'(x) = \begin{cases}
			\dfrac{2-2x}{1+(2x-x^2)^2}, & x \in (-\infty,2];
		\\[3mm]
			\dfrac{2x-2}{1+(x^2-2x)^2}, & x \in (2,\infty).
		\end{cases}
	\end{align*}
	Lomená funkce $f'$ ovšem může nabývat nulové hodnoty pouze v nulových bodech polynomu v činiteli. Naše úloha hledání lokálních extrému se tedy zjednodušuje na rovnice $2-2x$ a $2x-2$, obě s řešením $x_1=1$. Máme tedy dva body podezřelé z extrému, a to $x_1 = 1$ a $x_2 = 2$.
	
	Jednoduchou úvahou o určení monotonie diferencovatelné funkce na intervalu pomocí znaménka první derivace lze poukázat na fakt, že
	\begin{align*}
		f'(0) &> 0,
	&
		f'(1,5) &< 0,
	&
		f'(3) &> 0.
	\end{align*}
	Opravdu tedy body $x_1, \, x_2$ rozdělují definiční obor funkce $f$ na intervaly monotonie $(-\infty, 1)$, $(1,2)$, $(2,\infty)$, přičemž funkce $f$ je rostoucí na $(-\infty, 1)$ a na $(2,\infty)$, klesající na $(1,2)$. Zároveň pak $x_1 = 1$ je bod lokálního maxima a $x_2 = 2$ lokálního minima.
	
	\paragraph*{Zadání.} Určete maximum a minimum funkce $f$ na intervalu $I$:
	\begin{align*}
		f(x) &= \ln(5-4x-x^2),
	&
		I &= [-3,0].
	\end{align*}
	
	\paragraph*{Řešení.} Nejprve určeme derivaci
	\begin{align*}
		f'(x) = \frac{2x+4}{x^2+4x-5}.
	\end{align*}
	Tato funkce nabývá nulové hodnoty pouze v bodě $x_0 = -2 \in I$. Stejnou úvahou jako v první úloze zjistěme, že
	\begin{align*}
		f'(-3) &> 0,
	&
		f'(-1) &< 0.
	\end{align*}
	Funkce je tedy rostoucí na $[-3,-2)$, klesající na $(-2,0]$ a v bodě $x_0 = -2$ má lokální maximum. Bod $x_0$ je tedy zároveň maximem na intervalu $I$ a minimem je jeden z krajních bodů intervalu. Jelikož platí
	\begin{align*}
		f(-3) \approx 2,079 > 1,609 \approx f(0),
	\end{align*}
	minimem na intervalu $I$ je bod $f(0)$.
	
	
\end{document}