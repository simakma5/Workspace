\documentclass[a4paper,11pt]{article}
\usepackage[a4paper,hmargin=1in,vmargin=1in]{geometry}

\usepackage{babel}
\usepackage[utf8]{inputenc}
\usepackage[T1]{fontenc}

\usepackage{stddoc}
\usetikzlibrary{intersections,decorations.pathreplacing}
\usepackage{lipsum}

\title{Introduction to Topology}
\author{Mini-lecture 1}
\date{}


\makeatletter
\def\thmheadbrackets#1#2#3{%
	\thmname{#1}\thmnumber{\@ifnotempty{#1}{ }\@upn{#2}}%
	\thmnote{{\;\;\the\thm@notefont[#3]}}}
\makeatother

\newtheoremstyle{theorem}		% name
{\topsep}						% Space above
{\topsep}						% Space below
{\normalfont}					% Body font
{}								% Indent amount
{\bfseries}						% Theorem head font
{.}								% Punctuation after theorem head
{.5em}							% Space after theorem head
{\thmheadbrackets{#1}{#2}{#3}}	% theorem head spec

\theoremstyle{theorem}
\newtheorem{theorem}{Theorem}[section]
% The additional parameter [section] restarts the theorem counter at every new section.
\newtheorem{corollary}{Corollary}[theorem]
% An environment called corollary is created, the counter of this new environment will be reset every time a new theorem environment is used.
\newtheorem{lemma}[theorem]{Lemma}
% In this case, the even though a new environment called lemma is created, it will use the same counter as the theorem environment.
\theoremstyle{remark}
\newtheorem*{remark}{Remark}
\newtheorem*{convention}{Convention}
% The syntax of the command \newtheorem* is the same as the non-starred version, except for the counter parameters. In this example a new unnumbered environment called remark is created.

\theoremstyle{definition}
\newtheorem{definition}{Definition}
\newtheorem*{observation*}{Observation}
\newtheorem{observation}{Observation}[theorem]
\newtheorem*{example}{Example}
\newtheorem*{revision}{Revision}

%\renewcommand\qedsymbol{$\blacksquare$}


\newcommand{\Span}{{\mathrm{Span}\,}}
\newcommand{\Ker}{{\mathrm{Ker}\,}}
%\newcommand{\Im}{{\mathrm{Im}\,}}
\newcommand{\Hom}{{\mathrm{Hom}}}
\newcommand{\Id}{{\mathrm{Id}}}


\begin{document}
	
	\pagenumbering{roman}
	\maketitle
	
	
	\section{Topological spaces}
	
		\begin{definition}
			Let $X$ be a set and $\tau$ a system of subsets $\mathcal O \subset X$ satisfying
			\begin{enumerate}[label=(\alph*)]
				\item $\emptyset \in \tau$, $X \in \tau$;
				\item the union of an arbitrary number of elements from $\tau$ is again an element of $\tau$;
				\item the intersection of a finite number of element from $\tau$ is again an element of $\tau$.
			\end{enumerate}
			Then we call the elements of $\tau$ \emph{open sets} and the pair $(X, \tau)$ a \emph{topological space}.
		\end{definition}
	
		\begin{remark}
		Two special topologies in regard to its fineness:
		\begin{itemize}
			\item \emph{trivial topology,} where the only sets contained in $\tau$ are $\emptyset$ and $X$;
			\item \emph{discrete topology} contains all possible subsets of $X$ meaning every subset of $X$ is open.
		\end{itemize}
		We can say that all other topologies reside "somewhere between" those two extreme instances.
		\end{remark}
		
		\begin{definition}
			Let $(X, \tau)$, and $(Y, \{\sigma\})$ be topological spaces. A mapping $f:X \to Y$ is called \emph{continuous} if $\forall A \in \{\sigma\} \,:\, f^{-1}(A) \in \tau$.%
				\footnote{Let is be know that by $f^{-1}$ we don't mean an inverse mapping (might not even exist). Instead, $f^{-1}(A)$ denotes the collection of all elements in $X$ which get mapped into $A$ by $f$, i.e. the \emph{inverse image} of the set $A$.}
			Moreover, if $f$ happens to be bijective and $f^{-1}$ is continuous as well, $f$ is called a \emph{homeomorphism}. Also we say that $X$ and $Y$ are then homeomorphic.
		\end{definition}
		
		\begin{remark}
			We can easily show that if $f: X \to Y$ and $g: Y \to Z$ are continuous, then the composition $g \circ f: X \to Z$ is continuous, too.
		\end{remark}
	
		\begin{remark}
			As an interesting fact we can observe that the relation "being homeomorphic" introduces a natural equivalence among topological spaces.%
				\footnote{That is, of course, given by the relation's reflexivity, symmetry and transitivity. This equivalence can be visualized via imagining topological spaces as if made of rubber (famous cup-doughnut transformation).}
		\end{remark}
		
		\begin{definition}
			A topological space $(X,\tau)$ is said to be \emph{Hausdorff} (also separated or $T_2$) if for any two point $x, \, y \in X$, there exist open non-intersecting neighborhoods of them, i.e. open sets $A, \, B \subset X$ for which $A \cap B = \emptyset$.
		\end{definition}
		Intuitively, this means that one can \emph{separate} any two points in the topological space by means of open sets. This is also the reason why Hausdorff topological spaces are often referred to as \emph{separated} or $T_2$ (this is because it fulfills a special separation axiom by which we sort separated spaces). From now on, we shall assume this particular property every time we speak about topological spaces.
		
		\begin{example}
			Cartesian space $\mathbb R^n$ naturally represents a topological space, where open sets coincide with those used already in elementary calculus of $n$ real variables. This is caused by the fact that Cartesian spaces are (by definition) equipped with the standard Euclidean distance $d(x,y)$ between any two points $x,y \in \mathbb R^n$, i.e.
			\begin{align*}
				d^2(x,y) \coloneqq (x^1-y^1)^2 + \cdots + (x^n-y^n)^2.
			\end{align*}
			Now let us construct and \emph{open ball} $D(a,r)$, centered at $a$, with radius $r$, defined as
			\begin{align*}
				D(a,r) \coloneqq \{x \in \mathbb R^n \mid d(x,a) < r\}.
			\end{align*}
			Now we can just define that a set $A \in \mathbb R^n$ is open if for any point $x \in A$ exists an open ball centered at $x$ which lies entirely in $A$. This topology over $\mathbb R^n$ is called the \emph{standard topology}.
		\end{example}
		
		\begin{definition}
			Let $A$ be an open set in $\mathbb R^n[x]$ and $f: A \to \mathbb R^m[y]$.%
				\footnote{The $y$ and $x$ simply mean that we are given $m$ functions of $n$ variables in the form of $y^a = y^a(x^1,\dots,x^n)$, where $a \in \{1,\dots,m\}$.}
			If all partial derivatives of $f$ up to order $k$ exist and are continuous, then $f$ is called a function of \emph{class $C^k$}. Specifically, it is called \emph{continuous} for $k=0$, \emph{smooth} for $k=\infty$, \emph{analytic} if for all $x \in A$ the Taylor series of $y^a(x)$ converges to the function $y^a(x)$ itself: $k=\omega$.
		\end{definition}
	
	\section{Charts, atlases, and manifolds}
	
		\begin{definition}
			Let $(X,\tau)$ be a topological space and $A \subset X$ an open set. A homeomorphism $\varphi: A \to \mathbb R^n$ is called a \emph{chart} or \emph{local coordinates}. The set $A$ is also called a \emph{coordinate patch} in this context.
		\end{definition}
		
		\begin{remark}
			It is of course impossible for all topological spaces to have just one chart covering all points.%
				\footnote{Naturally, when dealing with "flat spaces", e.g. $\mathbb R^n$, all we need is just one chart. In the given case, the chart would be an identity mapping $\mathrm{id}: \mathbb R^n \to \mathbb R^n$. Other charts - more complex that just an identity - could be constructed too. Some of those "more interesting" options correspond to the curvilinear coordinates in $\mathbb R^n$.}
			To achieve that, we will need a collection of charts covering several coordinate patches on $X$, just like we need several maps to cover the surface of the Earth in an atlas.
		\end{remark}
		
		\begin{definition}
			Let $(X,\tau)$ be a topological space and $\{A_i\}$ its \emph{open covering}, i.e. it holds that $X = \bigcup_{i} A_i$. Further, let $\varphi_i: A_i \to \mathbb R^n$ be a system of charts for each $i \in I$. The collection of charts $\mathcal A \equiv (A_i,\varphi_i)$ is called an \emph{atlas} on $X$.
		\end{definition}
	
		\begin{remark}
			If an arbitrary intersection $A_i \cap A_j$ happens to be non-empty, a mapping
			\begin{align*}
				\varphi_j \circ \varphi_i^{-1}: \varphi_i(A_i \cap A_j) \to \mathbb R^n
			\end{align*}
			called a \emph{change of coordinates} is naturally induced. Since this is a mapping of Cartesian spaces, it makes sense to talk about its class of smoothness. By definition, it must be at least $C^0$. There is, however, no reason for them not to be of even a higher class of smoothness. If, given an atlas, all mappings of this type happen to be $C^k$ or higher, we call it a \emph{$C^k$-atlas}.
		\end{remark}
		
		\begin{remark}
			An atlas may be expanded by adding additional charts, such as $\phi: A \to \mathbb R^n$. Such mapping is said to be \emph{$C^k$-related} if consistent with all charts $\varphi_i$ on the intersection $A \cap A_i$, i.e. if the class of the mapping $\varphi_i \circ \phi^{-1}$ is $C^k$ of higher. Such mapping may be added to $\mathcal A$. In the same manner, a $C^k$-atlas can be consecutively supplemented by all possible charts and we are left with a unique \emph{maximal $C^k$-atlas} $\hat{\mathcal A}$. This will be an important construction for our further work, as we can see already in the next definition.
		\end{remark}
	
		\begin{definition}
			Let $(X,\tau)$ be a topological space and $\hat{\mathcal A}$ it's maximal atlas. A pair $(X,\hat{\mathcal A})$ is called an $n$-dimensional \emph{$C^k$-manifold}.%
				\footnote{A similar to the already know terminology of smoothness is also present (in particular topological, differentiable, \dots, smooth, analytic). In our interest, we will be working exclusively with smooth manifolds.}
		\end{definition}
	
		\begin{example}
			On a circle $S^1$ of radius $R$ we introduce local coordinates $x, \, x'$ via the so-called \emph{stereographic projection} as shown in the figure.
			
			\begin{minipage}{0.45\textwidth}
	%			\begin{figure*}[!htb]
					\begin{tikzpicture}[scale=1.5]
						% circle
						\draw [dashed] (0,-1) node[anchor=north east]{$S$} -- (0,1) node[anchor=south east]{$N$};
						\draw (0,0) circle (1cm);
						
						% projection lines
						\draw [->] (-1,1) -- (3,1) node[anchor=west]{$\mathbb R$};
						\draw [->] (-1,-1) -- (3,-1) node[anchor=west]{$\mathbb R$};
						
						% x and x' coordinates
						\draw [name path=line 1] (0,-1) -- (1.55,1);
						\draw [name path=line 2] (0,1) -- (2.6,-1);
						\fill[gray, name intersections={of=line 1 and line 2,total=\t}]
						\foreach \s in {1,...,\t}{(intersection-\s) circle (1pt) node[anchor=west,black]{\footnotesize $P$}};
						\draw[decorate,decoration={brace,raise=0.5ex}] (0,1) -- (1.55,1) node[midway,above=1ex]{$x$};
						\draw[decorate,decoration={brace,mirror,raise=0.5ex}] (0,-1) -- (2.6,-1) node[midway,below=1ex]{$x'$};
					\end{tikzpicture}
	%			\end{figure*}
			\end{minipage}
			~
			\begin{minipage}{0.45\textwidth}
				A circle is indeed a manifold - in our case equipped with these two charts forming an analytic atlas. On the intersection of the patches, where primed and unprimed meet, we can write down a following transition relation:
				\begin{align*}
					x' = \frac{(2R)^2}{x}.
				\end{align*}
			\end{minipage}\\
			In higher dimensions, we obtain an $n$-dimensional sphere, which is in turn an $n$-dimensional manifold. The transition relation stays the same with the only difference being replacing $x$, resp. $x'$, with $\vec r$, resp. $\vec r'$. This transformation is justified by projecting an $n$-dimensional sphere with tangent $\mathbb R^n$ planes "glued" to the poles to a 2-dimensional plane, where the problem boils down to the $S^1$ problem.
		\end{example}
		
		\begin{example}
			Let $(X,\hat{\mathcal A})$ and $(Y,\hat{\mathcal B})$ be smooth manifolds and $\varphi_i : A_i \to \mathbb R^n$, $\psi_i: B_i \to \mathbb R^n$ be two charts on $X$ and $Y$ respectively. Then the mapping
			\begin{align*}
				\varphi_i \times \psi_i : A_i \times B_i &\to \mathbb R^{n+m},
			\\
				(x,y) &\mapsto (\varphi_i(x), \psi_i(y)),
			\end{align*}
			where $x \in A_i$ and $y \in B_i$, introduces a smooth atlas on $X \times Y$. This way we can construct a smooth manifold of dimension $n+m$.
			
			Using this further insight, we can expand our practical intuition of manifolds. Thus, we can consider e.g. a plane double pendulum as $S^1 \times S^1$ or a wheel of a car as $\mathbb R^2 \times S^1 \times S^1$ to be manifolds.
		\end{example}

	\section{Mapping manifolds}
	
		First, let us endorse the circumstance of how important the perk of smoothness is to us. To withhold such virtue, when mapping between two manifolds, we will simply demand that the smoothness is not violated. Such chosen mapping are called \emph{smooth}, specifically...
		\begin{definition}
			Let $(M, \hat{\mathcal A})$ and $(N, \hat{\mathcal B})$ be smooth manifolds of dimensions $m$ and $n$, respectively, $f: M \to N$ be a mapping, and
			\begin{align*}
				\varphi: A &\to \mathbb R^m, \qquad A \subset M,
			\\
				\psi: B &\to \mathbb R^n, \qquad\, B \subset N,
			\end{align*}
			be local coordinates such that $f(A) \subset B$. Then the composition map
			\begin{align*}
				\hat f \coloneqq \psi \circ f \circ \varphi^{-1}: \mathbb R^m \to \mathbb R^n
			\end{align*}
			is induced. We call this mapping the \emph{coordinate presentation} of the mapping $f$.
		\end{definition}
	
		\begin{remark}
			Once again, we're dealing with a mapping between two Cartesian spaces. Therefore it makes sense to talk about its class of smoothness. By definition, $f$ itself is called \emph{smooth} (generally $C^k$) if the coordinate presentation $\hat f$ with respect to any pair of charts from $\hat{\mathcal A}$ and $\hat{\mathcal B}$ happens to be smooth (generally $C^k$).
		\end{remark}
	
		\begin{definition}
			Let $(M, \hat{\mathcal A})$ and $(N, \hat{\mathcal B})$ be smooth manifolds with dimensions $\dim M = \dim N = n$. Further let $f: M \to N$ be a bijection and both $f$ and $f^{-1}$ be smooth. Then the mapping $f$ is called a \emph{diffeomorphism}. Also, $M$ and $N$ are said to be \emph{diffeomorphic manifolds}.
		\end{definition}
	
		\begin{remark}
			Note that the relation of two manifolds being diffeomorphic once again introduces an equivalence relation. Further, all diffeomorphisms $M \to M$ form a group denoted as $\mathrm{Diff}(M)$.
		\end{remark}
		
		\begin{example}
			Let $T^2$ be a 2-dimensional torus and $Q$ a square. Then it holds that
			\begin{align*}
				S^1 \times \mathbb R &\approx \text{"surface of a cylinder"},
			&
				\mathbb R^n \times \mathbb R^m &\approx \mathbb R^{n+m},
			\\
				S^1 \times S^1 &\approx T^2,
			&
				Q &\approx S^1.
			\end{align*}
		\end{example}
		
		\begin{definition}
			Let $(M, \hat{\mathcal A})$ and $(N, \hat{\mathcal B})$ be smooth manifolds, $\dim M = m \leq n = \dim N$, $f: M \to N$ be a smooth mapping, $x \in M$ and $y \in N$. This mapping is said to be an \emph{immersion} if in some neighborhood $B$ of each point $y \in f(M) \subset N$ there exist local coordinates $y^1, \dots, y^n$ such that for a sufficiently small neighborhood $A$ of $x$ ($f(A) \subset B$) it holds that the subset $f(A)$ is given by the system of equations
			\begin{align*}
				y^{m+1} = \cdots = y^n = 0.
			\end{align*}
		\end{definition}
		
		\begin{remark}
			We can interpret the preceding definition as that the image of this immersion may be locally expressed in terms of vanishing of some part of the coordinates on $N$. It is possible that this behavior corresponds to the existence of some sort of edges and cusps. For testing whether a mapping is an immersion, the formal definition might not be very suitable. Without a proof, we can introduce a much better property of immersions:
			\begin{align*}
				&\text{$f: M \to N$ is an immersion}
			&
				&\iff
			&
				&\text{rank of $J_i^a \equiv \dfrac{\p y^a}{\p x^i}$ is maximal ($=m$) on $f(M)$.}
			\end{align*}
		\end{remark}
		
		\begin{definition}
			Injective immersion is called an \emph{embedding}.
		\end{definition}
	
		\begin{remark}
			From what we know about immersions from earlier, embedding provides no self-intersections in $f(M)$.
		\end{remark}
		
		\begin{definition}
			Let $(M, \hat{\mathcal A})$ and $(N, \hat{\mathcal B})$ be smooth manifolds, $f: M \to N$ an embedding. The subset $f(M) \subset N$ is naturally endowed with the structure of a manifold and is called a \emph{submanifold} of a manifold $M$.
		\end{definition}
	
	
	
\end{document}