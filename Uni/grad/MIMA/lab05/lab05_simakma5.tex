\documentclass[11pt,a4paper]{article}
\usepackage[a4paper,hmargin=1in,vmargin=1in]{geometry}
\usepackage{pgfplots}
\pgfplotsset{compat=1.17}

\usepackage[czech]{babel}
\usepackage[utf8]{inputenc}
\usepackage[T1]{fontenc}

\usepackage{stddoc}
\usepackage{lipsum}
\usepackage{subcaption}

\newcommand{\plus}{{\texttt{+}}}
\renewcommand{\Re}{\operatorname{Re}}
\renewcommand{\Im}{\operatorname{Im}}
\newcommand{\fourier}[3]{\mathcal{F}_{#1}\!\left[#2\right]\!\left(#3\right)}
\newcommand{\ifourier}[3]{\mathcal{F}^{-1}_{#1}\!\left[#2\right]\!\left(#3\right)}


\begin{document}

\pagenumbering{arabic}

% Header
\begin{center}
    {\LARGE\textbf{Laboratorní úloha č. 5}}\\[3mm]
    \begin{minipage}{0.4\textwidth}
        \begin{flushleft}
            \textsc{\today}
        \end{flushleft}
    \end{minipage}
    ~
    \begin{minipage}{0.4\textwidth}
        \begin{flushright}
            \textsc{Martin Šimák}
        \end{flushright}
    \end{minipage}
    \noindent\rule{14.5cm}{0.4pt}
\end{center}

\paragraph*{Měření šumového čísla} Laboratorní úloha poskytuje seznámení se základními metodami měření šumového čísla pasivních i aktivních komponent. Úloha obsahuje měření jak zapomoci specializovaného měřiče, tak i pomocí spektrálního analyzátoru.

\subsection*{Úkoly měření}
\begin{enumerate}
    \item Měření šumového čísla spektrálního analyzátoru.
    \item Měření šumového čísla pomocí Y-metody
    \item Měření šumového čísla pomocí HP 8970A
\end{enumerate}

\subsection*{Použité přístroje a komponenty}
\begin{itemize}
    \item Spektrální analyzátor R\&S FSP30 (10~MHz až 1,5~GHz)
    \item Měřič šumového čísla HP 8970A (10~MHz až 1,5~GHz)
    \item Šumivka HP 346B (do 18~GHz)
    \item Zesilovač Mini-Circuits GALI-2\plus
    \item Stejnosměrný napájecí zdroj Gwinstek
    \item Propojovací BNC kabely
\end{itemize}

\subsection*{Měřené komponenty}
\begin{itemize}
    \item Nízkošumový zesilovač s tranzistorem BFP 840ESD
    \item Atenuátor Mini-Circuits s útlumem 8~dB
\end{itemize}

\subsection*{Popis měření}
\lipsum[1]

% Task 1
\paragraph*{Úloha 1} \lipsum[5]

% \begin{table}[!ht]
% \begin{center}
% \begin{tabular}{| l || c | c | c | c | c | c | c | c | c |}
%     \hline
%     $f$ [GHz] & 8 & 8,5 & 9 & 9,5 & 10 & 10,5 & 11 & 11,5 & 12 \\
%     \hline
%     $P_G$ [dBm] & 0 & -0,15 & -0,58 & -0,8 & -0,43 & -0,15 & -0,07 & 0,06 & -0,2 \\
%     \hline
% \end{tabular}
% \caption{Hodnoty výkonu generátoru HP 86250D na různých frekvencích}
% \label{table:stability-HP}
% \end{center}
% \end{table}

% Task 2
\paragraph*{Úloha 2} \lipsum[3]

% \begin{table}[!ht]
% \begin{center}
% \begin{tabular}{| l || c | c | c | c | c |}
%     \hline
%     $f_{\mathrm{off}}$ [kHz] & 1 & 10 & 100 & 1000 & 10000 \\
%     \hline
%     $\mathcal L_{\mathrm{HP}}$ [dBc/Hz] & -34,5 & -92,93 & -116,18 & -139,4 & -149,18 \\
%     \hline
%     $\mathcal L_{\mathrm{R\&S}}$ [dBc/Hz] & -103,4 & -111,83 & -117,06 & -131,64 & -143,58 \\
%     \hline
% \end{tabular}
% \caption{Hodnoty fázového šumu generátorů HP 86250D a R\&S SMF 100A}
% \label{table:phase-noise-HP-and-RS}
% \end{center}
% \end{table}

\subparagraph*{Úkol} Zadání
\lipsum[1]

% \begin{figure}[!ht]
% \begin{center}
%     \includegraphics[width=0.8\textwidth]{src/}
% \end{center}
% \caption{Blokové schéma zapojení generátoru ELSY SG2000}
% \label{fig:elsy-sg2000-connection-scheme}
% \end{figure}

% Task 3
\paragraph*{Úloha 3} \lipsum[5]

% \begin{figure}[!ht]
%     \centering
% \begin{subfigure}{0.45\textwidth}
%     \centering
%     \includegraphics[width=\textwidth]{src/
%     \caption{ELSY SG2000}
%     \label{fig:elsy-sg2000-intermodulation}
% \end{subfigure}
% \begin{subfigure}{0.45\textwidth}
%     \centering
%     \includegraphics[width=\textwidth]{src/}
%     \caption{R\&S SMF 100A}
%     \label{fig:rs-smf-100a-intermodulation}
% \end{subfigure}
% \caption{Grafické zpracování dat pro ilustraci zahrazení IM2 a IM3}
% \end{figure}

\subparagraph*{Úkol} Zadání

\lipsum[5]

\subsection*{Závěr}
\lipsum[5]


\end{document}
