\documentclass[11pt,a4paper]{article}
\usepackage[a4paper,hmargin=1in,vmargin=1in]{geometry}
\usepackage{pgfplots}
\pgfplotsset{compat=1.17}

\usepackage[czech]{babel}
\usepackage[utf8]{inputenc}
\usepackage[T1]{fontenc}

\usepackage{stddoc}
\usepackage{lipsum}
\usepackage{subcaption}
\usepackage[normalem]{ulem}

\newcommand{\plus}{{\texttt{+}}}
\renewcommand{\ohm}{{\mathrm{\Omega}}}
\newcommand{\kiloohm}{{\mathrm{k\Omega}}}


\begin{document}

\pagenumbering{arabic}

% Header
\noindent\LARGE\textbf{B2M37RNVA -- Otázky ke zkoušce}\normalsize\\
\noindent\rule{12.5cm}{0.4pt}

\begin{enumerate}
    \item \textbf{Základní metody rádiové navigace AoA, SS, ToA, TDoA, DS}\\
    \emph{Angle of Arival (AoA)}
    \begin{itemize}
        \item Směrové antény (Radiokompas DF, Radar): Triangulace -- určí se kurs ke dvěma všesměrových majákům NDB a poloha se určí jako průsečík příslušných radiál v navigační mapě.
        \item Dopplerovský směrový zaměřovač: Rotující anténa/spínané antény na kružnici -- Rotační pohyb způsobí, že v důsledku Dopplerova jevu bude přijímaný signál frekvenčně modulovaný harmonickým signálem s periodou otáčení antén.
    \end{itemize}

    \emph{Signal Strength (SS)} měří vzdálenost od majáku na základě úrovně přijímaného signálu. Uživatel se nachází na kružnici resp. na povrchu koule o poloměru $R$. Při použití více majáků lze určit polohu jako průsečík příslušných kruhů resp. koulí. Aplikace v měření vzdálenosti k prahu vzletové a přistávací dráhy u systému ILS (nepoužívá se), pokusy o indoor navigaci.

    \emph{Time of Arrival (ToA)} měří vzdálenost od majáku na základě doby šíření signálu. Uživatel se nachází na kružnici resp. na povrchu koule o poloměru $R$. Při použití více majáků lze určit polohu jako průsečík příslušných kruhů resp. koulí.
    \begin{itemize}
        \item Aktivní systémy: radar (primárná i sekundární), výškoměr, dálkoměr
        \item Pasivní systémy: družicové navigační systémy
    \end{itemize}

    \emph{Time Difference of Arrival (TDoA)} řeší problém přesnosti časové základny uživatele u systémů ToA. Signál vysílán dvěma nebo více synchronizovanými majáky. Uživatel vyhodnocuje rozdíl časů příchodu signálu od majáků. Apliakce např. v Loran C.

    \emph{Doppler Shift (DS)} měří vzájemnou rychlost uživatele a majáku. Aplikace v GNSS pro stanovení vektoru rychlosti.
    
    \item \textbf{Dráhy družice LEO, MEO, GEO, IGSO}
    \begin{itemize}
        \item \emph{Low Earth Orbit (LEO):} 500 až 1200 km, primárně pro vědecké účely, imaging a telekomunikační služby. Nová generace LEO satelitů je určena pro potřeby komunikačního trhu jako broadband internet.
        \item \emph{Medium Earth Orbit (MEO):} 5000 až 20000 km, pokrytí pro vysokokapacitní přenosy na těžko dostupných místech, dříve pro GNSS.
        \item \emph{Geostationary Earth Orbit (GEO):} 36000 km, geostacionární orbita pro TV, nízkokapacitní komunikaci a přenos dat.
        \item \emph{Inclined Geostationary Earth Orbit (IGSO):} 37000 km, geostacionární s inklinací orbity.
    \end{itemize}
    
    \item \textbf{Dálkoměrná metoda výpočtu polohy.} Neznámá poloha uřivatele $(x_u,y_u,z_u)$, poloha $k$-té družice $(x_k,y_k,z_k)$, která v čase $t_c$ vysílá dálkoměrný signál. Uživatel přijímá signál v čase $t_{r,k}$ v časové základně přijímače, která se liší od časové základny vysílače systému o $\tau_c$. Lze vypočítat zdánlivou vzdálenost $\rho_k = (t_{r,k}-t_t)c$, přičemž skutečná vzdálenost je
    \begin{align}
        r_k = (t_{r,k}-t_t+\tau_c)c \overset{!}{=} \sqrt{(x_u-x_k)^2+(y_u-y_k)^2+(z_u-z_k)^2}.
    \end{align}
    
    \item \textbf{Souřadnicové systémy LLH, ECEF, ECI, NEU}
    \begin{itemize}
        \item \emph{Longitude, Latitude, Height (LLH):} geodetický souřadnicový systém, kde je zemské těleso reprezentováno rotačním elipsoidem a poloha bodu se udává pomocí geodetické zeměpisné šířky, geodetické zeměpisné délky a geodetické výšky.
        \item \emph{Earth-Centered Earth-Fixed (ECEF):} Pevně svázán se Zemí
        \item \emph{Earth-Centered Intertial (ECI):} Inerciální systémy, tj. platí v nich Newtonovy pohybové zákony a je zde známá rychlost šíření signálu.
        \item \emph{East, North, Up (ENU):} Zástupce lokálních souřadnicových systémů s nutným referenčním bodem. Aplikace: lokální souřadnice, rychlost pohybu v horizontální rovině, stoupání a klesání, azimut a elevace navigačních družic.
    \end{itemize}
    
    \item \textbf{Elipsoid, geoid, výška nad elipsoidem a geoidem}
    \begin{itemize}
        \item Elipsoid: rotační těleso eliptické ho průřezu charakterizováno délkami hlavní a vedlejší poloosy, excentricitou a druhou excentricitou.
        \item Geoid: reálné zemské těleso, které modelujeme referenčním elipsoidem, který nezahrnuje členitost zemského povrchu.
        \item Výška nad elipsoidem a geoidem: geodetická výška nad elipsoidem je součtem výšky nad geoidem a výšky geoidu, která udává, o kolik je v daném místě geoid vyšší než referenční elipsoid.
    \end{itemize}
    Referenční elipsoidy: WGS 84 (GPS, NATO aj.), PZ-90.02 (GLONASS), Besselův aj.
    
    \item \textbf{Rovnice dráhy družice} je rovnice rovnováhy gravitační síly a síly podle II. Newtonova pohybového zákona. Jejím řešením jsou dráhy tvaru kuželoseček.
    \begin{align}
        \frac{\d^2 \vec r}{\d t^2} &= -\frac{GM\vec r}{r^3}.
    \end{align}
    
    \item \textbf{Keplerovské parametry dráhy družice} je sada 6 parametrů popisujících dráhu družice a rozděluje se do dvou následujících skupin.
    \begin{enumerate}[label=\Roman*.]
        \item orientace oběžné roviny vůči Zemi:
        \begin{itemize}
            \item inklinace oběžné dráhy družice $k$,
            \item zeměpisná délka vzestupného uzlu (rektascenze) $\Omega$,
            \item argument perigea $\omega$.
        \end{itemize}
        \item tvar dráhy družice:
        \begin{itemize}
            \item délka hlavní poloosy oběžné dráhy $a$,
            \item excentricita oběžné dráhy $e$,
            \item čas průchodu perigeem $t_p$.
        \end{itemize}
    \end{enumerate}
    
    \item \textbf{Časy používané v GNSS, atomový a astronomický čas, přestupná sekunda, čas družice, čas navigačního systému, čas uživatele}
    \begin{itemize}
        \item Atomový čas: UTC (Coordinated Universal Time)
        \item Astronomický čas: GMT (Greenwich Mean Time)
    \end{itemize}
    K disparitě časů dochází kvůli tomu, že rotace Země se zpomaluje, a tak se GMT spomaluje za UTC. Řeší se to vkládáním přestupných sekund do UTC, a to podle potřeby o půlnoci z 30. června na 1. července a půnoci z 31. prosince na 1. ledna.\\
    Družice je vybavena přesnými atomovými hodinami. Odchylku hodin družice měří řídící segment a počítá korekční koeficienty.
    
    \item \textbf{Relativistické jevy a jejich řešení.} Podle teorie relativity je navigační úloha zatížena hned třemi následujícími efekty.
    \begin{itemize}
        \item \emph{Pohyb družice GNSS} má vůči pozorovateli na Zemi rychlost zhruba $3300\ \mathrm m \cdot \mathrm s^{-1}$, což způsobuje zpomalení toku jejího času.
        \item \emph{Gravitace,} respektive rozdíl gravitačního potenciálu, je dalším faktorem ovlivňujícím plynutí času. Jelikož gravitační potenciál je na Zemi zhruba 4krát vyšší, čas na Zemi vlivem tohoto efektu plyne pomaleji než na družici.
        \item \emph{Šíření signálu,} zejména jeho rychlost, je známé pouze v inerciálních vztažných soustavách. Rotace Země způsobuje, že souřadnicová soustava ECEF inerciální není, a tak je nutné výpočet polohy provádět po transformaci do libovolné soustavy ECI.
    \end{itemize}
    
    \item \textbf{Výpočet polohy metodou nejmenších čtverců, váhovnou metodou nejmenších čtverců, Kálmánova filtrace, chyba polohy, koeficienty DOP.} Dálkoměrná metoda nám poskytuje akvizici zdánlivé rychlosti $\rho_k = (t_{r,k}-t_k)c$, kde $t_{r,k}$ je čas příjmu signálu $k$-té družice v časové základně přijímače a $t_k$ je čas vyslání signálu k-té družice v časové základně navigačního systému.
    
    Skutečná vzdálenost $r_k = \rho_k + \tau_c c$, kde $\tau_c$ je časový posun mezi časovou základnou přijímače a časem navigačního systému, musí odpovídat geometrii úlohy, a tak můžeme formulovat rovnici měření
    \begin{align}
        \rho_k = \sqrt{(x_u-x_k)^2+(y_u-y_k)^2+(z_u-z_k)^2} - \tau_c c,
    \end{align}
    kde $k = 1,2,\dots$. Pro tuto metodu existuje neefektivní přímá Bancroftova metoda, ale moderní způsob řešení je zprostředkováván numericky (Newtonova-Gaussova metoda). Pro numerické řešení úlohu linearizujeme v bodě s předpokládanou polohou pomocí rozvoje do Taylorovy řady. Úloha pak nabývá tvaru
    \begin{align}
        \Delta \vec \rho = H \Delta \vec x,
    \end{align}
    kde $H$ je matice směrových kosinů a $\Delta \vec x$ je čtyřrozměrný vektor odchylky od skutečné polohy. Tato formulace je vhodná pro metodu nejmenších čtverců, která je ve váhované podobě implementována pro efektivní řešení:
    \begin{align}
        \Delta \vec x &= (H^TWH)^{-1}H^TW\Delta \vec \rho.
    \end{align}

    Dokonalejším způsobem řešení navigační úlohy je Kálmánova filtrace využívající stavového modelu a modelu měření. Kálmánův filtr počítá s historií, zohledňuje polohy, které byly vypočítány v předchozích časových okamžicích.

    Koeficienty DOP vyjadřují, kolikrát je směrodatná odchylka chyby polohy větší než je směrodatná odchylka měření zdánlivé vzdálenosti $\sigma$. Dělíme je na \emph{Geometric Dilution of Precision (GDOP)}, \emph{Position Dilution of Precision (PDOP)}, \emph{Horizontal Dilution of Precision (HDOP)}, \emph{Vertical Dilution of Precision (VDOP)} a \emph{Time Dilution of Precision (VDOP)}.
    
    \item \textbf{Signály používané v GNSS (BPSK, BOC, spektrum a korelační vlastnosti, datový a pilotní signál, dálkoměrné kódy).}
    Signál používaný pro navigaci musí být schopen přesného měření doby příchodu (ToA) a mít možnost přenášet navigační zprávu.
    \begin{itemize}
        \item BPSK (Binary Phase-Shift Keying) modulace: C/A kódy GPS
        \begin{itemize}
            \item Datový signál (komplexní obálka) $s_d(t) = \sqrt Pd(t)c(t)$, kde $d$ je signál přenášející navigační zprávu a $c$ je dálkoměrný kód.
            \item Pilotní signál $s_p(t) = \sqrt Pc(t)$.
            \item Autokorelační funkce: trojúhelník kolem nuly, nula jinde.
            \item Spektrální výkonová hustota: ostrá vzorkovací funkce.
        \end{itemize}

        \item BOC%
            \footnote{Centrální myšlenka BOC modulace je redukování interference s BPSK signálem, který má spektrum ve tvaru vzorkovací funkce, tj. většina jeho spektrální energie je soustředěna kolem nosné frekvence. Mezitím BOC signály mají na nosné nízkou energii a dva hlavní spektrální laloky odsunuté mimo nosnou (proto název \emph{split-spectrum}).}
        (Binary Offset Carrier) modulace: GPS, Galileo
        \begin{itemize}
            \item Datový signál (komplexní obálka) $s_d(t) = \sqrt Pd(t)c(t)\operatorname{sign}\(\sin\(\pi f_St\)\)$, kde $d$ je signál přenášející navigační zprávu a $c$ je dálkoměrný kód.
            \item Pilotní signál asi zase jen bez navigační zprávy.
            \item Autokorelační funkce: ostrý trojúhelník kolem nuly přesahující pod osu $x$ s následným návratem na nulu jinde.
            \item Spektrální výkonová hustota: dva BPSK signály zrcadlově kolem nosné, kde je propad mezi hlavními dvěma laloky.
        \end{itemize}
    \end{itemize}

    Dálkoměrný kód má charakter tzv. pseudonáhodného šumu PRN (Pseudo Random Noise), což je deterministická posloupnost, jejíž vlastnosti je blíží náhodné posloupnosti. Používají se následující.
    \begin{itemize}
        \item Posloupnost maximální délky: posloupnost maximální délky $2^m-1$ generovaná ireducibilním (primitivním) polynomem řádu $m$. Autokorelační funkce nabývá dvou hodnot.
        \item Goldovy kódy: generované jako součet dvou různých posloupností maximální délky. Lze je použít v CDMA. Vzájemná korelační funkce preferovaných posloupností nabývá tří hodnot.
    \end{itemize}
    
    \item \textbf{Sledování signálu, DLL, PLL, FLL, \sout{chyby měření, stabilita}.} Cílem zpracování navigačních signálů je odhad doby příchodu signálu (zpoždění modulační obálky), odhad frekvence a fáze nosné vlny, demodulace navigační zprávy a odhad kvalitativních parametrů jako je odstup nosné od šumu, intenzitu mnohocestného šíření apod. Tento odhad realizuje modelováním komunikačního kanálu jako ML (Maximum Likelihood) estimátoru, tj.
    \begin{align}
        \hat\theta &\in \argmax_{\theta}L,
    \end{align}
    kde $\theta$ je uspořádaná $n$-tice odhadovaných parametrů, $\hat\theta$ učiněný odhad a $L$ je věrohodnostní funkce. Opět existují formy přímého odhadu používané pro počáteční synchronizaci či tzv. \emph{snapshot navigation}, ale přední je iterativní forma odhadu pomocí sledovacích smyček DLL, PLL a FLL. Obecně smyčky vytváří odhad předpisem
    \begin{align}
        \hat\theta[k+1] &= \hat\theta[k] + C\[\mu\(\hat\theta[k]\)\],
    \end{align}
    kde $C$ je filtr smyčky a $\mu$ je detektor smyčky.
    \begin{itemize}
        \item \emph{Delay Locked Loop (DLL)} je použita pro odhad a sledování časového zpoždění přijatého signálu oproti lokálně generované replice. Základní princip je porovnávání přijatého signálu s replikou, kterou iterativně časově posouvá, dokud nedosáhne dostatečné shody. Ve smyčce figuruje diskriminátor zpoždění, který vyhodnocuje zpoždění mezi signály, a filtr smyčky, který výstup diskriminátoru použije pro generování řídícího signálu, pomocí něhož upraví zpoždění v generátoru repliky signálu.
        \item \emph{Phase Locked Loop (PLL)} je využívána pro odhad a sledování fáze přijatého signálu a jejího zpoždění oproti lokálně generované replice. Tento rozdíl je opět kvantifikován v diskriminátoru fáze a nadále použit pro řízení generátoru lokální repliky v další iteraci. V tomto případě se může jednat o klasické PLL jakožto kaskádu fázového detektoru, dolní propusti a VCO.
        \item \emph{Frequency Locked Loop (FLL)} je použita pro odhad a sledování frekvence přijatého signálu a jejího zpoždění oproti lokálně generované replice. Opět využívá diskriminátoru frekvence a filtru, která řídí generátor příští repliky signálu.
    \end{itemize}
    Existuje mnoho druhů jednotlivých diskriminátorů, každý typ přináší jisté výhody i nevýhody, a s tím i aplikovatelnost dle účelu. Filtry smyček lze rozdělit podle jejich řádů. Používají se filtry prvního, druhého a třetího řádu a řád filtru zároveň odpovídá řádu derivace polohy, na který je nejvíce citlivý: Filtry prvního řádu jsou citlivé na rychlost, druhého řádu na zrychlení apod.
    
    \item \textbf{Akvizice, sériové a paralelní metody.} Velikost prostoru, který prohledáváme, záleží na apriorní informaci (almanach). V případě, že absentuje např. v případě studeného startu přijímače, je třeba prohledávat celý prostor. Velikost prostoru ve zpoždění je přímo perioda dálkoměrného kódu a ve frekvenci je to závislé na několika faktorech (jako např. maximální radiální rychlost, maximální chyba frekvenčního normálu přijímače aj.), ale běžný mobilní uživatel na L1 se pohybuje cca na 8~kHz. Tento prostor je vzorkován po diskrétních časových a frekvenčních krocích a následně prohledáván.
    \begin{itemize}
        \item \emph{Sériové prohledávání:} Využívá standardního IQ korelátoru, porovnávání modulu korelační funkce a pravděpodobnostní analýzy. Tento způsob byl vyvinut v dobách, kdy na paralelní vyhledávání nebyly prostředky, a v dnešní době není používán.
        \item \emph{Paralelní vyhledávání:} Možnosti paralelního vyhledávání v časové doméně, frekvenční doméně či 2D paralelní vyhledávání v obou doménách zároveň.
    \end{itemize}
    \item \textbf{Navigační systémy GPS, GLONASS, Galileo, Compass, podpůrné systémy WAAS, EGNOS, QZSS.}
    \begin{itemize}
        \item \emph{Global Positioning System (GPS):} Projekt zprvu začal jako sloučení navigačních projektů letectva, námořnictva a armády USA. Později se uvolnila služba i pro civilní sektor. Systém je rozložen do následujících tří segmentů.
        \begin{itemize}
            \item Kosmický segment: konstelace 32 družic v 6 oběžných rovinách.
            \item Řídící segment: Tento segment, složený z hlavní řídící stanice, záložní řídící stanice, pozemních antén a monitorovací stanice, slouží především k měření drah navigačních družic, monitorování a údržba jejich funkčnosti, udržování GPS času, výpočet a nahrávání navigační zprávy, omezené manévrování aj.
            \item Uživatelský segment
        \end{itemize}
        Navigační zpráva GPS obsahuje efemeridy vyjadřující přesné parametry dráhy družice, přesný čas a parametry pro korekci hodin družice, servisní parametry jako je stav družice, parametry ionosférického modelu a almanach systémů, ve kterém jsou obsaženy nepřesné parametry drah všech družic systému.

        \item \emph{GLONASS:} Ruská varianta GPS, která během vývoje trpěla na neduhy vývoji ruských technologií blízké jako například \uv{úpadek systému v důsledku ekonomických potíží Ruska}, naivní investive do kvantity nad kvalitu technologie apod.\\
        Konstelace GLONASS sestává ze 24 družic v 3 oběžných rovinách.

        \item \emph{Galileo:} GPS variant Evropské unie. Zpočátku financována soukromými investory, což o pár let později zkrachovalo, a o rok později se EU rozhodla projekt financovat z vlastních zdrojů.\\
        Konstelace Galileo sestává ze 30 družic v 3 oběžných rovinách. Hovoří se o rozšíření konstelace o geostacionární družice podpůrného systému EGNOS.

        \item \emph{Compass (dříve BeiDou):} Čínská varianta GPS, která v plné konstelaci čítá 3 družice na IGSO a 27 družic na MEO.
        
        \item \emph{Rozšiřující navigační systémy:} Navigační systémy navržené pro vylepšení přesnosti, dostupnosti a spolehlivosti navigační informace od globálních systémů. Využívají diferenčních metod měření (DGNSS) pro korekci korelovaných chyb pomocí signálů od referenční stanice a uživatelského přijímače. Mezi nejznámnější \emph{Satellite-Based Augmentation System (SBAS)} patří WAAS, EGNOS a QZSS.
    \end{itemize}
    
    \item \textbf{Principy fázových měření, RTK, PPP.}
    \begin{itemize}
        \item \emph{Precision Point Positioning (PPP):} dvoufrekvenční měření s dokonalým fyzikálním modelem šíření signálu pro zpřesňování informací o polohách navigačních družic a jejich časových základen. Data nejsou k dispozici v reálném čase, nýbrž jsou zpožděny od několika minut až po dny. Fázových měření může využívat, ale nemusí.
        \item \emph{Real Time Kinematic (RTK):} relativní měření využívající fázových měření.
    \end{itemize}
    
    \item \textbf{Aplikace softwarového rádia v GNSS, GNSS simulátory.}
    \begin{itemize}
        \item \emph{GNSS simulátory:} generátor GNSS signálu pro účely testování a vývoje GNSS přijímačů a jejich aplikací. Lze konstruovat jednokanálové simulátory, které dokáží generovat signál jedné družice, což je využitelné pro testování smyček DLL a PLL, vyhledávání signálu apod. Existuje však i možnost konstrukce vícekanálových simulátorů, kterými generujeme signály od více/všech viditelných družic a které nám tedy umožňují kompletní funkční testy GNSS přijímačů.
        \item \emph{Věda, výzkum a vývoj} 
        \item \emph{Studium GNSS} 
    \end{itemize}
    
    \item \textbf{Terestriální navigační systémy, traťová, prostorová navigace, LORAN C, NDB, VOR, DME, ILS, kategorie ILS.} Terestriální navigační systémy dělíme na \emph{hyperbolické} a \emph{letecké}. Jediným dosud dýchajícím zástupcem hyperbolických navigačních systémů je systém Long RAnge Navigation vesion C (LORAN-C), což je systém pracující na frekvenci 100~kHz, původně určen pro námočnictvo, ale později používán v letectví. Nadále se budeme bavit o leteckých navigačních systémech.
    \begin{itemize}
        \item \emph{Traťová navigace (En Route Navigation):} Systém pevných letových tratí, v jejíž uzlech jsou umístěny navigační radiomajáky buď NDB, nebo VOR (DME). V dnešní době nahrazeno prostorovou navigací.
        \begin{itemize}
            \item \emph{Non-Directional Beacon (NDB):} všesměrový maják určený k měření směru k radiomajáku. Použití pro Automatic Directional Finder (ADF) pro zaměření kurzu k radiomajáku podle směrové antény. Letová trasa je vyznačena majáky NDB a letadlo letí od majáku k majáku.
            \item \emph{VHF Omnidirectional Range (VOR):} VKV radiomaják vytyčující radiály, které umožňují let k majáku po jakékoliv radiále. VOR radiomaják vysílá dva signály (referenční a variabilní), které pak navigační přijímač vyhodnocuje pro jejich vzájemný fázový posuv.
            \item \emph{Distance Measuring Equipment (DME):} měřič vzdálenosti na základě měření doby šíření signálu. Letadlo (dotazovač) vyšle dva pulzy, pozemní stanice vyčká časový interval a odpoví také dvojící pulzů. Palubní stanice měří čas mezi vysláním a příjmem. Častá je tzv. koaxiální instalace majáků VOR v kombinaci s majákem DME, což dává dohromady kompletní navigační informaci radiála + vzdálenost.
            \item \emph{Instrument Landing System (ILS):} Skládá se ze směrového majáku, sestupového majáku a 2 až 3 markerů.
            \begin{itemize}
                \item Směrový maják: Zajišťuje směrové vedení letadla v ose přistávací dráhy. Signál se snadno odráží od překážek, což vyžaduje kalibraci majáku.
                \item Sestupový maják: Zajišťuje výškové vedení letadla po sestupové ose.
                \item Marker (vnější, střední a vnitřní): indikátor vzdálenosti od přistávací dráhy.
            \end{itemize}
        \end{itemize}
        \item \emph{Prostorová navigace RNAV (Area Navigation):} Trať je definovaná posloupností bodů, které jsou nezávislé na umístění radiomajáků. Vyžaduje navigační počítač FMS.
    \end{itemize}
    
    \item \textbf{FMS, Primary flight display.}
    \begin{itemize}
        \item \emph{Flight Management System (FMS):} navigační počítač na palubě letadla. Umožňuje zadání a modifikaci letového plánu a následně na základě VOR, DME, GPS apod. odhaduje polohu. Dále počítá odchylku od plánované trasy, určuje nový kurz a zadává jej do autopilota.
        \item \emph{Primary flight display} obashuje rychloměr, zatáčkoměr, umělý horizont, břevna ILS, výškoměr, variometr a směrový indikátor.
    \end{itemize}
    
    \item \textbf{Primární, sekundární a pasivní radar, pulzní a CW radar, radarová rovnice, závoj (clutter), Dopplerovská filtrace, sekundární radar \sout{modů A, C a S, squitter, extended squitter}, ACAS/TCAS.}\\
    Základní klasifikace radaru:
    \begin{itemize}
        \item Primární radar: aktivní vysílač, pasivní cíl;
        \item Sekundární radar: aktivní vysílač i cíl;
        \item Pasivní radar: aktivní cíl, pasivní příjem.
    \end{itemize}
    \begin{itemize}
        \item Monostatický radar: vysílač i přijímač jsou na jenom místě (společná anténa);
        \item Bistatický radar: vysílač a pijímač jsou na vzdálené bázi;
        \item Radarová síť.
    \end{itemize}
    \begin{itemize}
        \item \emph{Continuous Wave (CW) radar,} neboli Doppler radar, pracuje s VF kontinuálním signálem. Nemodulovaná varianta má konstantní amplitudu i frekvenci, a tak nelze měřit vzdálenost, pouze rychlost.
        \item \emph{Frequency Modulated CW (FMCW) radar} má konstantní amplitudu a modulovanou frekvenci (frequency sweep), což Umožňuje současné měření vzdálenosti a rychlosti. Není nutné vysílat vysokovýkonný pulz a přijatý signál se směšuje do NF.
        \item \emph{Frequency Modulated interrupted CW (FMiCW) radar} umožňuje během vysílání přerušit FM, ale ponechat funkční pro potřeby příjmu. To v okamžicích vypnutí zvyšuje citlivost přijímače a umožňuje tak vyšší provozní dosah než klasický FMCW radar.
        \item \emph{Pulse radar}
    \end{itemize}
    Základní princip radaru je popsán radarovou rovnicí:
    \begin{align}
        P_{\mathrm{RX}} = \frac{P_{\mathrm{TX}}G\sigma_{\mathrm{RCS}}}{\(4\pi R^2\)^2} \frac{\lambda^2}{4\pi}G = \frac{P_{\mathrm{TX}}G^2\sigma_{\mathrm{RCS}}\lambda^2}{(4\pi)^3R^4}
    \end{align}
    za předpokladu primárního radaru (stejný zisk vysílací a přijímací antény, stejné vzdálenosti).\\
    Vzdálenost následně určíme jako tzv. \emph{slant range}, neboli vzdálenost přímé spojnice, z časového rozdílu příchodu a vyslání signálu.\\
    Určení směru je relativní vzhledem k referenčnímu natočení antény.\\
    Radar má daný maximální dosah (Maximum Unambiguous Range), který je daný jako vzdálenost, z níž dorazí odraz před dalším vysílaním pulzu.\\
    Primární radar zároveň trpí naduhem zvaným \emph{blind range}, neboli minimální detekovatelná vzdálenost, která je způsobena konečnou rychlostí přepnutí (a tedy nenulovým časem zotavení) duplexeru z vysílání na příjem.\\
    \emph{Clutter:} zdroj nežádoucích příjmů. Může se jednat o \emph{surface clutter} jako např. odraz od Země či od mořské hladiny, \emph{volume clutter} jako např. meteorologické objekty nebo prostředky pro rušení radaru, \emph{point clutter} jako např. ptáci, větrné elektrárny aj.\\
    \emph{Dopplerovská filtrace:} odlišení pohyblivého cíle od fixního odrazu na základě Dopplerova posuvu. Mluvíme o \emph{Moving Target Indication (MTI)} a \emph{Moving Target Detection (MTD)}. Dopplerovská filtrace má jednu výraznou nevýhodu, a to fakt, že vždy existuje tzv. \emph{slepá rychlost (blind speed)}, a níž dochází k nejednoznačnosti určení.\\
    \emph{Sekundární radar:} Pracuje na bázi obousměrné aktivity dotaz--odpověď. Musí se vypořádávat s mnoha problémy jako je více odpovědí, garbling apod.\\
    \emph{Mód S sekundárního radaru:} Zvýšení bezpečnosti (mód A - jen 4096 kódů), selektivní adresace konkrétního letadla, zvýšení integrity dat pomocí redundance (parity check).\\
    \emph{ACAS/TCAS (Airborne Collision Avoidance System/Traffic advisory and Collision Avoidance System):} Odvozené systémy od sekundárního radaru módu S, kdy dotazovač je na palubě letadel a vyhodnocuje se poloha okolních letadel a rizika kolize. Systém případně doporučí manévr.
\end{enumerate}

\end{document}