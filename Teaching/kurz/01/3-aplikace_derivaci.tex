\documentclass[11pt,a4paper]{report}

\setlength\textwidth{145mm}
\setlength\textheight{247mm}
\setlength\oddsidemargin{15mm}
\setlength\evensidemargin{15mm}
\setlength\topmargin{0mm}
\setlength\headsep{0mm}
\setlength\headheight{0mm}
\let\openright=\clearpage

\usepackage[czech]{babel}
\usepackage{lmodern}
\usepackage[T1]{fontenc}
\usepackage{textcomp}

\usepackage[utf8]{inputenc}

\usepackage{stddoc}


\renewcommand{\vec}{\boldsymbol}
\def\endl{\\[3mm]}
\newcommand*\colvec[3][]{
	\begin{pmatrix}
		\ifx \relax#1 \relax
		\else #1\\
		\fi
		#2 \\ #3
	\end{pmatrix}
}



\begin{document}
	
	\pagenumbering{gobble}
	
	\section*{Aplikace derivace v kinematice}
	
	\noindent
	Minule jsme se dozvěděli o Větě o aritmetice derivací (zkráceně VoAD, vizme \eqref{eq:voad}) a o Větě od derivaci složené funkce (zkráceně VoDSF, vizme \eqref{eq:vodsf})
	\begin{equation}
		\label{eq:voad}
		\begin{aligned}
			(f \pm g)' &= f' + g',
		\\
			\Aboxed{(f \cdot g)' &= f' \cdot g + f \cdot g',}
		\\
			\( \frac{1}{g} \)' &= \frac{-g'}{g^2},
		\end{aligned}
	\end{equation}
	\begin{align}
		\label{eq:vodsf}
		\Aboxed{(f(g(x)))' &= f'(g(x)) \cdot g'(x).}
	\end{align}
	a také jak počítat derivace podle definice. Ačkoli se jedná o jednu z náročnějších elementárních funkcí na derivování, odvodíme funkci sinus, protože se nám bude dnes hodit (celý postup je uveden pouze pro zajímavost, o podobných výpočtech limit se budete učit v rámci prvního kursu Matematické analýzy)
	\begin{align*}
		(\sin(x))' &= \lim_{y \to x} \frac{\sin(y) - \sin(x)}{y-x} = \lim_{y \to x} \frac{2 \sin\(\frac{y-x}{2}\)\cos\(\frac{y+x}{2}\)}{y-x} =
	\\
		&= \lim_{y \to x} \( \frac{\sin\(\frac{y-x}{2}\)}{\frac{y-x}{2}} \) \cdot \lim_{y \to x} \( \cos\(\frac{y+x}{2}\) \) = 1 \cos\( \frac{2x}{2} \) =
	\\
		&=\cos(x).
	\end{align*}
	Funkce kosinus má odvození podobné, proto ho již neuvádíme. Každopádně výsledek této práce, který se dnes ukáže dobrým nástrojem, je
	\begin{align}
		\label{eq:sinder}
		(\sin(x))' &= \cos(x),
	\\
		\label{eq:cosder}
		(\cos(x))' &= - \sin(x).
	\end{align}
	
	Dnes se tedy pokusíme o určitou aplikaci nově nabitých znalostí (zarámečkovaných rovností obzvlášť) do oblasti mechaniky. Začneme tedy přirozeně s rychlostí. Vzorec pro rychlost je základním kamenem kinematiky, partie mechaniky, která se zabývá popisem pohybu bez toho, abychom se příliš starali o jeho příčinu. Vztah rychlosti a polohy znám ze středoškolské fyziky i jeho vyjádření v jazyce nám již známých derivací je tedy
	\begin{align*}
		\overline v(t) &= \frac{\Delta s}{\Delta t} = \frac{s_B - s_A}{t_B-t_A},
	\\
		v(t) &= \lim\limits_{\Delta t \to 0} \frac{\Delta s}{\Delta t} = \frac{\d s}{\d t}(t) = s'(t),
	\\
		a(t) &= \lim\limits_{\Delta t \to 0} \frac{\Delta v}{\Delta t} = \frac{\d v}{\d t}(t) = v'(t) = s''(t),
	\end{align*}
	kde $\overline v(t)$ (středoškolsky definovaná rychlost) je tedy pouze rychlostí průměrnou za daný časový interval, kdežto rychlost $v(t)$ zadefinována pomocí derivace je rychlost okamžitá pro daný čas $t$.
	
	\noindent
	\textit{Připomínka.} \quad \underline{Polární souřadnice} kartézsky vyjádřeného bodu $[x,y]$ můžeme určit dle obrázku níže jako
	\begin{align*}
	x = r \cos(\varphi),
	\\
	y = r \sin(\varphi).
	\end{align*}
	
	\begin{figure}[h!] \begin{center}
			\begin{tikzpicture}
			
			%% iniciace: nakres souradnicoveho krize
			\draw[black,thick,->] (-3.5,0)--(3.5,0) node[black] at(3.15,-0.35){$x$};
			\draw[black,thick,->] (0,-2.5)--(0,2.5) node[black] at(-0.35,2.15){$y$};
			%\filldraw[black] (0,0) circle(1pt)
			\node[black] at(0.25,-0.3){0};
			
			%% bod P a jeho souradnice (kartezsky i polarni)
			\filldraw[gray] (2,1) circle(2pt) node[black] at(2.4,0.6){P};
			\draw[gray,thin] (0,0)--(2,1) node[black] at(0.85,0.65){$r$};
			\draw[gray,dashed] (2,0)--(2,1) node[black] at(2,-0.35){$x_P$};
			\draw[gray,dashed] (0,1)--(2,1) node[black] at(-0.35,1){$y_P$};
			\draw[black,thin,->] (1.5,0) arc(0:25:1.5) node[black] at(1.15,0.25){$\varphi$};
			
			\end{tikzpicture}
			\caption{Ilustrace polárních souřadnic v kartézské souřadné soustavě}
	\end{center} \end{figure}
	
	Takto tedy můžeme určit polohu bodu stojícího v prostoru. My se však zabýváme obecně pohybem, který nemusí být vůči nám (pozorovateli) v klidu, ale jeho poloha bude záviset na čase. V případě polohy budou tedy obě souřadnice funkcemi času
	\begin{align*}
		x(t) = r(t) \cos(\varphi(t)),
	\\
		y(t) = r(t) \sin(\varphi(t))
	\end{align*}
	Rychlost tedy, jak jsme si ukázali již dříve, je časovou derivací právě funkce polohy, můžeme tedy psát
	\begin{align*}
		v_x(t) &= (r(t) \cos(\varphi(t)))',
	\\
		v_y(t) &= (r(t) \sin(\varphi(t)))'.
	\end{align*}
	Nasasdíme-li tedy na takto obecně vyjádřený pohyb nám již známé věty o derivacích \eqref{eq:voad}, \eqref{eq:vodsf} a znalost derivace funkce kosinus \eqref{eq:cosder}, x-ová složka rychlosti bude vypadat
	\begin{align*}
		v_x(t) &= (s_x(t))' = (r(t) \cdot \cos(\varphi(t)))' = (r(t))' \cdot \cos(\varphi(t)) + r(t) \cdot (\cos(\varphi(t)))' =
	\\
		&= r'(t) \cos(\varphi(t)) + r(t) ((-\sin(\varphi(t)) \varphi'(t)) = r'(t) \cos(\varphi(t)) - r(t) \sin(\varphi(t)) \varphi'(t),
	\end{align*}
	
	\noindent
	y-ová složka rychlosti pak zase s pomocí \eqref{eq:voad}, \eqref{eq:vodsf}, akorát tentokrát se nám vyskytne ve vyjádření sinus, takže použijeme \eqref{eq:sinder}
	\begin{align*}
		v_y(t) &= (r(t) \sin(\varphi(t)))' = r'(t) \sin(\varphi(t)) + r(t) (\sin(\varphi(t)))' =
	\\
		&= r'(t) \sin(\varphi(t)) + r(t) \cos(\varphi(t)) \varphi'(t).
	\end{align*}
	Celkovou rychlost v polárních souřadnicích pak můžeme vyjádřit jako (na universitě již sloupcový) vektor a dostáváme tak ucelený výsledek
	\begin{align*}
		\Aboxed{\vec v(t) = \begin{pmatrix}
			v_x(t) \\
			v_y(t)
		\end{pmatrix} = \begin{pmatrix}
			r'(t) \cos(\varphi(t)) - r(t) \sin(\varphi(t)) \varphi'(t) \\
			r'(t) \sin(\varphi(t)) + r(t) \cos(\varphi(t)) \varphi'(t)
		\end{pmatrix}.}
	\end{align*}
	
		
	
\end{document}